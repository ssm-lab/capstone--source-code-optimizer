\documentclass[12pt, titlepage]{article}

\usepackage{booktabs}
\usepackage{tabularx}
\usepackage{hyperref}
\hypersetup{
    colorlinks,
    citecolor=blue,
    filecolor=black,
    linkcolor=red,
    urlcolor=blue
}

%Includes "References" in the table of contents
\usepackage[nottoc]{tocbibind}

\usepackage[toc,page]{appendix}

\usepackage[square,numbers,compress]{natbib}
\bibliographystyle{abbrvnat}
\usepackage{amssymb}

%% Comments

\usepackage{color}

% \newif\ifcomments\commentstrue %displays comments
\newif\ifcomments\commentsfalse %so that comments do not display

\ifcomments
\newcommand{\authornote}[3]{\textcolor{#1}{[#3 ---#2]}}
\newcommand{\todo}[1]{\textcolor{red}{[TODO: #1]}}
\else
\newcommand{\authornote}[3]{}
\newcommand{\todo}[1]{}
\fi

\newcommand{\wss}[1]{\authornote{blue}{SS}{#1}} 
\newcommand{\plt}[1]{\authornote{magenta}{TPLT}{#1}} %For explanation of the template
\newcommand{\an}[1]{\authornote{cyan}{Author}{#1}}

%% Common Parts

\newcommand{\progname}{Software Engineering} % PUT YOUR PROGRAM NAME HERE
\newcommand{\authname}{\textbf{Team 4, EcoOptimizers} \\
  \\ Nivetha Kuruparan
  \\ Sevhena Walker
  \\ Tanveer Brar
  \\ Mya Hussain
\\ Ayushi Amin} % AUTHOR NAMES

\usepackage{hyperref}
\hypersetup{colorlinks=true, linkcolor=blue, citecolor=blue, filecolor=blue,
urlcolor=blue, unicode=false}
\urlstyle{same}



\newcommand{\SRS}{\href{https://github.com/ssm-lab/capstone--source-code-optimizer/blob/main/docs/SRS/SRS.pdf}{SRS}}
\newcommand{\MG}{\href{https://github.com/ssm-lab/capstone--source-code-optimizer/blob/main/docs/Design/SoftArchitecture/MG.pdf}{MG}}
\newcommand{\MIS}{\href{https://github.com/ssm-lab/capstone--source-code-optimizer/blob/main/docs/Design/SoftDetailedDes/MIS.pdf}{MIS}}

\begin{document}

\title{System Verification and Validation Plan for \progname{}} 
\author{\authname}
\date{\today}
	
\maketitle

\pagenumbering{roman}

\section*{Revision History}

\begin{tabularx}{\textwidth}{p{3cm}p{2cm}X}
\toprule {\bf Date} & {\bf Version} & {\bf Notes}\\
\midrule
Date 1 & 1.0 & Notes\\
Date 2 & 1.1 & Notes\\
\bottomrule
\end{tabularx}

~\\
\wss{The intention of the VnV plan is to increase confidence in the software.
However, this does not mean listing every verification and validation technique
that has ever been devised.  The VnV plan should also be a \textbf{feasible}
plan. Execution of the plan should be possible with the time and team available.
If the full plan cannot be completed during the time available, it can either be
modified to ``fake it'', or a better solution is to add a section describing
what work has been completed and what work is still planned for the future.}

\wss{The VnV plan is typically started after the requirements stage, but before
the design stage.  This means that the sections related to unit testing cannot
initially be completed.  The sections will be filled in after the design stage
is complete.  the final version of the VnV plan should have all sections filled
in.}

\newpage

\tableofcontents

\listoftables
\wss{Remove this section if it isn't needed}

\listoffigures
\wss{Remove this section if it isn't needed}

\newpage

\section{Symbols, Abbreviations, and Acronyms}

\renewcommand{\arraystretch}{1.2}
\begin{tabular}{l l} 
  \toprule		
  \textbf{symbol} & \textbf{description}\\
  \midrule 
  T & Test\\
  \bottomrule
\end{tabular}\\

\wss{symbols, abbreviations, or acronyms --- you can simply reference the SRS
  \cite{SRS} tables, if appropriate}

\wss{Remove this section if it isn't needed}

\newpage

\pagenumbering{arabic}

This document ... \wss{provide an introductory blurb and roadmap of the
  Verification and Validation plan}

\section{General Information}

\subsection{Summary}

The software being tested is called EcoOptimizer. EcoOptimizer is a python refactoring library that focuses on optimizing code in a way that reduces its energy consumption. The system will be capable to analyze python code in order to spot inefficiencies (code smells) within, measuring the energy efficiency of the inputted code and, of course, apply appropriate refactorings that preserve the initial function of the source code. \\

Furthermore, peripheral tools such as a Visual Studio Code (VS Code) extension and GitHub Action are also to be tested. The extension will integrate the library with Visual Studio Code for a more efficient development process and the GitHub Action will allow a proper integration of the library into continuous integration (CI) workflows.

\subsection{Objectives}

The primary objective of this project is to build confidence in the \textbf{correctness} and \textbf{energy efficiency} of the refactoring library, ensuring that it performs as expected in improving code efficiency while maintaining functionality. Usability is also emphasized, particularly in the user interfaces provided through the \textbf{VS Code extension} and \textbf{GitHub Action} integrations, as ease of use is critical for adoption by software developers. These qualities—correctness, energy efficiency, and usability—are central to the project’s success, as they directly impact user experience, performance, and the sustainable benefits of the tool.\\

Certain objectives are intentionally left out-of-scope due to resource constraints. We will not independently verify external libraries or dependencies; instead, we assume they have been validated by their respective development teams. 

\subsection{Challenge Level and Extras}

Our project, set at a \textbf{general} challenge level, includes two additional focuses: \textbf{user documentation} and \textbf{usability testing}. The user documentation aims to provide clear, accessible guidance for developers, making it easy to understand the tool’s setup, functionality, and integration into existing workflows. Usability testing will ensure that the tool is intuitive and meets user needs effectively, offering insights to refine the user interface and optimize interactions with it's features.

\subsection{Relevant Documentation}

The Verification and Validation (VnV) plan relies on three key documents to guide testing and assessment: 
\begin{itemize}
  \item[] \textbf{Software Requirements Specification (\SRS)\cite{SRS}:} The foundation for the VnV plan, as it defines the functional and non-functional requirements the software must meet; aligning tests with these requirements ensures that the software performs as expected in terms of correctness, performance, and usability.
  
  \item[] \textbf{Module Interface Specification (\MG)\cite{MGDoc}:} Provides detailed information about each module's interfaces, which is crucial for integration testing to verify that all modules interact correctly within the system.
  
  \item[] \textbf{Module Guide (\MIS)\cite{MISDoc}:} Outlines the system's architectural design and module structure, ensuring the design of tests that align with the intended flow and dependencies within the system.
\end{itemize}

\section{Plan}

\wss{Introduce this section.  You can provide a roadmap of the sections to
  come.}

\subsection{Verification and Validation Team}

\wss{Your teammates.  Maybe your supervisor.
  You should do more than list names.  You should say what each person's role is
  for the project's verification.  A table is a good way to summarize this information.}

\subsection{SRS Verification Plan}

\textbf{Function \& Non-Functional Requirements:}
\begin{itemize}
    \item A comprehensive test suite that covers all requirements specified in the SRS will be created.
    \item Each requirement will be mapped to specific test cases to ensure maximum coverage.
    \item Automated and manual testing will be conducted to verify that the implemented system meets each functional requirement.
    \item Usability testing with representative users will be carried out to validate user experience requirements and other non-functional requirements.
    \item Performance tests will be conducted to verify that the system meets specified performance requirements.
\end{itemize}

\textbf{Traceability Matrix:}
\begin{itemize}
    \item We will create a requirements traceability matrix that links each SRS requirement to its corresponding implementation, test cases, and test results.
    \item This matrix will help identify any requirements that may have been overlooked during development.
\end{itemize}

\textbf{Supervisor Review:}
\begin{itemize}
    \item After the implementation of the system, we will conduct a formal review session with key stakeholders such as our project supervisor, Dr. Istvan David.
    \item The stakeholders will be asked to verify that each requirement in the SRS is mapped out to specific expectations of the project. 
    \item Prior to meeting, we will provide a summary of key requirements and design decisions and prepare a list specific questions or areas where we seek guidance.
    \item During the meeting, we will present an overview of the SRS using tables and other visual aids. We will conduct a walk through of critical section. Finally, we will discuss any potential risks or challenges identified.
\end{itemize}

\textbf{User Acceptance Testing (UAT):}
\begin{itemize}
    \item We will involve potential end-users in testing the system to ensure it meets real-world usage scenarios.
    \item Feedback from UAT will be used to identify any discrepancies between the SRS and user expectations.
\end{itemize}

\textbf{Continuous Verification:}
\begin{itemize}
    \item Throughout the development process, we will regularly review and update the SRS to ensure it remains aligned with the evolving system.
    \item Any changes to requirements will be documented and their impact on the system assessed.
\end{itemize}

\textbf{\textit{\\Checklist for SRS Verification Plan}}
\begin{itemize}
    \item[$\square$] Create comprehensive test suite covering all SRS requirements
    \item[$\square$] Map each requirement to specific test cases
    \item[$\square$] Conduct automated testing for functional requirements
    \item[$\square$] Perform manual testing for functional requirements
    \item[$\square$] Carry out usability testing with representative users
    \item[$\square$] Conduct performance tests to verify system meets requirements
    \item[$\square$] Create requirements traceability matrix
    \item[$\square$] Link each SRS requirement to implementation in traceability matrix
    \item[$\square$] Link each SRS requirement to test cases in traceability matrix
    \item[$\square$] Link each SRS requirement to test results in traceability matrix
    \item[$\square$] Schedule formal review session with project supervisor
    \item[$\square$] Prepare summary of key requirements and design decisions for supervisor review
    \item[$\square$] Prepare list of specific questions for supervisor review
    \item[$\square$] Create visual aids for SRS overview presentation
    \item[$\square$] Conduct walkthrough of critical SRS sections during review
    \item[$\square$] Discuss potential risks and challenges with supervisor
    \item[$\square$] Organize User Acceptance Testing (UAT) with potential end-users
    \item[$\square$] Collect and analyze UAT feedback
    \item[$\square$] Identify discrepancies between SRS and user expectations from UAT
    \item[$\square$] Establish process for regular SRS review and updates
    \item[$\square$] Document any changes to requirements
    \item[$\square$] Assess impact of requirement changes on the system
\end{itemize}

\subsection{Design Verification Plan}

\textbf{Peer Review Plan:}
\begin{itemize}
    \item Each team member along with other classmates will thoroughly review the entire Design Document.
    \item A checklist-based approach will be used to ensure all key elements are covered.
    \item Feedback will be collected and discussed in a dedicated team meeting.
\end{itemize}

\textbf{Supervisor Review:}
\begin{itemize}
    \item A structured review meeting will be scheduled with our project supervisor, Dr. Istvan David.
    \item We will present an overview of the design using visual aids (e.g., diagrams, tables).
    \item We will conduct a walkthrough of critical sections.
    \item We will use our project's issue tracker to document and follow up on any action items or changes resulting from this review.
\end{itemize}

\begin{itemize}
  \item[$\square$] All functional requirements are mapped to specific design elements 
  \item[$\square$] Each functional requirement is fully addressed by the design 
  \item[$\square$] No functional requirements are overlooked or partially implemented 
  \item[$\square$] Performance requirements are met by the design 
  \item[$\square$] Scalability considerations are incorporated 
  \item[$\square$] Reliability and availability requirements are satisfied 
  \item[$\square$] Usability requirements are reflected in the user interface design
  \item[$\square$] High-level architecture is clearly defined 
  \item[$\square$] Architectural decisions are justified with rationale 
  \item[$\square$] Architecture aligns with project constraints and goals 
  \item[$\square$] All major components are identified and described 
  \item[$\square$] Interactions between components are clearly specified 
  \item[$\square$] Component responsibilities are well-defined 
  \item[$\square$] Appropriate data structures are chosen for each task 
  \item[$\square$] Efficient algorithms are selected for critical operations 
  \item[$\square$] Rationale for data structure and algorithm choices is provided
  \item[$\square$] UI design is consistent with usability requirements 
  \item[$\square$] User flow is logical and efficient 
  \item[$\square$] Accessibility considerations are incorporated 
  \item[$\square$] All external interfaces are properly specified 
  \item[$\square$] Interface protocols and data formats are defined 
  \item[$\square$] Error handling for external interfaces is addressed 
  \item[$\square$] Comprehensive error handling strategy is in place
  \item[$\square$] Exception scenarios are identified and managed 
  \item[$\square$] Error messages are clear and actionable 
  \item[$\square$] Authentication and authorization mechanisms are described 
  \item[$\square$] Data encryption methods are specified where necessary 
  \item[$\square$] Security best practices are followed in the design
  \item[$\square$] Design allows for future expansion and feature additions 
  \item[$\square$] Code modularity and reusability are considered 
  \item[$\square$] Documentation standards are established for maintainability 
  \item[$\square$] Performance bottlenecks are identified and addressed 
  \item[$\square$] Resource utilization is optimized 
  \item[$\square$] Performance testing strategies are outlined 
  \item[$\square$] Design adheres to established coding standards 
  \item[$\square$] Industry best practices are followed 
  \item[$\square$] Design patterns are appropriately applied
  \item[$\square$] All major design decisions are justified 
  \item[$\square$] Trade-offs are explained with pros and cons 
  \item[$\square$] Alternative approaches considered are documented 
  \item[$\square$] Documents is clear, concise, and free of ambiguities 
  \item[$\square$] Documents follows a logical structure 
\end{itemize}

\subsection{Verification and Validation Plan Verification Plan}

\wss{The verification and validation plan is an artifact that should also be
verified.  Techniques for this include review and mutation testing.}

\wss{The review will include reviews by your classmates}

\wss{Create a checklists?}

\subsection{Implementation Verification Plan}

\wss{You should at least point to the tests listed in this document and the unit
  testing plan.}

\wss{In this section you would also give any details of any plans for static
  verification of the implementation.  Potential techniques include code
  walkthroughs, code inspection, static analyzers, etc.}

\wss{The final class presentation in CAS 741 could be used as a code
walkthrough.  There is also a possibility of using the final presentation (in
CAS741) for a partial usability survey.}

\subsection{Automated Testing and Verification Tools}

\wss{What tools are you using for automated testing.  Likely a unit testing
  framework and maybe a profiling tool, like ValGrind.  Other possible tools
  include a static analyzer, make, continuous integration tools, test coverage
  tools, etc.  Explain your plans for summarizing code coverage metrics.
  Linters are another important class of tools.  For the programming language
  you select, you should look at the available linters.  There may also be tools
  that verify that coding standards have been respected, like flake9 for
  Python.}

\wss{If you have already done this in the development plan, you can point to
that document.}

\wss{The details of this section will likely evolve as you get closer to the
  implementation.}

\subsection{Software Validation Plan}

\wss{If there is any external data that can be used for validation, you should
  point to it here.  If there are no plans for validation, you should state that
  here.}

\wss{You might want to use review sessions with the stakeholder to check that
the requirements document captures the right requirements.  Maybe task based
inspection?}

\wss{For those capstone teams with an external supervisor, the Rev 0 demo should 
be used as an opportunity to validate the requirements.  You should plan on 
demonstrating your project to your supervisor shortly after the scheduled Rev 0 demo.  
The feedback from your supervisor will be very useful for improving your project.}

\wss{For teams without an external supervisor, user testing can serve the same purpose 
as a Rev 0 demo for the supervisor.}

\wss{This section might reference back to the SRS verification section.}

\section{System Tests}

\subsection{Tests for Functional Requirements}

The subsections below outline tests corresponding to functional 
requirements in the SRS. Each test is associated with a unique 
functional area, helping to confirm that the tool meets the 
specified requirements. Each functional area has its own subsection 
for clarity.

\subsubsection{Code Input Acceptance Tests}

This section covers the tests for ensuring the system correctly accepts 
Python source code files, detects errors in invalid files, and provides 
suitable feedback (FR 1).
		
\paragraph{1. Input Acceptance}
\begin{enumerate}
\item{Valid Python File Acceptance\\}

Control: Automatic
					
Initial State: Tool is idle.
					
Input: A valid Python file (filename.py) with valid standard syntax.
					
Output: The system accepts the file without errors.

Test Case Derivation: Confirming that the system correctly processes a valid Python file as per FR 1.
					
How test will be performed: Feed a syntactically valid .py file to the tool and observe if it’s accepted without issues.
					
\item{Feedback for Python File with Bad Syntax\\}

Control: Automatic
					
Initial State: Tool is idle.
					
Input: A .py file (badSyntax.py) containing deliberate syntax errors that render the file unrunable.
					
Output: The system rejects the file and provides an error message detailing the syntax issue.

Test Case Derivation: Verifies the tool’s handling of syntactically invalid Python files to ensure user awareness of the syntax issue, meeting FR 1.

How test will be performed: Feed a .py file with syntax errors to the tool and check that the system identifies it as invalid and produces an appropriate error message.

\item{Feedback for Non-Python File\\}

Control: Automatic
					
Initial State: Tool is idle.
					
Input: A non-Python file (document.txt) or a file with an incorrect extension (script.js).
					
Output: The system rejects the file and provides an error message indicating the invalid file format.

Test Case Derivation: Ensures the tool detects unsupported file types and provides feedback, satisfying FR 1.

How test will be performed: Attempt to load a .txt or other non-Python file, and verify that the system rejects it with a message indicating an invalid file type.


\item{Missing Original Test Suite Handling\\}

Control: Automatic

Initial State: Tool is idle.

Input: User runs the tool without providing an original test suite.

Output: The system returns an error message indicating that the test suite is required.

Test Case Derivation: Confirms that the tool appropriately requires an original test suite to ensure functionality preservation after refactoring.

How test will be performed: Execute the tool command without specifying a test suite and observe the error message for missing input.


\end{enumerate}


\subsubsection{Code Smell Detection Tests}

This area includes tests to verify the detection of specified code 
smells that impact energy efficiency (FR 2).
		
\paragraph{2. Code Smell Detection}
\begin{enumerate}
\item{Detection of Large Class (LC)\\}

Control: Automatic
          
Initial State: Tool has loaded a `.py` file containing a class with a large number of methods and attributes.
          
Input: Python file with a class that exceeds the threshold for "Large Class" smell.
          
Output: Tool identifies the "Large Class" smell and suggests refactoring options like breaking the class into smaller classes.

Test Case Derivation: Ensures "Large Class" code smells are identified and appropriately refactored.
          
How test will be performed: Load a file with a large class and verify detection.

\item{Detection of Long Parameter List (LPL)\\}

Control: Automatic
					
Initial State: Tool has loaded a `.py` file containing a method with a long parameter list.
					
Input: Python file with a method using more parameters than the threshold.
					
Output: Tool flags the "Long Parameter List" smell and suggests bundling parameters into objects or reducing parameters.

Test Case Derivation: Ensures "Long Parameter List" code smell detection.
					
How test will be performed: Load a file with a method having a long parameter list and confirm detection.

\item{Detection of Long Method (LM)\\}

Control: Automatic
					
Initial State: Tool has loaded a `.py` file with a method that exceeds the line limit threshold.
					
Input: Python file containing a long method.
					
Output: Tool detects "Long Method" and suggests breaking it into smaller methods.

Test Case Derivation: Ensures "Long Method" detection and suggestions for improving readability.
					
How test will be performed: Load a file with a long method and check for detection.

\item{Detection of Long Message Chain (LMC)\\}

Control: Automatic
					
Initial State: Tool has loaded a `.py` file with a chain of method calls.
					
Input: Python file containing a message chain exceeding the threshold.
					
Output: Tool flags the "Long Message Chain" smell and suggests ways to simplify it, such as introducing intermediary methods.

Test Case Derivation: Validates "Long Message Chain" detection and suggestions for code simplification.
					
How test will be performed: Load a file with a long chain of method calls and confirm detection.

\item{Detection of Long Scope Chaining (LSC)\\}

Control: Automatic
					
Initial State: Tool has loaded a `.py` file containing deeply nested scopes.
					
Input: Python file with excessive scope chaining.
					
Output: Tool detects "Long Scope Chaining" and suggests reducing nesting or refactoring.

Test Case Derivation: Ensures tool detects deep nesting and provides ways to make code more readable.
					
How test will be performed: Load a file with nested scopes and confirm detection.

\item{Detection of Long Base Class List (LBCL)\\}

Control: Automatic
					
Initial State: Tool has loaded a `.py` file with a class that inherits from many base classes.
					
Input: Python file containing a class with an extensive inheritance list.
					
Output: Tool flags "Long Base Class List" and suggests refactoring, such as restructuring inheritance.

Test Case Derivation: Validates that long inheritance lists are detected and refactoring options are provided.
					
How test will be performed: Load a file with a long base class list and confirm detection.

\item{Detection of Useless Exception Handling (UEH)\\}

Control: Automatic
					
Initial State: Tool has loaded a `.py` file with empty or redundant `try-except` blocks.
					
Input: Python file containing useless exception handling blocks.
					
Output: Tool flags "Useless Exception Handling" and suggests meaningful handling or removal.

Test Case Derivation: Confirms detection of redundant exception handling and refactoring options.
					
How test will be performed: Load a file with empty `try-except` blocks and verify detection.

\item{Detection of Long Lambda Function (LLF)\\}

Control: Automatic
					
Initial State: Tool has loaded a `.py` file containing lambda functions that exceed the line or complexity threshold.
					
Input: Python file with a long lambda function.
					
Output: Tool detects "Long Lambda Function" and suggests converting it to a named function.

Test Case Derivation: Validates detection of long lambda functions and refactoring suggestions for clarity.
					
How test will be performed: Load a file with a long lambda and verify detection.

\item{Detection of Complex List Comprehension (CLC)\\}

Control: Automatic
					
Initial State: Tool has loaded a `.py` file with list comprehensions containing nested conditions.
					
Input: Python file with a complex list comprehension.
					
Output: Tool flags "Complex List Comprehension" and suggests simplifying the expression.

Test Case Derivation: Ensures tool detects complex list comprehensions and suggests simplifications.
					
How test will be performed: Load a file with complex list comprehension and confirm detection.

\item{Detection of Long Element Chain (LEC)\\}

Control: Automatic
					
Initial State: Tool has loaded a `.py` file with a long sequence of chained elements (e.g., dictionary access).
					
Input: Python file containing an element chain exceeding the length threshold.
					
Output: Tool detects "Long Element Chain" and suggests restructuring the code for readability.

Test Case Derivation: Confirms tool detects long element chains and suggests simplification.
					
How test will be performed: Load a file with a long element chain and verify detection.

\item{Detection of Long Ternary Conditional Expression (LTCE)\\}

Control: Automatic
					
Initial State: Tool has loaded a `.py` file with a ternary conditional expression that exceeds the line or complexity threshold.
					
Input: Python file containing a long ternary conditional.
					
Output: Tool flags "Long Ternary Conditional Expression" and suggests converting to a standard `if-else` block.

Test Case Derivation: Ensures long ternary expressions are detected and refactoring options are provided.
					
How test will be performed: Load a file with a long ternary expression and confirm detection.


\item{No Code Smells Detected Handling\\}

Control: Automatic

Initial State: Tool is idle.

Input: A valid Python file (filename.py) that adheres to best practices and contains no detectable code smells.

Output: The system returns a message indicating that no code smells were found in the code.

Test Case Derivation: This test ensures that the tool can correctly identify when there are no code smells present, as per functional requirement FR 2.

How test will be performed: Provide a Python file that is well-structured and free of common code smells, and verify that the tool outputs a message confirming the absence of smells.

\end{enumerate}

\subsubsection{Refactoring Suggestion Tests}

The following tests aim to validate the tool's capability to suggest appropriate refactorings in response to identified code smells, as outlined in the functional requirements (FR 5). These tests ensure that for each detected code smell, the tool provides actionable code modifications that not only enhance maintainability but also lead to measurable reductions in energy consumption.
		
\paragraph{3. Refactoring Suggestion Tests}
\begin{enumerate}
  \item{Large Class (LC) Refactoring Suggestion\\}

  Control: Automatic
            
  Initial State: Tool has identified a large class in the provided Python file.
            
  Input: A Python file containing a class with a high number of lines of code and methods.
            
  Output: The tool suggests splitting the large class into smaller, more manageable classes and displays the suggested modifications.
  
  Test Case Derivation: Ensures that the tool provides a refactoring suggestion that reduces energy consumption while maintaining functionality as per FR 5.
            
  How test will be performed: Feed a Python file with a large class to the tool and verify that it displays a refactoring suggestion to break the class into smaller parts.
  
  \item{Long Parameter List (LPL) Refactoring Suggestion\\}
  
  Control: Automatic
            
  Initial State: Tool has detected a method with too many parameters.
            
  Input: A Python file with a method signature that contains an excessive number of parameters.
            
  Output: The tool suggests using a data structure (e.g., a dictionary or an object) to encapsulate the parameters and shows the modified method signature.
  
  Test Case Derivation: Confirms that the tool can identify long parameter lists and suggest refactoring to improve code clarity and energy efficiency.
            
  How test will be performed: Submit a Python file with a method featuring a long parameter list and check that the tool provides a refactoring suggestion.
  
  \item{Long Method (LM) Refactoring Suggestion\\}
  
  Control: Automatic
            
  Initial State: Tool has identified a method that exceeds a predefined line count.
            
  Input: A Python file containing a method that is excessively long.
            
  Output: The tool suggests breaking the long method into smaller methods and displays the proposed modifications.
  
  Test Case Derivation: Validates that the tool recognizes long methods and suggests refactoring to enhance maintainability and reduce energy usage.
            
  How test will be performed: Provide the tool with a Python file containing a long method and observe if it suggests breaking it into smaller methods with clear modifications.
  
  \item{Long Message Chain (LMC) Refactoring Suggestion\\}
  
  Control: Automatic
            
  Initial State: Tool has identified a long message chain in the code.
            
  Input: A Python file with chained method calls resulting in a long message chain.
            
  Output: The tool suggests simplifying the message chain by assigning intermediate results to variables and shows the refactored code.
  
  Test Case Derivation: Ensures that the tool can detect long message chains and provide effective refactoring suggestions that improve energy efficiency.
            
  How test will be performed: Use a Python file containing a long message chain and confirm that the tool displays a suggestion to simplify it.
  
  \item{Long Scope Chaining (LSC) Refactoring Suggestion\\}
  
  Control: Automatic
            
  Initial State: Tool has detected a long scope chain in the provided code.
            
  Input: A Python file with multiple nested function calls or scope references.
            
  Output: The tool suggests flattening the scope chain by refactoring into clearer, standalone function calls and displays the proposed modifications.
  
  Test Case Derivation: Confirms that the tool can identify long scope chaining and suggest refactoring to enhance clarity and maintainability.
            
  How test will be performed: Feed a Python file with a long scope chain to the tool and check if it suggests appropriate refactoring.
  
  \item{Long Base Class List (LBCL) Refactoring Suggestion\\}
  
  Control: Automatic
            
  Initial State: Tool has identified a class inheriting from many base classes.
            
  Input: A Python file with a class declaration that inherits from multiple base classes.
            
  Output: The tool suggests refactoring the class to reduce the number of base classes, possibly by using composition instead of inheritance, and displays the suggested changes.
  
  Test Case Derivation: Validates that the tool recognizes long base class lists and provides suggestions for improving class design and energy efficiency.
            
  How test will be performed: Provide the tool with a Python file containing a class with a long base class list and observe if it suggests refactoring.
  
  \item{Useless Exception Handling (UEH) Refactoring Suggestion\\}
  
  Control: Automatic
            
  Initial State: Tool has detected unnecessary exception handling in the code.
            
  Input: A Python file containing try-except blocks that do not provide meaningful handling.
            
  Output: The tool suggests removing or modifying the exception handling and displays the modified code.
  
  Test Case Derivation: Validates that the tool can identify useless exception handling and suggests actionable refactorings to enhance clarity and efficiency.
            
  How test will be performed: Feed the tool a Python file with unnecessary exception handling and check if it suggests modifications.
  
  \item{Long Lambda Function (LLF) Refactoring Suggestion\\}
  
  Control: Automatic
            
  Initial State: Tool has identified a lambda function that is too complex or lengthy.
            
  Input: A Python file containing a long or complex lambda function.
            
  Output: The tool suggests refactoring the lambda function into a named function and shows the proposed changes.
  
  Test Case Derivation: Confirms that the tool recognizes long lambda functions and provides suggestions for refactoring to enhance readability and performance.
            
  How test will be performed: Submit a Python file with a long lambda function to the tool and verify that it suggests converting it into a named function.
  
  \item{Complex List Comprehension (CLC) Refactoring Suggestion\\}
  
  Control: Automatic
            
  Initial State: Tool has detected a complex list comprehension.
            
  Input: A Python file containing a list comprehension that is hard to read or understand.
            
  Output: The tool suggests breaking the list comprehension into a for loop and displays the modified code.
  
  Test Case Derivation: Ensures that the tool identifies complex list comprehensions and suggests refactoring for clarity and efficiency.
            
  How test will be performed: Provide a Python file with a complex list comprehension and observe if the tool offers a simpler alternative.
  
  \item{Long Element Chain (LEC) Refactoring Suggestion\\}
  
  Control: Automatic
            
  Initial State: Tool has detected a long chain of method calls on an object.
            
  Input: A Python file with an object undergoing multiple chained calls.
            
  Output: The tool suggests breaking the chain into separate calls and displays the refactored code.
  
  Test Case Derivation: Validates that the tool can recognize long element chains and provide suggestions to improve code structure and efficiency.
            
  How test will be performed: Use a Python file featuring a long element chain and verify that the tool suggests breaking it apart.
  
  \item{Long Ternary Conditional Expression (LTCE) Refactoring Suggestion\\}
  
  Control: Automatic
            
  Initial State: Tool has detected a complex ternary conditional expression.
            
  Input: A Python file with a long ternary expression that is difficult to read.
            
  Output: The tool suggests refactoring the ternary expression into a standard if-else statement and shows the suggested changes.
  
  Test Case Derivation: Ensures that the tool identifies long ternary conditional expressions and provides clearer refactoring alternatives.
            
  How test will be performed: Submit a Python file containing a long ternary expression to the tool and confirm that it suggests refactoring it into an if-else statement.
  
  \item{Energy Consumption Measurement for Suggested Refactoring\\}
  
  Control: Automatic
            
  Initial State: Tool has suggested a refactoring for detected code smells.
            
  Input: Suggested refactored code.
            
  Output: Measurement showing improved energy consumption in joules.
  
  Test Case Derivation: Confirms suggestions provide measurable energy efficiency improvement, per FR 5.
            
  How test will be performed: Apply a refactoring and measure energy consumption before and after.
  
  \item{Optimal Refactoring Selection\\}
  
  Control: Automatic
            
  Initial State: Tool has identified multiple refactoring options for a single code smell.
            
  Input: Multiple refactoring options.
            
  Output: System chooses the refactoring with the lowest measured energy consumption.
  
  Test Case Derivation: Ensures that the tool optimizes for energy efficiency, meeting FR 7.
            
  How test will be performed: Provide multiple refactoring options and verify the tool selects the most energy-efficient one.
  
  \item{Refactoring Suggestion Not Possible Handling\\}

  Control: Automatic
  
  Initial State: Tool is idle.
  
  Input: Code containing a complex smell that cannot be refactored due to constraints.
  
  Output: The system provides a message indicating that no refactoring suggestions can be made for the identified smell or given code.
  
  Test Case Derivation: Ensures the tool gracefully handles situations where refactoring is too complex or not feasible.
  
  How test will be performed: Provide a code example that includes a complex smell and observe the output for an appropriate message regarding the lack of suggestions.

\end{enumerate}

\subsubsection{Output Validation Tests}

The following tests are designed to validate that the functionality of the original Python code remains intact after refactoring. Each test ensures that the refactored code passes the same test suite as the original code, confirming compliance with functional requirement FR 3.
		
\paragraph{4. Output Validation Tests}
\begin{enumerate}
  \item{Validate Refactored Code Functionality On Provided Test Suite\\}

  Control: Automatic
  
  Initial State: The original Python code is equipped with an existing test suite it passes.
  
  Input: The original Python code and its associated test suite.
  
  Output: The refactored code passes 100\% of the original test suite.
  
  Test Case Derivation: This test confirms that the refactored code preserves the original functionality by passing all tests from the original suite, as stipulated in FR 3.
  
  How test will be performed: The tool will refactor the code, and then the original test suite will be executed against the refactored code to check for passing results.
  
  \item{Verification of Valid Python Output\\}

  Control: Automatic
  
  Initial State: Tool has processed a file with detected code smells.
  
  Input: Output refactored Python code.
  
  Output: Refactored code is syntactically correct and Python-compliant.
  
  Test Case Derivation: Ensures refactored code remains valid and usable, satisfying FR 6.
  
  How test will be performed: Run a linter on the output code and verify it passes without syntax errors.
  
\end{enumerate}

\subsubsection{Tests for Reporting Functionality}

The reporting functionality of the tool is crucial for providing users with comprehensive insights into the refactoring process, including detected code smells, refactorings applied, energy consumption measurements, and the results of the original test suite. This section outlines tests that ensure the reporting feature operates correctly and delivers accurate, well-structured information as specified in the functional requirements (FR 9). 
		
\paragraph{5. Tests for Report Generation\\}
\begin{enumerate}
  \item{A Report With All Components Is Generated\\}

Control: Automatic

Initial State: The tool has completed refactoring a Python code file.

Input: The refactoring results, including detected code smells, applied refactorings, and energy consumption metrics.

Output: A well-structured report is generated, summarizing the refactoring process.

Test Case Derivation: This test ensures that the tool generates a comprehensive report that includes all necessary information as required by FR 9.

How test will be performed: After refactoring, the tool will invoke the report generation feature and validate that the output meets the structure and content specifications.


\item{Validation of Code Smell and Refactoring Data in Report\\}

Control: Automatic

Initial State: The tool has identified code smells and performed refactorings.

Input: The results of the refactoring process.

Output: The generated report accurately lists all detected code smells and the corresponding refactorings applied.

Test Case Derivation: This test verifies that the report includes correct and complete information about code smells and refactorings, in compliance with FR 9.

How test will be performed: The tool will compare the contents of the generated report against the detected code smells and refactorings to ensure accuracy.


\item{Energy Consumption Metrics Included in Report\\}

Control: Automatic

Initial State: The tool has measured energy consumption before and after refactoring.

Input: Energy consumption metrics obtained during the refactoring process.

Output: The report presents a clear comparison of energy usage before and after the refactorings.

Test Case Derivation: This test confirms that the reporting feature effectively communicates energy consumption improvements, aligning with FR 9.

How test will be performed: The tool will analyze the energy metrics in the report to ensure they accurately reflect the measurements taken during the refactoring.


\item{Functionality Test Results Included in Report\\}

Control: Automatic

Initial State: The original test suite has been executed against the refactored code.

Input: The outcomes of the test suite execution.

Output: The report summarizes the test results, indicating which tests passed and failed.

Test Case Derivation: This test ensures that the reporting functionality accurately reflects the results of the test suite as specified in FR 9.

How test will be performed: The tool will generate the report and validate that it contains a summary of test results consistent with the actual test outcomes.

\end{enumerate}


\subsubsection{Documentation Availability Tests}

The following test is designed to ensure the availability of documentation as per FR 10.
		
\paragraph{6. Documentation Availability}
\begin{enumerate}
  \item{Test for Documentation Availability\\}

  Control: Manual
  
  Initial State: The may or may not be installed.
  
  Input: User attempts to access the documentation.
  
  Output: The documentation is available and covers installation, usage, and troubleshooting.
  
  Test Case Derivation: Validates that the documentation meets user needs (FR 10).
  
  How test will be performed: Review the documentation for completeness and clarity.
\end{enumerate}

\subsubsection{IDE Extension Tests}

The following tests are designed to ensure that the user can integrate the tool into VSCode IDE as specified in FR 11 and that the tool works as intended as an extension.
		
\paragraph{6. IDE Integration}
\begin{enumerate}
  \item{Installation of Extension in Visual Studio Code\\}

  Control: Manual
  
  Initial State: The user has Visual Studio Code installed on their machine.
  
  Input: The user attempts to install the refactoring tool extension from the Visual Studio Code Marketplace.
  
  Output: The extension installs successfully, and the user is able to see it listed in the Extensions view.
  
  Test Case Derivation: This test validates the installation process of the extension to ensure that users can easily add the tool to their development environment.
  
  How test will be performed: 
  \begin{enumerate}
      \item Open Visual Studio Code.
      \item Navigate to the Extensions view (Ctrl+Shift+X).
      \item Search for the refactoring tool extension in the marketplace.
      \item Click on the "Install" button.
      \item After installation, verify that the extension appears in the installed extensions list.
      \item Confirm that the extension is enabled and ready for use by checking its functionality within the editor.
  \end{enumerate}

  \item{Running the Extension in Visual Studio Code\\}

Control: Manual

Initial State: The user has successfully installed the refactoring tool extension in Visual Studio Code.

Input: The user opens a Python file and activates the refactoring tool extension.

Output: The extension runs successfully, and the user can see a list of detected code smells and suggested refactorings.

Test Case Derivation: This test validates that the extension can be executed within the development environment and that it correctly identifies code smells as per the functional requirements in the SRS.

How test will be performed:
\begin{enumerate}
    \item Open Visual Studio Code.
    \item Open a valid Python file that contains known code smells.
    \item Activate the refactoring tool extension using the command palette (Ctrl+Shift+P) and selecting the extension command.
    \item Observe the output panel for the detection of code smells.
    \item Verify that the extension lists the identified code smells and provides appropriate refactoring suggestions.
    \item Confirm that the suggestions are relevant and feasible for the detected code smells.
\end{enumerate}

\end{enumerate}

\subsection{Tests for Nonfunctional Requirements}

\wss{The nonfunctional requirements for accuracy will likely just reference the
  appropriate functional tests from above.  The test cases should mention
  reporting the relative error for these tests.  Not all projects will
  necessarily have nonfunctional requirements related to accuracy.}

\wss{For some nonfunctional tests, you won't be setting a target threshold for
passing the test, but rather describing the experiment you will do to measure
the quality for different inputs.  For instance, you could measure speed versus
the problem size.  The output of the test isn't pass/fail, but rather a summary
table or graph.}

\wss{Tests related to usability could include conducting a usability test and
  survey.  The survey will be in the Appendix.}

\wss{Static tests, review, inspections, and walkthroughs, will not follow the
format for the tests given below.}

\wss{If you introduce static tests in your plan, you need to provide details.
How will they be done?  In cases like code (or document) walkthroughs, who will
be involved? Be specific.}

\subsubsection{Area of Testing1}
		
\paragraph{Title for Test}

\begin{enumerate}

\item{test-id1\\}

Type: Functional, Dynamic, Manual, Static etc.
					
Initial State: 
					
Input/Condition: 
					
Output/Result: 
					
How test will be performed: 
					
\item{test-id2\\}

Type: Functional, Dynamic, Manual, Static etc.
					
Initial State: 
					
Input: 
					
Output: 
					
How test will be performed: 

\end{enumerate}

\subsubsection{Area of Testing2}

...

\subsection{Traceability Between Test Cases and Requirements}

\wss{Provide a table that shows which test cases are supporting which
  requirements.}

\section{Unit Test Description}

\wss{This section should not be filled in until after the MIS (detailed design
  document) has been completed.}

\wss{Reference your MIS (detailed design document) and explain your overall
philosophy for test case selection.}  

\wss{To save space and time, it may be an option to provide less detail in this section.  
For the unit tests you can potentially layout your testing strategy here.  That is, you 
can explain how tests will be selected for each module.  For instance, your test building 
approach could be test cases for each access program, including one test for normal behaviour 
and as many tests as needed for edge cases.  Rather than create the details of the input 
and output here, you could point to the unit testing code.  For this to work, you code 
needs to be well-documented, with meaningful names for all of the tests.}

\subsection{Unit Testing Scope}

\wss{What modules are outside of the scope.  If there are modules that are
  developed by someone else, then you would say here if you aren't planning on
  verifying them.  There may also be modules that are part of your software, but
  have a lower priority for verification than others.  If this is the case,
  explain your rationale for the ranking of module importance.}

\subsection{Tests for Functional Requirements}

\wss{Most of the verification will be through automated unit testing.  If
  appropriate specific modules can be verified by a non-testing based
  technique.  That can also be documented in this section.}

\subsubsection{Module 1}

\wss{Include a blurb here to explain why the subsections below cover the module.
  References to the MIS would be good.  You will want tests from a black box
  perspective and from a white box perspective.  Explain to the reader how the
  tests were selected.}

\begin{enumerate}

\item{test-id1\\}

Type: \wss{Functional, Dynamic, Manual, Automatic, Static etc. Most will
  be automatic}
					
Initial State: 
					
Input: 
					
Output: \wss{The expected result for the given inputs}

Test Case Derivation: \wss{Justify the expected value given in the Output field}

How test will be performed: 
					
\item{test-id2\\}

Type: \wss{Functional, Dynamic, Manual, Automatic, Static etc. Most will
  be automatic}
					
Initial State: 
					
Input: 
					
Output: \wss{The expected result for the given inputs}

Test Case Derivation: \wss{Justify the expected value given in the Output field}

How test will be performed: 

\item{...\\}
    
\end{enumerate}

\subsubsection{Module 2}

...

\subsection{Tests for Nonfunctional Requirements}

\wss{If there is a module that needs to be independently assessed for
  performance, those test cases can go here.  In some projects, planning for
  nonfunctional tests of units will not be that relevant.}

\wss{These tests may involve collecting performance data from previously
  mentioned functional tests.}

\subsubsection{Module ?}
		
\begin{enumerate}

\item{test-id1\\}

Type: \wss{Functional, Dynamic, Manual, Automatic, Static etc. Most will
  be automatic}
					
Initial State: 
					
Input/Condition: 
					
Output/Result: 
					
How test will be performed: 
					
\item{test-id2\\}

Type: Functional, Dynamic, Manual, Static etc.
					
Initial State: 
					
Input: 
					
Output: 
					
How test will be performed: 

\end{enumerate}

\subsubsection{Module ?}

...

\subsection{Traceability Between Test Cases and Modules}

\wss{Provide evidence that all of the modules have been considered.}
			

\bibliography{../../refs/References}

\newpage

\begin{appendices}

\section{Appendix}

This is where you can place additional information.

\subsection{Symbolic Parameters}

The definition of the test cases will call for SYMBOLIC\_CONSTANTS.
Their values are defined in this section for easy maintenance.

\subsection{Usability Survey Questions?}

\wss{This is a section that would be appropriate for some projects.}

\newpage{}
\section{Reflection}

\wss{This section is not required for CAS 741}

The information in this section will be used to evaluate the team members on the
graduate attribute of Lifelong Learning.

\input{../Reflection.tex}

\begin{enumerate}
  \item What went well while writing this deliverable? 
  \item What pain points did you experience during this deliverable, and how
    did you resolve them?
  \item What knowledge and skills will the team collectively need to acquire to
  successfully complete the verification and validation of your project?
  Examples of possible knowledge and skills include dynamic testing knowledge,
  static testing knowledge, specific tool usage, Valgrind etc.  You should look to
  identify at least one item for each team member.
  \item For each of the knowledge areas and skills identified in the previous
  question, what are at least two approaches to acquiring the knowledge or
  mastering the skill?  Of the identified approaches, which will each team
  member pursue, and why did they make this choice?
\end{enumerate}

\end{appendices}

\end{document}