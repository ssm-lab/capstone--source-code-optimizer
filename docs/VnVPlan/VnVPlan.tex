\documentclass[12pt, titlepage]{article}

\usepackage{booktabs}
\usepackage{tabularx}
\usepackage{hyperref}
\hypersetup{
    colorlinks,
    citecolor=blue,
    filecolor=black,
    linkcolor=red,
    urlcolor=blue
}

%Includes "References" in the table of contents
\usepackage[nottoc]{tocbibind}

\usepackage[toc,page]{appendix}

\usepackage[square,numbers,compress]{natbib}
\bibliographystyle{abbrvnat}
\usepackage{amssymb}

%% Comments

\usepackage{color}

% \newif\ifcomments\commentstrue %displays comments
\newif\ifcomments\commentsfalse %so that comments do not display

\ifcomments
\newcommand{\authornote}[3]{\textcolor{#1}{[#3 ---#2]}}
\newcommand{\todo}[1]{\textcolor{red}{[TODO: #1]}}
\else
\newcommand{\authornote}[3]{}
\newcommand{\todo}[1]{}
\fi

\newcommand{\wss}[1]{\authornote{blue}{SS}{#1}} 
\newcommand{\plt}[1]{\authornote{magenta}{TPLT}{#1}} %For explanation of the template
\newcommand{\an}[1]{\authornote{cyan}{Author}{#1}}

%% Common Parts

\newcommand{\progname}{Software Engineering} % PUT YOUR PROGRAM NAME HERE
\newcommand{\authname}{\textbf{Team 4, EcoOptimizers} \\
  \\ Nivetha Kuruparan
  \\ Sevhena Walker
  \\ Tanveer Brar
  \\ Mya Hussain
\\ Ayushi Amin} % AUTHOR NAMES

\usepackage{hyperref}
\hypersetup{colorlinks=true, linkcolor=blue, citecolor=blue, filecolor=blue,
urlcolor=blue, unicode=false}
\urlstyle{same}



\newcommand{\SRS}{\href{https://github.com/ssm-lab/capstone--source-code-optimizer/blob/main/docs/SRS/SRS.pdf}{SRS}}
\newcommand{\MG}{\href{https://github.com/ssm-lab/capstone--source-code-optimizer/blob/main/docs/Design/SoftArchitecture/MG.pdf}{MG}}
\newcommand{\MIS}{\href{https://github.com/ssm-lab/capstone--source-code-optimizer/blob/main/docs/Design/SoftDetailedDes/MIS.pdf}{MIS}}

\begin{document}

\title{System Verification and Validation Plan for \progname{}} 
\author{\authname}
\date{\today}
	
\maketitle

\pagenumbering{roman}

\section*{Revision History}

\begin{tabularx}{\textwidth}{p{3cm}p{2cm}X}
\toprule {\bf Date} & {\bf Version} & {\bf Notes}\\
\midrule
Date 1 & 1.0 & Notes\\
Date 2 & 1.1 & Notes\\
\bottomrule
\end{tabularx}

~\\
\wss{The intention of the VnV plan is to increase confidence in the software.
However, this does not mean listing every verification and validation technique
that has ever been devised.  The VnV plan should also be a \textbf{feasible}
plan. Execution of the plan should be possible with the time and team available.
If the full plan cannot be completed during the time available, it can either be
modified to ``fake it'', or a better solution is to add a section describing
what work has been completed and what work is still planned for the future.}

\wss{The VnV plan is typically started after the requirements stage, but before
the design stage.  This means that the sections related to unit testing cannot
initially be completed.  The sections will be filled in after the design stage
is complete.  the final version of the VnV plan should have all sections filled
in.}

\newpage

\tableofcontents

\listoftables
\wss{Remove this section if it isn't needed}

\listoffigures
\wss{Remove this section if it isn't needed}

\newpage

\section{Symbols, Abbreviations, and Acronyms}

\renewcommand{\arraystretch}{1.2}
\begin{tabular}{l l} 
  \toprule		
  \textbf{symbol} & \textbf{description}\\
  \midrule 
  T & Test\\
  \bottomrule
\end{tabular}\\

\wss{symbols, abbreviations, or acronyms --- you can simply reference the SRS
  \cite{SRS} tables, if appropriate}

\wss{Remove this section if it isn't needed}

\newpage

\pagenumbering{arabic}

This document ... \wss{provide an introductory blurb and roadmap of the
  Verification and Validation plan}

\section{General Information}

\subsection{Summary}

The software being tested is called EcoOptimizer. EcoOptimizer is a python refactoring library that focuses on optimizing code in a way that reduces its energy consumption. The system will be capable to analyze python code in order to spot inefficiencies (code smells) within, measuring the energy efficiency of the inputted code and, of course, apply appropriate refactorings that preserve the initial function of the source code. \\

Furthermore, peripheral tools such as a Visual Studio Code (VS Code) extension and GitHub Action are also to be tested. The extension will integrate the library with Visual Studio Code for a more efficient development process and the GitHub Action will allow a proper integration of the library into continuous integration (CI) workflows.

\subsection{Objectives}

The primary objective of this project is to build confidence in the \textbf{correctness} and \textbf{energy efficiency} of the refactoring library, ensuring that it performs as expected in improving code efficiency while maintaining functionality. Usability is also emphasized, particularly in the user interfaces provided through the \textbf{VS Code extension} and \textbf{GitHub Action} integrations, as ease of use is critical for adoption by software developers. These qualities—correctness, energy efficiency, and usability—are central to the project’s success, as they directly impact user experience, performance, and the sustainable benefits of the tool.\\

Certain objectives are intentionally left out-of-scope due to resource constraints. We will not independently verify external libraries or dependencies; instead, we assume they have been validated by their respective development teams. 

\subsection{Challenge Level and Extras}

Our project, set at a \textbf{general} challenge level, includes two additional focuses: \textbf{user documentation} and \textbf{usability testing}. The user documentation aims to provide clear, accessible guidance for developers, making it easy to understand the tool’s setup, functionality, and integration into existing workflows. Usability testing will ensure that the tool is intuitive and meets user needs effectively, offering insights to refine the user interface and optimize interactions with it's features.

\subsection{Relevant Documentation}

The Verification and Validation (VnV) plan relies on three key documents to guide testing and assessment: 
\begin{itemize}
  \item[] \textbf{Software Requirements Specification (\SRS)\cite{SRS}:} The foundation for the VnV plan, as it defines the functional and non-functional requirements the software must meet; aligning tests with these requirements ensures that the software performs as expected in terms of correctness, performance, and usability.
  
  \item[] \textbf{Module Interface Specification (\MG)\cite{MGDoc}:} Provides detailed information about each module's interfaces, which is crucial for integration testing to verify that all modules interact correctly within the system.
  
  \item[] \textbf{Module Guide (\MIS)\cite{MISDoc}:} Outlines the system's architectural design and module structure, ensuring the design of tests that align with the intended flow and dependencies within the system.
\end{itemize}

\section{Plan}

\wss{Introduce this section.  You can provide a roadmap of the sections to
  come.}

\subsection{Verification and Validation Team}

\wss{Your teammates.  Maybe your supervisor.
  You should do more than list names.  You should say what each person's role is
  for the project's verification.  A table is a good way to summarize this information.}

\subsection{SRS Verification Plan}

\textbf{Function \& Non-Functional Requirements:}
\begin{itemize}
    \item A comprehensive test suite that covers all requirements specified in the SRS will be created.
    \item Each requirement will be mapped to specific test cases to ensure maximum coverage.
    \item Automated and manual testing will be conducted to verify that the implemented system meets each functional requirement.
    \item Usability testing with representative users will be carried out to validate user experience requirements and other non-functional requirements.
    \item Performance tests will be conducted to verify that the system meets specified performance requirements.
\end{itemize}

\textbf{Traceability Matrix:}
\begin{itemize}
    \item We will create a requirements traceability matrix that links each SRS requirement to its corresponding implementation, test cases, and test results.
    \item This matrix will help identify any requirements that may have been overlooked during development.
\end{itemize}

\textbf{Supervisor Review:}
\begin{itemize}
    \item After the implementation of the system, we will conduct a formal review session with key stakeholders such as our project supervisor, Dr. Istvan David.
    \item The stakeholders will be asked to verify that each requirement in the SRS is mapped out to specific expectations of the project. 
    \item Prior to meeting, we will provide a summary of key requirements and design decisions and prepare a list specific questions or areas where we seek guidance.
    \item During the meeting, we will present an overview of the SRS using tables and other visual aids. We will conduct a walk through of critical section. Finally, we will discuss any potential risks or challenges identified.
\end{itemize}

\textbf{User Acceptance Testing (UAT):}
\begin{itemize}
    \item We will involve potential end-users in testing the system to ensure it meets real-world usage scenarios.
    \item Feedback from UAT will be used to identify any discrepancies between the SRS and user expectations.
\end{itemize}

\textbf{Continuous Verification:}
\begin{itemize}
    \item Throughout the development process, we will regularly review and update the SRS to ensure it remains aligned with the evolving system.
    \item Any changes to requirements will be documented and their impact on the system assessed.
\end{itemize}

\textbf{\textit{\\Checklist for SRS Verification Plan}}
\begin{itemize}
    \item[$\square$] Create comprehensive test suite covering all SRS requirements
    \item[$\square$] Map each requirement to specific test cases
    \item[$\square$] Conduct automated testing for functional requirements
    \item[$\square$] Perform manual testing for functional requirements
    \item[$\square$] Carry out usability testing with representative users
    \item[$\square$] Conduct performance tests to verify system meets requirements
    \item[$\square$] Create requirements traceability matrix
    \item[$\square$] Link each SRS requirement to implementation in traceability matrix
    \item[$\square$] Link each SRS requirement to test cases in traceability matrix
    \item[$\square$] Link each SRS requirement to test results in traceability matrix
    \item[$\square$] Schedule formal review session with project supervisor
    \item[$\square$] Prepare summary of key requirements and design decisions for supervisor review
    \item[$\square$] Prepare list of specific questions for supervisor review
    \item[$\square$] Create visual aids for SRS overview presentation
    \item[$\square$] Conduct walkthrough of critical SRS sections during review
    \item[$\square$] Discuss potential risks and challenges with supervisor
    \item[$\square$] Organize User Acceptance Testing (UAT) with potential end-users
    \item[$\square$] Collect and analyze UAT feedback
    \item[$\square$] Identify discrepancies between SRS and user expectations from UAT
    \item[$\square$] Establish process for regular SRS review and updates
    \item[$\square$] Document any changes to requirements
    \item[$\square$] Assess impact of requirement changes on the system
\end{itemize}

\subsection{Design Verification Plan}

\textbf{Peer Review Plan:}
\begin{itemize}
    \item Each team member along with other classmates will thoroughly review the entire Design Document.
    \item A checklist-based approach will be used to ensure all key elements are covered.
    \item Feedback will be collected and discussed in a dedicated team meeting.
\end{itemize}

\textbf{Supervisor Review:}
\begin{itemize}
    \item A structured review meeting will be scheduled with our project supervisor, Dr. Istvan David.
    \item We will present an overview of the design using visual aids (e.g., diagrams, tables).
    \item We will conduct a walkthrough of critical sections.
    \item We will use our project's issue tracker to document and follow up on any action items or changes resulting from this review.
\end{itemize}

\begin{itemize}
  \item[$\square$] All functional requirements are mapped to specific design elements 
  \item[$\square$] Each functional requirement is fully addressed by the design 
  \item[$\square$] No functional requirements are overlooked or partially implemented 
  \item[$\square$] Performance requirements are met by the design 
  \item[$\square$] Scalability considerations are incorporated 
  \item[$\square$] Reliability and availability requirements are satisfied 
  \item[$\square$] Usability requirements are reflected in the user interface design
  \item[$\square$] High-level architecture is clearly defined 
  \item[$\square$] Architectural decisions are justified with rationale 
  \item[$\square$] Architecture aligns with project constraints and goals 
  \item[$\square$] All major components are identified and described 
  \item[$\square$] Interactions between components are clearly specified 
  \item[$\square$] Component responsibilities are well-defined 
  \item[$\square$] Appropriate data structures are chosen for each task 
  \item[$\square$] Efficient algorithms are selected for critical operations 
  \item[$\square$] Rationale for data structure and algorithm choices is provided
  \item[$\square$] UI design is consistent with usability requirements 
  \item[$\square$] User flow is logical and efficient 
  \item[$\square$] Accessibility considerations are incorporated 
  \item[$\square$] All external interfaces are properly specified 
  \item[$\square$] Interface protocols and data formats are defined 
  \item[$\square$] Error handling for external interfaces is addressed 
  \item[$\square$] Comprehensive error handling strategy is in place
  \item[$\square$] Exception scenarios are identified and managed 
  \item[$\square$] Error messages are clear and actionable 
  \item[$\square$] Authentication and authorization mechanisms are described 
  \item[$\square$] Data encryption methods are specified where necessary 
  \item[$\square$] Security best practices are followed in the design
  \item[$\square$] Design allows for future expansion and feature additions 
  \item[$\square$] Code modularity and reusability are considered 
  \item[$\square$] Documentation standards are established for maintainability 
  \item[$\square$] Performance bottlenecks are identified and addressed 
  \item[$\square$] Resource utilization is optimized 
  \item[$\square$] Performance testing strategies are outlined 
  \item[$\square$] Design adheres to established coding standards 
  \item[$\square$] Industry best practices are followed 
  \item[$\square$] Design patterns are appropriately applied
  \item[$\square$] All major design decisions are justified 
  \item[$\square$] Trade-offs are explained with pros and cons 
  \item[$\square$] Alternative approaches considered are documented 
  \item[$\square$] Documents is clear, concise, and free of ambiguities 
  \item[$\square$] Documents follows a logical structure 
\end{itemize}

\subsection{Verification and Validation Plan Verification Plan}

\wss{The verification and validation plan is an artifact that should also be
verified.  Techniques for this include review and mutation testing.}

\wss{The review will include reviews by your classmates}

\wss{Create a checklists?}

\subsection{Implementation Verification Plan}

\wss{You should at least point to the tests listed in this document and the unit
  testing plan.}

\wss{In this section you would also give any details of any plans for static
  verification of the implementation.  Potential techniques include code
  walkthroughs, code inspection, static analyzers, etc.}

\wss{The final class presentation in CAS 741 could be used as a code
walkthrough.  There is also a possibility of using the final presentation (in
CAS741) for a partial usability survey.}

\subsection{Automated Testing and Verification Tools}

\wss{What tools are you using for automated testing.  Likely a unit testing
  framework and maybe a profiling tool, like ValGrind.  Other possible tools
  include a static analyzer, make, continuous integration tools, test coverage
  tools, etc.  Explain your plans for summarizing code coverage metrics.
  Linters are another important class of tools.  For the programming language
  you select, you should look at the available linters.  There may also be tools
  that verify that coding standards have been respected, like flake9 for
  Python.}

\wss{If you have already done this in the development plan, you can point to
that document.}

\wss{The details of this section will likely evolve as you get closer to the
  implementation.}

\subsection{Software Validation Plan}

\wss{If there is any external data that can be used for validation, you should
  point to it here.  If there are no plans for validation, you should state that
  here.}

\wss{You might want to use review sessions with the stakeholder to check that
the requirements document captures the right requirements.  Maybe task based
inspection?}

\wss{For those capstone teams with an external supervisor, the Rev 0 demo should 
be used as an opportunity to validate the requirements.  You should plan on 
demonstrating your project to your supervisor shortly after the scheduled Rev 0 demo.  
The feedback from your supervisor will be very useful for improving your project.}

\wss{For teams without an external supervisor, user testing can serve the same purpose 
as a Rev 0 demo for the supervisor.}

\wss{This section might reference back to the SRS verification section.}

\section{System Tests}

\wss{There should be text between all headings, even if it is just a roadmap of
the contents of the subsections.}

\subsection{Tests for Functional Requirements}

\wss{Subsets of the tests may be in related, so this section is divided into
  different areas.  If there are no identifiable subsets for the tests, this
  level of document structure can be removed.}

\wss{Include a blurb here to explain why the subsections below
  cover the requirements.  References to the SRS would be good here.}

\subsubsection{Area of Testing1}

\wss{It would be nice to have a blurb here to explain why the subsections below
  cover the requirements.  References to the SRS would be good here.  If a section
  covers tests for input constraints, you should reference the data constraints
  table in the SRS.}
		
\paragraph{Title for Test}

\begin{enumerate}

\item{test-id1\\}

Control: Manual versus Automatic
					
Initial State: 
					
Input: 
					
Output: \wss{The expected result for the given inputs.  Output is not how you
are going to return the results of the test.  The output is the expected
result.}

Test Case Derivation: \wss{Justify the expected value given in the Output field}
					
How test will be performed: 
					
\item{test-id2\\}

Control: Manual versus Automatic
					
Initial State: 
					
Input: 
					
Output: \wss{The expected result for the given inputs}

Test Case Derivation: \wss{Justify the expected value given in the Output field}

How test will be performed: 

\end{enumerate}

\subsubsection{Area of Testing2}

...

\subsection{Tests for Nonfunctional Requirements}

\wss{The nonfunctional requirements for accuracy will likely just reference the
  appropriate functional tests from above.  The test cases should mention
  reporting the relative error for these tests.  Not all projects will
  necessarily have nonfunctional requirements related to accuracy.}

\wss{For some nonfunctional tests, you won't be setting a target threshold for
passing the test, but rather describing the experiment you will do to measure
the quality for different inputs.  For instance, you could measure speed versus
the problem size.  The output of the test isn't pass/fail, but rather a summary
table or graph.}

\wss{Tests related to usability could include conducting a usability test and
  survey.  The survey will be in the Appendix.}

\wss{Static tests, review, inspections, and walkthroughs, will not follow the
format for the tests given below.}

\wss{If you introduce static tests in your plan, you need to provide details.
How will they be done?  In cases like code (or document) walkthroughs, who will
be involved? Be specific.}

\subsubsection{Area of Testing1}
		
\paragraph{Title for Test}

\begin{enumerate}

\item{test-id1\\}

Type: Functional, Dynamic, Manual, Static etc.
					
Initial State: 
					
Input/Condition: 
					
Output/Result: 
					
How test will be performed: 
					
\item{test-id2\\}

Type: Functional, Dynamic, Manual, Static etc.
					
Initial State: 
					
Input: 
					
Output: 
					
How test will be performed: 

\end{enumerate}

\subsubsection{Area of Testing2}

...

\subsection{Traceability Between Test Cases and Requirements}

\wss{Provide a table that shows which test cases are supporting which
  requirements.}

\section{Unit Test Description}

\wss{This section should not be filled in until after the MIS (detailed design
  document) has been completed.}

\wss{Reference your MIS (detailed design document) and explain your overall
philosophy for test case selection.}  

\wss{To save space and time, it may be an option to provide less detail in this section.  
For the unit tests you can potentially layout your testing strategy here.  That is, you 
can explain how tests will be selected for each module.  For instance, your test building 
approach could be test cases for each access program, including one test for normal behaviour 
and as many tests as needed for edge cases.  Rather than create the details of the input 
and output here, you could point to the unit testing code.  For this to work, you code 
needs to be well-documented, with meaningful names for all of the tests.}

\subsection{Unit Testing Scope}

\wss{What modules are outside of the scope.  If there are modules that are
  developed by someone else, then you would say here if you aren't planning on
  verifying them.  There may also be modules that are part of your software, but
  have a lower priority for verification than others.  If this is the case,
  explain your rationale for the ranking of module importance.}

\subsection{Tests for Functional Requirements}

\wss{Most of the verification will be through automated unit testing.  If
  appropriate specific modules can be verified by a non-testing based
  technique.  That can also be documented in this section.}

\subsubsection{Module 1}

\wss{Include a blurb here to explain why the subsections below cover the module.
  References to the MIS would be good.  You will want tests from a black box
  perspective and from a white box perspective.  Explain to the reader how the
  tests were selected.}

\begin{enumerate}

\item{test-id1\\}

Type: \wss{Functional, Dynamic, Manual, Automatic, Static etc. Most will
  be automatic}
					
Initial State: 
					
Input: 
					
Output: \wss{The expected result for the given inputs}

Test Case Derivation: \wss{Justify the expected value given in the Output field}

How test will be performed: 
					
\item{test-id2\\}

Type: \wss{Functional, Dynamic, Manual, Automatic, Static etc. Most will
  be automatic}
					
Initial State: 
					
Input: 
					
Output: \wss{The expected result for the given inputs}

Test Case Derivation: \wss{Justify the expected value given in the Output field}

How test will be performed: 

\item{...\\}
    
\end{enumerate}

\subsubsection{Module 2}

...

\subsection{Tests for Nonfunctional Requirements}

\wss{If there is a module that needs to be independently assessed for
  performance, those test cases can go here.  In some projects, planning for
  nonfunctional tests of units will not be that relevant.}

\wss{These tests may involve collecting performance data from previously
  mentioned functional tests.}

\subsubsection{Module ?}
		
\begin{enumerate}

\item{test-id1\\}

Type: \wss{Functional, Dynamic, Manual, Automatic, Static etc. Most will
  be automatic}
					
Initial State: 
					
Input/Condition: 
					
Output/Result: 
					
How test will be performed: 
					
\item{test-id2\\}

Type: Functional, Dynamic, Manual, Static etc.
					
Initial State: 
					
Input: 
					
Output: 
					
How test will be performed: 

\end{enumerate}

\subsubsection{Module ?}

...

\subsection{Traceability Between Test Cases and Modules}

\wss{Provide evidence that all of the modules have been considered.}
			

\bibliography{../../refs/References}

\newpage

\begin{appendices}

\section{Appendix}

This is where you can place additional information.

\subsection{Symbolic Parameters}

The definition of the test cases will call for SYMBOLIC\_CONSTANTS.
Their values are defined in this section for easy maintenance.

\subsection{Usability Survey Questions?}

\wss{This is a section that would be appropriate for some projects.}

\newpage{}
\section{Reflection}

\wss{This section is not required for CAS 741}

The information in this section will be used to evaluate the team members on the
graduate attribute of Lifelong Learning.

\input{../Reflection.tex}

\begin{enumerate}
  \item What went well while writing this deliverable? 
  \item What pain points did you experience during this deliverable, and how
    did you resolve them?
  \item What knowledge and skills will the team collectively need to acquire to
  successfully complete the verification and validation of your project?
  Examples of possible knowledge and skills include dynamic testing knowledge,
  static testing knowledge, specific tool usage, Valgrind etc.  You should look to
  identify at least one item for each team member.
  \item For each of the knowledge areas and skills identified in the previous
  question, what are at least two approaches to acquiring the knowledge or
  mastering the skill?  Of the identified approaches, which will each team
  member pursue, and why did they make this choice?
\end{enumerate}

\end{appendices}

\end{document}