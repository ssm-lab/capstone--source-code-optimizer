\documentclass[12pt, titlepage]{article}

\usepackage{amsmath, mathtools}

\usepackage[round]{natbib}
\usepackage{amsfonts}
\usepackage{amssymb}
\usepackage{graphicx}
\usepackage{colortbl}
\usepackage{xr}
\usepackage{hyperref}
\usepackage{longtable}
\usepackage{xfrac}
\usepackage{tabularx}
\usepackage{float}
\usepackage{siunitx}
\usepackage{booktabs}
\usepackage{multirow}
\usepackage[section]{placeins}
\usepackage{caption}
\usepackage{fullpage}

\hypersetup{
bookmarks=true,     % show bookmarks bar?
colorlinks=true,       % false: boxed links; true: colored links
linkcolor=red,          % color of internal links (change box color with linkbordercolor)
citecolor=blue,      % color of links to bibliography
filecolor=magenta,  % color of file links
urlcolor=cyan          % color of external links
}

\usepackage{array}

\externaldocument{../../SRS/SRS}

%% Comments

\usepackage{color}

% \newif\ifcomments\commentstrue %displays comments
\newif\ifcomments\commentsfalse %so that comments do not display

\ifcomments
\newcommand{\authornote}[3]{\textcolor{#1}{[#3 ---#2]}}
\newcommand{\todo}[1]{\textcolor{red}{[TODO: #1]}}
\else
\newcommand{\authornote}[3]{}
\newcommand{\todo}[1]{}
\fi

\newcommand{\wss}[1]{\authornote{blue}{SS}{#1}} 
\newcommand{\plt}[1]{\authornote{magenta}{TPLT}{#1}} %For explanation of the template
\newcommand{\an}[1]{\authornote{cyan}{Author}{#1}}

%% Common Parts

\newcommand{\progname}{Software Engineering} % PUT YOUR PROGRAM NAME HERE
\newcommand{\authname}{\textbf{Team 4, EcoOptimizers} \\
  \\ Nivetha Kuruparan
  \\ Sevhena Walker
  \\ Tanveer Brar
  \\ Mya Hussain
\\ Ayushi Amin} % AUTHOR NAMES

\usepackage{hyperref}
\hypersetup{colorlinks=true, linkcolor=blue, citecolor=blue, filecolor=blue,
urlcolor=blue, unicode=false}
\urlstyle{same}



\begin{document}

\title{Module Interface Specification for \progname{}}

\author{\authname}

\date{\today}

\maketitle

\pagenumbering{roman}

\section{Revision History}

\begin{tabularx}{\textwidth}{p{3cm}p{2cm}X}
\toprule {\bf Date} & {\bf Version} & {\bf Notes}\\
\midrule
Date 1 & 1.0 & Notes\\
Date 2 & 1.1 & Notes\\
\bottomrule
\end{tabularx}

~\newpage

\section{Symbols, Abbreviations and Acronyms}

See SRS Documentation at \wss{give url}

\wss{Also add any additional symbols, abbreviations or acronyms}

\newpage

\tableofcontents

\newpage

\pagenumbering{arabic}

\section{Introduction}

The following document details the Module Interface Specifications (MIS) for the Source Code Optimizer project. The Source Code Optimizer is a software tool designed to analyze, refactor, and optimize Python source code to improve energy efficiency, maintainability, and performance. This tool incorporates a combination of static code analysis using Pylint, abstract syntax tree (AST) parsing, and custom refactoring techniques to detect and address various code smells in Python programs.

The application allows developers to identify inefficient coding patterns, refactor them into optimized alternatives, and validate the results through built-in testing mechanisms. Key features include support for custom smell detection, energy profiling, and modular refactorers tailored to specific code smells, such as long method chains or inefficient list comprehensions. By automating parts of the optimization process, the Source Code Optimizer helps developers have the option of choosing to reduce emissions and produce more efficient software.

Complementary documents include the System Requirement Specifications (SRS) and Module Guide (MG). The full documentation and implementation can be found at: \url{https://github.com/ssm-lab/capstone--source-code-optimizer}

\section{Notation}

\wss{You should describe your notation.  You can use what is below as
  a starting point.}

The structure of the MIS for modules comes from \citet{HoffmanAndStrooper1995},
with the addition that template modules have been adapted from
\cite{GhezziEtAl2003}.  The mathematical notation comes from Chapter 3 of
\citet{HoffmanAndStrooper1995}.  For instance, the symbol := is used for a
multiple assignment statement and conditional rules follow the form $(c_1
\Rightarrow r_1 | c_2 \Rightarrow r_2 | ... | c_n \Rightarrow r_n )$.

The following table summarizes the primitive data types used by \progname. 

\begin{center}
\renewcommand{\arraystretch}{1.2}
\noindent 
\begin{tabular}{l l p{7.5cm}} 
\toprule 
\textbf{Data Type} & \textbf{Notation} & \textbf{Description}\\ 
\midrule
character & char & a single symbol or digit\\
integer & $\mathbb{Z}$ & a number without a fractional component in (-$\infty$, $\infty$) \\
natural number & $\mathbb{N}$ & a number without a fractional component in [1, $\infty$) \\
real & $\mathbb{R}$ & any number in (-$\infty$, $\infty$)\\
\bottomrule
\end{tabular} 
\end{center}

\noindent
The specification of \progname \ uses some derived data types: sequences, strings, and
tuples. Sequences are lists filled with elements of the same data type. Strings
are sequences of characters. Tuples contain a list of values, potentially of
different types. In addition, \progname \ uses functions, which
are defined by the data types of their inputs and outputs. Local functions are
described by giving their type signature followed by their specification.

\section{Module Decomposition}

The following table is taken directly from the Module Guide document for this project.

\begin{table}[h!]
\centering
\begin{tabular}{p{0.3\textwidth} p{0.6\textwidth}}
\toprule
\textbf{Level 1} & \textbf{Level 2}\\
\midrule

{Hardware-Hiding} & ~ \\
\midrule

\multirow{7}{0.3\textwidth}{Behaviour-Hiding} & Input Parameters\\
& Output Format\\
& Output Verification\\
& Temperature ODEs\\
& Energy Equations\\ 
& Control Module\\
& Specification Parameters Module\\
\midrule

\multirow{3}{0.3\textwidth}{Software Decision} & {Sequence Data Structure}\\
& ODE Solver\\
& Plotting\\
\bottomrule

\end{tabular}
\caption{Module Hierarchy}
\label{TblMH}
\end{table}

\newpage
~\newpage

\section{MIS of \wss{Module Name}} \label{Module} \wss{Use labels for
  cross-referencing}

\wss{You can reference SRS labels, such as R\ref{R_Inputs}.}

\wss{It is also possible to use \LaTeX for hypperlinks to external documents.}

\subsection{Module}

\wss{Short name for the module}

\subsection{Uses}


\subsection{Syntax}

\subsubsection{Exported Constants}

\subsubsection{Exported Access Programs}

\begin{center}
\begin{tabular}{p{2cm} p{4cm} p{4cm} p{2cm}}
\hline
\textbf{Name} & \textbf{In} & \textbf{Out} & \textbf{Exceptions} \\
\hline
\wss{accessProg} & - & - & - \\
\hline
\end{tabular}
\end{center}

\subsection{Semantics}

\subsubsection{State Variables}

\wss{Not all modules will have state variables.  State variables give the module
  a memory.}

\subsubsection{Environment Variables}

\wss{This section is not necessary for all modules.  Its purpose is to capture
  when the module has external interaction with the environment, such as for a
  device driver, screen interface, keyboard, file, etc.}

\subsubsection{Assumptions}

\wss{Try to minimize assumptions and anticipate programmer errors via
  exceptions, but for practical purposes assumptions are sometimes appropriate.}

\subsubsection{Access Routine Semantics}

\noindent \wss{accessProg}():
\begin{itemize}
\item transition: \wss{if appropriate} 
\item output: \wss{if appropriate} 
\item exception: \wss{if appropriate} 
\end{itemize}

\wss{A module without environment variables or state variables is unlikely to
  have a state transition.  In this case a state transition can only occur if
  the module is changing the state of another module.}

\wss{Modules rarely have both a transition and an output.  In most cases you
  will have one or the other.}

\subsubsection{Local Functions}

\wss{As appropriate} \wss{These functions are for the purpose of specification.
  They are not necessarily something that is going to be implemented
  explicitly.  Even if they are implemented, they are not exported; they only
  have local scope.}

\newpage

\section{MIS of Base Refactorer} \label{mis:baseR}

\texttt{BaseRefactorer}

\subsection{Module}

The interface that all refactorers of this system will inherit from.

\subsection{Uses}

None

\subsection{Syntax}
\noindent
\textbf{Exported Constants}: None

\noindent
\textbf{Exported Access Programs}:

\begin{tabularx}{\linewidth}{|l|>{\raggedright\arraybackslash}X|l|l|}
  \toprule Name & In & Out & Exceptions \\\hline
  \midrule
  \texttt{\_\_init\_\_} & \texttt{output\_dir: Path} & None & None \\\hline
  \texttt{refactor} & \texttt{file\_path: Path, pylint\_smell: dict, initial\_emissions: float} & None & None \\
  \hline
  \bottomrule
\end{tabularx}

\subsection{Semantics}

\subsubsection{State Variables}
\begin{itemize}
  \item \texttt{temp\_dir: Path}: Directory path for storing refactored files.
\end{itemize}

\subsubsection{Environment Variables}
None

\subsubsection{Assumptions}
\begin{itemize}
  \item \texttt{output\_dir} exists or can be created, and write permissions are available.
\end{itemize}

\subsubsection{Access Routine Semantics}

\paragraph{\texttt{\_\_init\_\_(self, output\_dir: Path)}}
\begin{itemize}
  \item \textbf{transition}: Initializes the \texttt{temp\_dir} variable within \texttt{output\_dir}.
  \item \textbf{output}: None.
  \item \textbf{exception:} None.
\end{itemize}

\paragraph{\texttt{refactor(self, file\_path: Path, pylint\_smell: dict, initial\_emissions: float)}}
\begin{itemize}
  \item \textbf{transition}: Abstract method. No transition defined.
  \item \textbf{output}: None.
  \item \textbf{exception:} None.
\end{itemize}

\subsubsection{Local Functions}
None.

\newpage

\section{MIS of Smell Data Type} \label{mis:smell}
\texttt{Smell}

\subsection{Module}
Contains data related to a code smell.

\subsection{Uses}
None

\subsection{Syntax}
\noindent
\textbf{Exported Constants}: None

\noindent
\textbf{Exported Access Programs}: None

\subsection{Semantics}

\subsubsection{State Variables}
\begin{itemize}
  \item \texttt{absolutePath: str}: Absolute path to the source file containing the smell.
  \item \texttt{column: int}: Starting column in the source file where the smell is detected.
  \item \texttt{confidence: str}: Confidence level for the smell detection.
  \item \texttt{endColumn: int | None}: Ending column for the smell location, if applicable.
  \item \texttt{endLine: int | None}: Ending line number for the smell location, if applicable.
  \item \texttt{occurences: dict}: Contains positional data related to where the smell is located in a code file.
  \item \texttt{message: str}: Descriptive message explaining the smell.
  \item \texttt{messageId: str}: Unique identifier for the specific message or warning.
  \item \texttt{module: str}: Module or component name containing the smell.
  \item \texttt{obj: str}: Specific object associated with the smell.
  \item \texttt{path: str}: Relative path to the source file from the project root.
  \item \texttt{symbol: str}: Symbol or code construct involved in the smell.
  \item \texttt{type: str}: Type or category of the smell.
\end{itemize}

\subsubsection{Environment Variables}
None

\subsubsection{Assumptions}
\begin{itemize}
  \item All values provided to the fields of \texttt{Smell} conform to the expected data types and constraints.
\end{itemize}

\subsubsection{Access Routine Semantics}

\paragraph{\texttt{Smell()}}
\begin{itemize}
  \item \textbf{transition}: Creates a dictionary-like structure with the defined attributes representing a code smell.
  \item \textbf{output}: Returns a \texttt{Smell} instance.
\end{itemize}

\subsubsection{Local Functions}
None.
  

\newpage

\section{MIS of Use List Accumulation Refactorer} \label{mis:ListAccum}

\texttt{UseListAccumulationRefactorer}

\subsection{Module}

The \texttt{UseListAccumulationRefactorer} module identifies and refactors 
string concatenations in loops in Python code to improve the performance and energy efficiency of the software. It specifically handles these concatenations by, instead, adding the string for each iteration to a list that is then converted to a string using Python's \texttt{join()} function, ensuring proper refactoring while maintaining the original functionality.

\subsection{Uses}
\begin{itemize}
  \item Uses \texttt{Smell} interface for data access
  \item Inherits from \texttt{BaseRefactorer}
  \item Inherits from Python's \texttt{ast} module's \texttt{NodeTransformer}
\end{itemize}
  
\subsection{Syntax}
\noindent
\textbf{Exported Constants}: None

\noindent
\textbf{Exported Access Programs}:
  
\begin{tabularx}{\linewidth}{|
    l|
    >{\raggedright\arraybackslash}X|
    l|
    l|}
  \toprule Name & In & Out & Exceptions \\
  \midrule
  \texttt{\_\_init\_\_} & \texttt{output\_dir: Path} & None & None \\
  \hline
  \texttt{refactor} & \texttt{file\_path: Path, pylint\_smell: Smell, initial\_emissions: Real} & None & \texttt{TypeError}, \texttt{IOError} \\
  \hline
  \texttt{visit} & \texttt{node: nodes.NodeNG} & None & None \\
  \hline
  \texttt{find\_last\_assignment} & \texttt{scope: nodes.NodeNG} & None & \texttt{TypeError} \\
  \hline
  \texttt{find\_scope} & None & None & \texttt{TypeError} \\
  \hline
  \texttt{add\_node\_to\_body} & \texttt{code\_file: str} & \texttt{str} & \texttt{TypeError} \\
  \bottomrule
\end{tabularx}
  
\subsection{Semantics}
  
\subsubsection{State Variables}
\begin{itemize}
  \item \texttt{target\_line: int}: Line number where refactoring is applied.
  \item \texttt{target\_node: ASTnode}: Node representing the concatenation variable.
  \item \texttt{assign\_var: str}: Name of the variable the \texttt{target\_node} represents.
  \item \texttt{last\_assign\_node: ASTnode}: Last initialization/assignment of the \texttt{assign\_var} prior to the start of the loop.
  \item \texttt{concat\_node: ASTnode}: Node where concatenation occurs.
  \item \texttt{scope\_node: ASTnode}: Scope where refactoring is inserted.
  \item \texttt{outer\_loop: ASTnode}: Outermost loop before the start of the concatenation.
\end{itemize}
  
\subsubsection{Environment Variables}
None
  
\subsubsection{Assumptions}
\begin{itemize}
  \item The input file contains valid Python syntax.
  \item \texttt{pylint\_smell} provides a valid line number for the detected code smell.
\end{itemize}
  
\subsubsection{Access Routine Semantics}
  
\paragraph{\texttt{\_\_init\_\_(self, output\_dir: Path)}}
\begin{itemize}
  \item \textbf{transition}: Initializes the refactorer with \texttt{output\_dir} and sets default state variables.
  \item \textbf{output}: None.
  \item \textbf{exception}: None
\end{itemize}
  
\paragraph{\texttt{refactor(self, file\_path: Path, pylint\_smell: Smell, initial\_emissions: float)}}
\begin{itemize}
  \item \textbf{transition}: Parses \texttt{file\_path}, identifies string concatenations in loops, modifies code for list accumulation, and writes refactored code to a file.
  \item \textbf{output}: None.
  \item \textbf{exception}: Raises \texttt{IOError} if input file cannot be read. Raises \texttt{TypeError} if source file cannot be parsed into an AST.
\end{itemize}

\paragraph{\texttt{find\_last\_assignment(self, scope: nodes.NodeNG)}}
\begin{itemize}
  \item \textbf{transition}: Identifies the last assignment of \texttt{assign\_var} within the given \texttt{scope}.
  \item \textbf{output}: None.
  \item \textbf{exception}: Raises \texttt{TypeError} if given scope is null.
\end{itemize}

\paragraph{\texttt{find\_scope(self)}}
\begin{itemize}
  \item \textbf{transition}: Finds the scope for refactoring based on AST node ancestry.
  \item \textbf{output}: None.
  \item \textbf{exception}: Raises \texttt{TypeError} if \texttt{concat\_node} is not set.
\end{itemize}

\paragraph{\texttt{add\_node\_to\_body(self, code\_file: str)}}
\begin{itemize}
  \item \textbf{transition}: Inserts list accumulation and join statements into \texttt{code\_file}.
  \item \textbf{output}: Returns the modified source code as a string.
  \item \textbf{exception}: Raises \texttt{TypeError} if \texttt{target\_node} or \texttt{outer\_loop} is not set.
\end{itemize}

\subsubsection{Local Functions}
Functions for internal AST parsing, node manipulation, and validation are defined within the class but are not exported.
  
\newpage

\section{MIS of Make Method Static Refactorer} \label{mis:MakeStatic}

\texttt{MakeStaticRefactorer}

\subsection{Module}

The \texttt{MakeStaticRefactorer} module identifies and refactors 
class methods that don't make use of their instance attributes to improve the readability, performance and energy efficiency of the software. It specifically handles these methods by turning them into static functions and ensuring any calls to this method use the proper calling syntax. This ensures proper refactoring while maintaining the original functionality.

\subsection{Uses}
\begin{itemize}
  \item Uses \texttt{Smell} interface for data access
  \item Inherits from \texttt{BaseRefactorer}
  \item Inherits from Python's \texttt{ast} module's \texttt{NodeTransformer}
\end{itemize}
  
\subsection{Syntax}
\noindent
\textbf{Exported Constants}: None

\noindent
\textbf{Exported Access Programs}:
  
\begin{tabularx}{\linewidth}{|l|>{\raggedright\arraybackslash}X|l|l|}
  \toprule Name & In & Out & Exceptions \\
  \midrule
  \texttt{\_\_init\_\_} & \texttt{output\_dir: Path} & None & None \\
  \hline
  \texttt{refactor} & \texttt{file\_path: Path, pylint\_smell: Smell, initial\_emissions: $\mathbb{R}$} & None & \texttt{TypeError}, \texttt{IOError} \\
  \hline
  \texttt{visit\_FunctionDef} & \texttt{node: FunctionDef} & \texttt{FunctionDef} & None \\
  \hline
  \texttt{visit\_ClassDef} & \texttt{node: ClassDef} & \texttt{ClassDef} & None \\
  \hline
  \texttt{visit\_Call} & \texttt{node: Call} & \texttt{Call} & None \\
  \bottomrule
\end{tabularx}
  
\subsection{Semantics}
  
\subsubsection{State Variables}
\begin{itemize}
  \item \texttt{target\_line: int}: Line number where refactoring is applied.
  \item \texttt{mim\_method\_class: str}: Class name containing the method to refactor.
  \item \texttt{mim\_method: str}: Method name to refactor.
\end{itemize}
  
\subsubsection{Environment Variables}
None
  
\subsubsection{Assumptions}
\begin{itemize}
  \item The input file contains valid Python syntax.
  \item \texttt{pylint\_smell} provides a valid line number for the detected code smell.
\end{itemize}
  
\subsubsection{Access Routine Semantics}
  
\paragraph{\texttt{\_\_init\_\_(self, output\_dir: Path)}}
\begin{itemize}
  \item \textbf{transition}: Initializes the refactorer with \texttt{output\_dir} and sets default state variables.
  \item \textbf{output}: None.
  \item \textbf{exception}: None.
\end{itemize}
  
\paragraph{\texttt{refactor(self, file\_path: Path, pylint\_smell: Smell, initial\_emissions: float)}}
\begin{itemize}
  \item \textbf{transition}: Parses \texttt{file\_path}, identifies the target function, modifies it to be static, and validates refactoring.
  \item \textbf{output}: None.
  \item \textbf{exception}: Raises \texttt{IOError} if input file cannot be read. Raises \texttt{TypeError} if source file cannot be parsed into an AST.
\end{itemize}

\paragraph{\texttt{visit\_FunctionDef(self, node: ast.FunctionDef)}}
\begin{itemize}
  \item \textbf{transition}: Adds the \texttt{staticmethod} decorator to the target method and removes the \texttt{self} parameter if present.
  \item \textbf{output}: Returns the modified \texttt{FunctionDef} node.
  \item \textbf{exception}: None
\end{itemize}

\paragraph{\texttt{visit\_ClassDef(self, node: ast.ClassDef)}}
\begin{itemize}
  \item \textbf{transition}: Identifies the class containing the target method.
  \item \textbf{output}: Returns the modified \texttt{ClassDef} node.
  \item \textbf{exception}: None.
\end{itemize}

\paragraph{\texttt{visit\_Call(self, node: ast.Call)}}
\begin{itemize}
  \item \textbf{transition}: Updates method call references to use the class name instead of \texttt{self}.
  \item \textbf{output}: Returns the modified \texttt{Call} node.
  \item \textbf{exception}: None.
\end{itemize}

\subsubsection{Local Functions}
Functions for internal AST parsing, node transformation, and validation are defined within the class but are not exported.
  
\newpage  

\section{MIS of LongElementChainRefactorer}

\subsection{Module}

LongElementChainRefactorer is a module that refactors long element chains, specifically focusing on flattening nested dictionaries to improve readability, maintainability, and energy efficiency. The module uses a recursive flattening strategy while caching previously seen patterns for optimization.

\subsection{Uses}

\begin{itemize}
    \item Uses \texttt{Smell} interface for data access
    \item Inherits from \texttt{BaseRefactorer}
\end{itemize}

\subsection{Syntax}

\subsubsection{Exported Constants}
None

\subsubsection{Exported Access Programs}

\begin{center}
\begin{tabular}{|p{3cm}|p{5cm}|p{2cm}|p{3cm}|}
\hline
\textbf{Name} & \textbf{In} & \textbf{Out} & \textbf{Exceptions} \\
\hline
\texttt{\_\_init\_\_} & \texttt{output\_dir: Path} & None & None \\
\hline
\texttt{refactor} & \texttt{file\_path: Path, pylint\_smell: Smell, initial\_emissions: float} & None & Logging exceptions \\
\hline
\end{tabular}
\end{center}

\subsection{Semantics}

\subsubsection{State Variables}

\begin{itemize}
  \item \textbf{\_reference\_map}: Maps element chain references to their line numbers and corresponding values.
\end{itemize}

\subsubsection{Environment Variables}

\begin{itemize}
  \item \textbf{File system}: Used to read, write, and store temporary and refactored files.
  \item \textbf{Logger}: Logs information during the refactoring process.
\end{itemize}

\subsubsection{Assumptions}

\begin{itemize}
  \item Input files are valid Python scripts.
  \item Smells identified by \textbf{pylint\_smell} include valid line numbers.
  \item Refactored code must pass the provided test suite.
\end{itemize}

\subsubsection{Access Routine Semantics}

\paragraph{\texttt{\_\_init\_\_(output\_dir: Path)}}
\begin{itemize}
\item \textbf{Transition}: Initializes the refactorer with the specified output directory and sets up internal caching structures.
\item \textbf{Output}: None.
\item \textbf{Exception}: None.
\end{itemize}

\paragraph{\texttt{refactor(file\_path: Path, pylint\_smell: Smell, initial\_emissions: float)}}
\begin{itemize}
  \item \textbf{Transition}:
    \begin{itemize}
      \item Reads the file at \texttt{file\_path}.
      \item Identifies nested dictionary chains for flattening.
      \item Refactors the identified chain by flattening the dictionary and replacing its occurrences.
      \item Writes the refactored code to a temporary file.
\end{itemize}
  \item \textbf{Output}: None. Refactored file is saved if improvements are validated.
  \item \textbf{Exception}: Logs exceptions during file operations or refactoring.
\end{itemize}

\subsubsection{Local Functions}
\begin{itemize}
    \item \textbf{\_flatten\_dict(d: dict[str, Any], parent\_key: str = "")} \\
    Recursively flattens a nested dictionary by combining keys with underscores.

    \item \textbf{\_extract\_dict\_literal(node: ast.AST)} \\
    Converts an Abstract Syntax Tree (AST) dictionary literal into a Python dictionary.

    \item \textbf{\_find\_dict\_assignments(tree: ast.AST, name: str)} \\
    Extracts dictionary assignments given the name of the dictionary from the AST and returns them as a dictionary.

    \item \textbf{\_collect\_dict\_references(tree: ast.AST)} \\
    Identifies and stores all dictionary access patterns in the `\_reference\_map`.

    \item \textbf{\_generate\_flattened\_access(base\_var: str, access\_chain: list[str])} \\
    Generates a flattened dictionary key string by combining elements of an access chain with underscores.
\end{itemize}



\section{MIS of Measurements Module}

\subsection{Module}

The MeasurementsModule is a module designed to measure and track the carbon emissions generated by executing Python scripts. By leveraging the CodeCarbon library, it allows developers to assess the environmental impact of their code execution. The module runs a specified Python file, tracks the associated carbon emissions during the execution, and logs the results for further analysis. It provides functionality for measuring, logging, and extracting emissions data in a structured manner to help improve energy efficiency in software development.

\subsection{Uses}

\begin{itemize}
    \item Uses \texttt{CodeCarbon} library for track energy consumption
    \item Uses \texttt{TemporaryDirectory} to store temporary files
    \item Inherits from \texttt{BaseEnergyMeter}
\end{itemize}

\subsection{Syntax}

\subsubsection{Exported Constants}
None

\subsubsection{Exported Access Programs}

\begin{center}
\begin{tabular}{|p{3cm}|p{5cm}|p{2cm}|p{3cm}|}
\hline
\textbf{Name} & \textbf{In} & \textbf{Out} & \textbf{Exceptions} \\
\hline
\texttt{\_\_init\_\_} & \texttt{output\_dir: Path} & None & None \\
\hline
\texttt{measure\_energy} & \texttt{None} & None & CalledProcessError and FileReading exceptions \\
\hline
\end{tabular}
\end{center}

\subsection{Semantics}

\subsubsection{State Variables}

\begin{itemize}
    \item \textbf{Emissions\_data}: Stores the emissions data extracted from the CSV file generated by CodeCarbon. It is populated after the energy measurement process completes successfully. The value is either a dictionary containing the last row of emissions data or \texttt{None} if no data was extracted due to an error.

\end{itemize}

\subsubsection{Environment Variables}

\begin{itemize}
  \item \textbf{TEMP}: Sets the temporary directory location for Windows systems. Used during the CodeCarbon energy measurement process.
  \item \textbf{TMPDIR}: Sets the temporary directory location for Unix-based systems. Used during the CodeCarbon energy measurement process.
  \item \textbf{Logger}: A logging mechanism that logs various events during the energy measurement process, including errors, completion of measurements, and other key actions.
\end{itemize}

\subsubsection{Assumptions}

\begin{itemize}
  \item The file at \texttt{file\_path} is a valid Python script.
  \item The CodeCarbon tool is properly installed and configured.
  \item The \texttt{EmissionsTracker} can successfully execute the Python script specified by \texttt{file\_path}.
  \item The emissions data is captured in a CSV format and can be extracted correctly.
  \item The temporary directories are correctly set up and accessible during execution.
\end{itemize}

\subsubsection{Access Routine Semantics}
\paragraph{\texttt{\_\_init\_\_(file\_path: Path)}}
\begin{itemize}
  \item \textbf{Transition}: Initializes the \texttt{CodeCarbonEnergyMeter} with the specified file path and logger. It sets up the necessary internal state for energy measurement and prepares the environment.
  \item \textbf{Output}: None.
  \item \textbf{Exception}: None.
\end{itemize}

\paragraph{\texttt{measure\_energy()}}
\begin{itemize}
  \item \textbf{Transition}:
    \begin{itemize}
      \item Logs the start of the energy measurement process.
      \item Creates a temporary directory to store custom data.
      \item Initializes the \texttt{EmissionsTracker} from CodeCarbon.
      \item Runs the script specified by \texttt{file\_path} and captures the output.
      \item Stops the tracker after execution and stores the emissions data.
      \item If available, it extracts the emissions data from the generated CSV file.
    \end{itemize}
  \item \textbf{Output}: 
    \begin{itemize}
        \item Logs the results of the energy measurement process.
        \item Stores the emissions data in \texttt{self.emissions\_data}.
    \end{itemize}
   \item \textbf{Exception}: 
      \begin{itemize}
        \item Logs an error if the file cannot be executed or if the emissions file is not created.
        \item If the emissions data cannot be extracted from the CSV file, logs the issue.
      \end{itemize}
 \end{itemize}

\subsubsection{Local Functions}
\paragraph{\texttt{\_extract\_emissions\_csv(csv\_file\_path: Path)}}
    
    Extracts emissions data from a CSV file generated by CodeCarbon.
    \begin{itemize}
        \item \textbf{Input}: \texttt{csv\_file\_path} - The path to the CSV file containing emissions data.
        \item \textbf{Output}: Returns the last row of emissions data as a dictionary, or \texttt{None} if an error occurs.
    \end{itemize}

  
  
\newpage

\section{MIS of Pylint Analyzer} \label{mis:PylintAnalyzer}

\texttt{PylintAnalyzer}

\subsection{Module}

The \texttt{PylintAnalyzer} module performs static code analysis on Python files using Pylint, with additional custom checks for detecting specific code smells. It outputs detected smells in a structured format for further processing.

\subsection{Uses}
\begin{itemize}
  \item Uses Python's \texttt{pylint} library for code analysis
  \item Uses \texttt{ast} module for parsing and analyzing abstract syntax trees
  \item Uses \texttt{astor} library for converting AST nodes back to source code
  \item Integrates with custom checkers, including \texttt{StringConcatInLoopChecker}
  \item Accesses configuration settings from \texttt{analyzers\_config}
\end{itemize}

\subsection{Syntax}
\noindent
\textbf{Exported Constants}: None

\noindent
\textbf{Exported Access Programs}:\\
{\footnotesize
\begin{tabularx}{\linewidth}{|
    l|
    >{\raggedright\arraybackslash}X|
    l|
    l|}
  \toprule Name & In & Out & Exceptions \\
  \midrule
  \texttt{\_\_init\_\_} & \texttt{file\_path: Path, source\_code: ast.Module} & None & None \\
  \hline
  \texttt{build\_pylint\_options} & None & \texttt{list[str]} & None \\
  \hline
  \texttt{analyze} & None & None & \texttt{JSONDecodeError}, \texttt{Exception} \\
  \hline
  \texttt{configure\_smells} & None & None & None \\
  \hline
  \texttt{filter\_for\_one\_code\_smell} & \texttt{pylint\_results: list[Smell], code: str} & \texttt{list[Smell]} & None \\
  \hline
  \texttt{detect\_long\_message\_chain} & \texttt{threshold: int = 3} & \texttt{list[Smell]} & None \\
  \hline
  \texttt{detect\_long\_lambda\_expression} & \texttt{threshold\_length: int = 100, threshold\_count: int = 3} & \texttt{list[Smell]} & None \\
  \hline
  \texttt{detect\_long\_element\_chain} & \texttt{threshold: int = 3} & \texttt{list[Smell]} & None \\
  \hline
  \texttt{detect\_repeated\_calls} & \texttt{threshold: int = 2} & \texttt{list[Smell]} & None \\
  \bottomrule
\end{tabularx}
}

\subsection{Semantics}

\subsubsection{State Variables}
\begin{itemize}
  \item \texttt{file\_path: Path}: The path to the Python file being analyzed.
  \item \texttt{source\_code: ast.Module}: The parsed abstract syntax tree of the source file.
  \item \texttt{smells\_data: list[dict]}: A list of detected code smells, represented as dictionaries.
\end{itemize}

\subsubsection{Environment Variables}
None

\subsubsection{Assumptions}
\begin{itemize}
  \item The input file is valid Python code and can be parsed into an AST.
  \item Configuration settings, such as extra Pylint options and custom smell definitions, are valid.
\end{itemize}

\subsubsection{Access Routine Semantics}

\paragraph{\texttt{\_\_init\_\_(self, file\_path: Path, source\_code: ast.Module)}}
\begin{itemize}
  \item \textbf{transition}: Initializes the analyzer with the provided file path and AST of the source code.
  \item \textbf{output}: None.
  \item \textbf{exception:} None.
\end{itemize}

\paragraph{\texttt{build\_pylint\_options(self)}}
\begin{itemize}
  \item \textbf{transition}: Constructs the list of Pylint options based on the file path and configuration settings.
  \item \textbf{output}: Returns a list of strings representing Pylint options.
  \item \textbf{exception:} None.
\end{itemize}

\paragraph{\texttt{analyze(self)}}
\begin{itemize}
  \item \textbf{transition}: Executes Pylint analysis and custom checks, populating \texttt{smells\_data} with detected smells.
  \item \textbf{output}: None.
  \item \textbf{exception:} Raises \texttt{JSONDecodeError} if Pylint's output cannot be parsed. Raises \texttt{Exception} for other runtime errors.
\end{itemize}

\paragraph{\texttt{configure\_smells(self)}}
\begin{itemize}
  \item \textbf{transition}: Filters \texttt{smells\_data} to include only configured smells.
  \item \textbf{output}: None.
  \item \textbf{exception:} None.
\end{itemize}

\paragraph{\texttt{filter\_for\_one\_code\_smell(self, pylint\_results: list[Smell], code: str)}}
\begin{itemize}
  \item \textbf{transition}: Filters the given Pylint results for a specific code smell identified by \texttt{code}.
  \item \textbf{output}: Returns a list of smells matching the specified code.
  \item \textbf{exception:} None.
\end{itemize}

\paragraph{\texttt{detect\_long\_message\_chain(self, threshold: int = 3)}}
\begin{itemize}
  \item \textbf{transition}: Identifies method chains exceeding the specified \texttt{threshold}.
  \item \textbf{output}: Returns a list of smells for long method chains.
  \item \textbf{exception:} None.
\end{itemize}

\paragraph{\texttt{detect\_long\_lambda\_expression(self, threshold\_length: int = 100, threshold\_count: int = 3)}}
\begin{itemize}
  \item \textbf{transition}: Detects lambda expressions exceeding length or expression count thresholds.
  \item \textbf{output}: Returns a list of smells for long lambda expressions.
  \item \textbf{exception:} None.
\end{itemize}

\paragraph{\texttt{detect\_long\_element\_chain(self, threshold: int = 3)}}
\begin{itemize}
  \item \textbf{transition}: Detects dictionary access chains exceeding the specified \texttt{threshold}.
  \item \textbf{output}: Returns a list of smells for long dictionary chains.
  \item \textbf{exception:} None.
\end{itemize}

\paragraph{\texttt{detect\_repeated\_calls(self, threshold: int = 2)}}
\begin{itemize}
  \item \textbf{transition}: Identifies repeated function calls exceeding the \texttt{threshold}.
  \item \textbf{output}: Returns a list of smells for repeated function calls.
  \item \textbf{exception:} None.
\end{itemize}

\subsubsection{Local Functions}
\begin{itemize}
  \item \texttt{parse\_line(file\_path: Path, line: int)}: Parses a specific line of code into an AST node.
  \item \texttt{get\_lambda\_code(lambda\_node: ast.Lambda)}: Returns the string representation of a lambda expression.
\end{itemize}


\newpage

\section{MIS of Testing Functionality}

\texttt{TestRunner}

\subsection{Module}

Responsible for validating that any refactorings made to the source code do not modify it's original functionality.

\subsection{Uses}
\begin{itemize}
  \item Uses Python's subprocess library
\end{itemize}

\subsection{Syntax}
\noindent
\textbf{Exported Constants}: None

\noindent
\textbf{Exported Access Programs}:

\begin{tabularx}{\linewidth}{|l|>{\raggedright\arraybackslash}X|l|l|}
\hline
Name & In & Out & Exceptions \\
\hline
\texttt{\_\_init\_\_} & \texttt{run\_command: str, project\_path: Path} & None & None \\
\hline
\texttt{retained\_functionality} & None & \texttt{bool} & \texttt{CalledProcessError} \\
\hline
\end{tabularx}

\subsection{Semantics}

\subsubsection{State Variables}
\begin{itemize}
  \item \texttt{project\_path: Path}: Path to the source code directory.
  \item \texttt{run\_command: str}: Command used to run the tests.
\end{itemize}

\subsubsection{Environment Variables}
None

\subsubsection{Assumptions}
\begin{itemize}
  \item The provided \texttt{run\_command} is a valid shell command.
  \item \texttt{project\_path} is a valid path working source code directory.
\end{itemize}

\subsubsection{Access Routine Semantics}

\paragraph{\texttt{\_\_init\_\_(self, run\_command: str, project\_path: Path)}}
\begin{itemize}
  \item \textbf{transition}: Initializes the test runner with the given \texttt{run\_command} and \texttt{project\_path}.
  \item \textbf{output}: None.
  \item \textbf{exception}: None.
\end{itemize}

\paragraph{\texttt{retained\_functionality(self)}}
\begin{itemize}
  \item \textbf{transition}: Runs the specified test command in the given project path. Logs success or failure, including standard output and error streams.
  \item \textbf{output}: Returns \texttt{True} if the tests passed; otherwise, returns \texttt{False}.
  \item \textbf{exception}: Raises a \texttt{CalledProcessError} if an eror occurs while running the tests in a subprocess.
\end{itemize}

\subsubsection{Local Functions}
None.

\newpage

\section{MIS of Use A Generator Refactorer} \label{mis:UseGen}

\texttt{UseAGeneratorRefactorer}

\subsection{Module}

The \texttt{UseAGeneratorRefactorer} module identifies and refactors 
unnecessary list comprehensions in Python code by converting them to generator expressions. This refactoring improves energy efficiency while maintaining the original functionality.

\subsection{Uses}
\begin{itemize}
  \item Uses \texttt{Smell} interface for data access
  \item Inherits from \texttt{BaseRefactorer}
  \item Uses Python's \texttt{ast} module for parsing and manipulating abstract syntax trees
\end{itemize}

\subsection{Syntax}
\noindent
\textbf{Exported Constants}: None

\noindent
\textbf{Exported Access Programs}:\\
\begin{tabularx}{\linewidth}{|
    l|
    >{\raggedright\arraybackslash}X|
    l|
    l|}
  \toprule Name & In & Out & Exceptions \\
  \midrule
  \texttt{\_\_init\_\_} & \texttt{output\_dir: Path} & None & None \\
  \hline
  \texttt{refactor} & \texttt{file\_path: Path, pylint\_smell: Smell, initial\_emissions: float} & None & \texttt{IOError}, \texttt{TypeError} \\
  \hline
  \texttt{\_replace\_node} & \texttt{tree: ast.Module, old\_node: ast.ListComp, new\_node: ast.GeneratorExp} & None & None \\
  \bottomrule
\end{tabularx}

\subsection{Semantics}

\subsubsection{State Variables}
\begin{itemize}
  \item \texttt{temp\_dir: Path}: Directory path for storing refactored files.
  \item \texttt{output\_dir: Path}: Directory path for saving final refactored code.
\end{itemize}

\subsubsection{Environment Variables}
None

\subsubsection{Assumptions}
\begin{itemize}
  \item The input file contains valid Python syntax.
  \item \texttt{pylint\_smell} provides a valid line number for the detected code smell.
\end{itemize}

\subsubsection{Access Routine Semantics}

\paragraph{\texttt{\_\_init\_\_(self, output\_dir: Path)}}
\begin{itemize}
  \item \textbf{transition}: Initializes the \texttt{temp\_dir} variable within \texttt{output\_dir}.
  \item \textbf{output}: None.
  \item \textbf{exception:} None.
\end{itemize}

\paragraph{\texttt{refactor(self, file\_path: Path, pylint\_smell: Smell, initial\_emissions: float)}}
\begin{itemize}
  \item \textbf{transition}: Parses \texttt{file\_path}, identifies unnecessary list comprehensions, modifies the code to use generator expressions, and validates refactoring.
  \item \textbf{output}: None.
  \item \textbf{exception}: Raises \texttt{IOError} if input file cannot be read. Raises \texttt{TypeError} if source file cannot be parsed into an AST.
\end{itemize}

\paragraph{\texttt{\_replace\_node(self, tree: ast.Module, old\_node: ast.ListComp, new\_node: ast.GeneratorExp)}}
\begin{itemize}
  \item \textbf{transition}: Replaces an \texttt{old\_node} in the AST with a \texttt{new\_node}.
  \item \textbf{output}: None.
  \item \textbf{exception}: None.
\end{itemize}

\subsubsection{Local Functions}
Functions for internal AST parsing, node manipulation, and validation are defined within the class but are not exported.

\newpage

\section{MIS of Cache Repeated Calls Refactorer} \label{mis:CacheCalls}

\texttt{CacheRepeatedCallsRefactorer}

\subsection{Module}

The \texttt{CacheRepeatedCallsRefactorer} module identifies repeated function calls in Python code and refactors them by caching the result of the first call to a temporary variable. This refactoring improves performance and energy efficiency while preserving the original functionality.

\subsection{Uses}
\begin{itemize}
  \item Uses \texttt{Smell} interface for data access
  \item Inherits from \texttt{BaseRefactorer}
  \item Uses Python's \texttt{ast} module for parsing and manipulating abstract syntax trees
\end{itemize}

\subsection{Syntax}
\noindent
\textbf{Exported Constants}: None

\noindent
\textbf{Exported Access Programs}:\\
\begin{tabularx}{\linewidth}{|l|>{\raggedright\arraybackslash}X|l|l|}
  \toprule Name & In & Out & Exceptions \\
  \midrule
  \texttt{\_\_init\_\_} & \texttt{output\_dir: Path} & None & None \\
  \hline
  \texttt{refactor} & \texttt{file\_path: Path, pylint\_smell: Smell, initial\_emissions: float} & None & \texttt{IOError}, \texttt{TypeError} \\
  \bottomrule
\end{tabularx}

\subsection{Semantics}

\subsubsection{State Variables}
\begin{itemize}
  \item \texttt{cached\_var\_name: str}: Name of the temporary variable used for caching.
  \item \texttt{target\_line: int}: Line number where refactoring is applied.
\end{itemize}

\subsubsection{Environment Variables}
None

\subsubsection{Assumptions}
\begin{itemize}
  \item The input file contains valid Python syntax.
  \item \texttt{pylint\_smell} provides valid occurrences of repeated calls with line numbers and call strings.
\end{itemize}

\subsubsection{Access Routine Semantics}

\paragraph{\texttt{\_\_init\_\_(self, output\_dir: Path)}}
\begin{itemize}
  \item \textbf{transition}: Initializes the \texttt{temp\_dir} variable within \texttt{output\_dir}.
  \item \textbf{output}: None.
  \item \textbf{exception:} None.
\end{itemize}

\paragraph{\texttt{refactor(self, file\_path: Path, pylint\_smell: Smell, initial\_emissions: float)}}
\begin{itemize}
  \item \textbf{transition}: Parses \texttt{file\_path}, identifies repeated function calls, inserts a cached variable for the first call, updates subsequent calls to use the cached variable, and validates refactoring.
  \item \textbf{output}: None.
  \item \textbf{exception}: Raises \texttt{IOError} if input file cannot be read. Raises \texttt{TypeError} if source file cannot be parsed into an AST.
\end{itemize}

\subsubsection{Local Functions}
\begin{itemize}
  \item \texttt{\_get\_indentation(lines, line\_number)}: Determines the indentation of a specific line.
  \item \texttt{\_replace\_call\_in\_line(line, call\_string, cached\_var\_name)}: Replaces repeated calls with the cached variable.
  \item \texttt{\_find\_valid\_parent(tree)}: Identifies the valid parent node containing all occurrences of the repeated call.
  \item \texttt{\_find\_insert\_line(parent\_node)}: Determines the line to insert the cached variable.
\end{itemize}

\bibliographystyle {plainnat}
\bibliography {../../../refs/References}

\newpage

\section{Appendix} \label{Appendix}

\wss{Extra information if required}

\newpage{}

\section*{Appendix --- Reflection}

\wss{Not required for CAS 741 projects}

The information in this section will be used to evaluate the team members on the
graduate attribute of Problem Analysis and Design.

\input{../../Reflection.tex}

\begin{enumerate}
  \item What went well while writing this deliverable? 
  \item What pain points did you experience during this deliverable, and how
    did you resolve them?
  \item Which of your design decisions stemmed from speaking to your client(s)
  or a proxy (e.g. your peers, stakeholders, potential users)? For those that
  were not, why, and where did they come from?
  \item While creating the design doc, what parts of your other documents (e.g.
  requirements, hazard analysis, etc), it any, needed to be changed, and why?
  \item What are the limitations of your solution?  Put another way, given
  unlimited resources, what could you do to make the project better? (LO\_ProbSolutions)
  \item Give a brief overview of other design solutions you considered.  What
  are the benefits and tradeoffs of those other designs compared with the chosen
  design?  From all the potential options, why did you select the documented design?
  (LO\_Explores)
\end{enumerate}


\end{document}