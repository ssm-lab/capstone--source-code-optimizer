\documentclass[12pt, titlepage]{article}

\usepackage{amsmath, mathtools}

\usepackage[round]{natbib}
\usepackage{amsfonts}
\usepackage{amssymb}
\usepackage{graphicx}
\usepackage{colortbl}
\usepackage{xr}
\usepackage{hyperref}
\usepackage{longtable}
\usepackage{xfrac}
\usepackage{tabularx}
\usepackage{float}
\usepackage{siunitx}
\usepackage{booktabs}
\usepackage{multirow}
\usepackage[section]{placeins}
\usepackage{caption}
\usepackage{fullpage}

\hypersetup{
bookmarks=true,     % show bookmarks bar?
colorlinks=true,       % false: boxed links; true: colored links
linkcolor=red,          % color of internal links (change box color with linkbordercolor)
citecolor=blue,      % color of links to bibliography
filecolor=magenta,  % color of file links
urlcolor=cyan          % color of external links
}

\usepackage{array}

\externaldocument{../../SRS/SRS}

%% Comments

\usepackage{color}

% \newif\ifcomments\commentstrue %displays comments
\newif\ifcomments\commentsfalse %so that comments do not display

\ifcomments
\newcommand{\authornote}[3]{\textcolor{#1}{[#3 ---#2]}}
\newcommand{\todo}[1]{\textcolor{red}{[TODO: #1]}}
\else
\newcommand{\authornote}[3]{}
\newcommand{\todo}[1]{}
\fi

\newcommand{\wss}[1]{\authornote{blue}{SS}{#1}} 
\newcommand{\plt}[1]{\authornote{magenta}{TPLT}{#1}} %For explanation of the template
\newcommand{\an}[1]{\authornote{cyan}{Author}{#1}}

%% Common Parts

\newcommand{\progname}{Software Engineering} % PUT YOUR PROGRAM NAME HERE
\newcommand{\authname}{\textbf{Team 4, EcoOptimizers} \\
  \\ Nivetha Kuruparan
  \\ Sevhena Walker
  \\ Tanveer Brar
  \\ Mya Hussain
\\ Ayushi Amin} % AUTHOR NAMES

\usepackage{hyperref}
\hypersetup{colorlinks=true, linkcolor=blue, citecolor=blue, filecolor=blue,
urlcolor=blue, unicode=false}
\urlstyle{same}



\begin{document}

\title{Module Interface Specification for \progname{}}

\author{\authname}

\date{\today}

\maketitle

\pagenumbering{roman}

\section{Revision History}

\begin{tabularx}{\textwidth}{p{3cm}p{2cm}X}
\toprule {\bf Date} & {\bf Version} & {\bf Notes}\\
\midrule
Date 1 & 1.0 & Notes\\
Date 2 & 1.1 & Notes\\
\bottomrule
\end{tabularx}

~\newpage

\section{Symbols, Abbreviations and Acronyms}

See SRS Documentation at \wss{give url}

\wss{Also add any additional symbols, abbreviations or acronyms}

\newpage

\tableofcontents

\newpage

\pagenumbering{arabic}

\section{Introduction}

The following document details the Module Interface Specifications for
\wss{Fill in your project name and description}

Complementary documents include the System Requirement Specifications
and Module Guide.  The full documentation and implementation can be
found at \url{...}.  \wss{provide the url for your repo}

\section{Notation}

\wss{You should describe your notation.  You can use what is below as
  a starting point.}

The structure of the MIS for modules comes from \citet{HoffmanAndStrooper1995},
with the addition that template modules have been adapted from
\cite{GhezziEtAl2003}.  The mathematical notation comes from Chapter 3 of
\citet{HoffmanAndStrooper1995}.  For instance, the symbol := is used for a
multiple assignment statement and conditional rules follow the form $(c_1
\Rightarrow r_1 | c_2 \Rightarrow r_2 | ... | c_n \Rightarrow r_n )$.

The following table summarizes the primitive data types used by \progname. 

\begin{center}
\renewcommand{\arraystretch}{1.2}
\noindent 
\begin{tabular}{l l p{7.5cm}} 
\toprule 
\textbf{Data Type} & \textbf{Notation} & \textbf{Description}\\ 
\midrule
character & char & a single symbol or digit\\
integer & $\mathbb{Z}$ & a number without a fractional component in (-$\infty$, $\infty$) \\
natural number & $\mathbb{N}$ & a number without a fractional component in [1, $\infty$) \\
real & $\mathbb{R}$ & any number in (-$\infty$, $\infty$)\\
\bottomrule
\end{tabular} 
\end{center}

\noindent
The specification of \progname \ uses some derived data types: sequences, strings, and
tuples. Sequences are lists filled with elements of the same data type. Strings
are sequences of characters. Tuples contain a list of values, potentially of
different types. In addition, \progname \ uses functions, which
are defined by the data types of their inputs and outputs. Local functions are
described by giving their type signature followed by their specification.

\section{Module Decomposition}

The following table is taken directly from the Module Guide document for this project.

\begin{table}[h!]
\centering
\begin{tabular}{p{0.3\textwidth} p{0.6\textwidth}}
\toprule
\textbf{Level 1} & \textbf{Level 2}\\
\midrule

{Hardware-Hiding} & ~ \\
\midrule

\multirow{7}{0.3\textwidth}{Behaviour-Hiding} & Input Parameters\\
& Output Format\\
& Output Verification\\
& Temperature ODEs\\
& Energy Equations\\ 
& Control Module\\
& Specification Parameters Module\\
\midrule

\multirow{3}{0.3\textwidth}{Software Decision} & {Sequence Data Structure}\\
& ODE Solver\\
& Plotting\\
\bottomrule

\end{tabular}
\caption{Module Hierarchy}
\label{TblMH}
\end{table}

\newpage
~\newpage

\section{MIS of \wss{Module Name}} \label{Module} \wss{Use labels for
  cross-referencing}

\wss{You can reference SRS labels, such as R\ref{R_Inputs}.}

\wss{It is also possible to use \LaTeX for hypperlinks to external documents.}

\subsection{Module}

\wss{Short name for the module}

\subsection{Uses}


\subsection{Syntax}

\subsubsection{Exported Constants}

\subsubsection{Exported Access Programs}

\begin{center}
\begin{tabular}{p{2cm} p{4cm} p{4cm} p{2cm}}
\hline
\textbf{Name} & \textbf{In} & \textbf{Out} & \textbf{Exceptions} \\
\hline
\wss{accessProg} & - & - & - \\
\hline
\end{tabular}
\end{center}

\subsection{Semantics}

\subsubsection{State Variables}

\wss{Not all modules will have state variables.  State variables give the module
  a memory.}

\subsubsection{Environment Variables}

\wss{This section is not necessary for all modules.  Its purpose is to capture
  when the module has external interaction with the environment, such as for a
  device driver, screen interface, keyboard, file, etc.}

\subsubsection{Assumptions}

\wss{Try to minimize assumptions and anticipate programmer errors via
  exceptions, but for practical purposes assumptions are sometimes appropriate.}

\subsubsection{Access Routine Semantics}

\noindent \wss{accessProg}():
\begin{itemize}
\item transition: \wss{if appropriate} 
\item output: \wss{if appropriate} 
\item exception: \wss{if appropriate} 
\end{itemize}

\wss{A module without environment variables or state variables is unlikely to
  have a state transition.  In this case a state transition can only occur if
  the module is changing the state of another module.}

\wss{Modules rarely have both a transition and an output.  In most cases you
  will have one or the other.}

\subsubsection{Local Functions}

\wss{As appropriate} \wss{These functions are for the purpose of specification.
  They are not necessarily something that is going to be implemented
  explicitly.  Even if they are implemented, they are not exported; they only
  have local scope.}

\newpage

\section{MIS of Use A Generator Refactorer} \label{mis:UseGen}

\texttt{UseAGeneratorRefactorer}

\subsection{Module}

The \texttt{UseAGeneratorRefactorer} module identifies and refactors 
unnecessary list comprehensions in Python code by converting them to generator expressions. This refactoring improves energy efficiency while maintaining the original functionality.

\subsection{Uses}
\begin{itemize}
  \item Uses \texttt{Smell} interface for data access
  \item Inherits from \texttt{BaseRefactorer}
  \item Uses Python's \texttt{ast} module for parsing and manipulating abstract syntax trees
\end{itemize}

\subsection{Syntax}
\noindent
\textbf{Exported Constants}: None

\noindent
\textbf{Exported Access Programs}:\\
\begin{tabularx}{\linewidth}{|
    l|
    >{\raggedright\arraybackslash}X|
    l|
    l|}
  \toprule Name & In & Out & Exceptions \\
  \midrule
  \texttt{\_\_init\_\_} & \texttt{output\_dir: Path} & None & None \\
  \hline
  \texttt{refactor} & \texttt{file\_path: Path, pylint\_smell: Smell, initial\_emissions: float} & None & \texttt{IOError}, \texttt{TypeError} \\
  \hline
  \texttt{\_replace\_node} & \texttt{tree: ast.Module, old\_node: ast.ListComp, new\_node: ast.GeneratorExp} & None & None \\
  \bottomrule
\end{tabularx}

\subsection{Semantics}

\subsubsection*{State Variables}
\begin{itemize}
  \item \texttt{temp\_dir: Path}: Directory path for storing refactored files.
  \item \texttt{output\_dir: Path}: Directory path for saving final refactored code.
\end{itemize}

\subsubsection*{Environment Variables}
None

\subsubsection*{Assumptions}
\begin{itemize}
  \item The input file contains valid Python syntax.
  \item \texttt{pylint\_smell} provides a valid line number for the detected code smell.
\end{itemize}

\subsubsection*{Access Routine Semantics}

\paragraph{\texttt{\_\_init\_\_(self, output\_dir: Path)}}
\begin{itemize}
  \item \textbf{transition}: Initializes the \texttt{temp\_dir} variable within \texttt{output\_dir}.
  \item \textbf{output}: None.
  \item \textbf{exception:} None.
\end{itemize}

\paragraph{\texttt{refactor(self, file\_path: Path, pylint\_smell: Smell, initial\_emissions: float)}}
\begin{itemize}
  \item \textbf{transition}: Parses \texttt{file\_path}, identifies unnecessary list comprehensions, modifies the code to use generator expressions, and validates refactoring.
  \item \textbf{output}: None.
  \item \textbf{exception}: Raises \texttt{IOError} if input file cannot be read. Raises \texttt{TypeError} if source file cannot be parsed into an AST.
\end{itemize}

\paragraph{\texttt{\_replace\_node(self, tree: ast.Module, old\_node: ast.ListComp, new\_node: ast.GeneratorExp)}}
\begin{itemize}
  \item \textbf{transition}: Replaces an \texttt{old\_node} in the AST with a \texttt{new\_node}.
  \item \textbf{output}: None.
  \item \textbf{exception}: None.
\end{itemize}

\subsubsection*{Local Functions}
Functions for internal AST parsing, node manipulation, and validation are defined within the class but are not exported.

\newpage

\section{MIS of Cache Repeated Calls Refactorer} \label{mis:CacheCalls}

\texttt{CacheRepeatedCallsRefactorer}

\subsection{Module}

The \texttt{CacheRepeatedCallsRefactorer} module identifies repeated function calls in Python code and refactors them by caching the result of the first call to a temporary variable. This refactoring improves performance and energy efficiency while preserving the original functionality.

\subsection{Uses}
\begin{itemize}
  \item Uses \texttt{Smell} interface for data access
  \item Inherits from \texttt{BaseRefactorer}
  \item Uses Python's \texttt{ast} module for parsing and manipulating abstract syntax trees
\end{itemize}

\subsection{Syntax}
\noindent
\textbf{Exported Constants}: None

\noindent
\textbf{Exported Access Programs}:\\
\begin{tabularx}{\linewidth}{|l|>{\raggedright\arraybackslash}X|l|l|}
  \toprule Name & In & Out & Exceptions \\
  \midrule
  \texttt{\_\_init\_\_} & \texttt{output\_dir: Path} & None & None \\
  \hline
  \texttt{refactor} & \texttt{file\_path: Path, pylint\_smell: Smell, initial\_emissions: float} & None & \texttt{IOError}, \texttt{TypeError} \\
  \bottomrule
\end{tabularx}

\subsection{Semantics}

\subsubsection*{State Variables}
\begin{itemize}
  \item \texttt{cached\_var\_name: str}: Name of the temporary variable used for caching.
  \item \texttt{target\_line: int}: Line number where refactoring is applied.
\end{itemize}

\subsubsection*{Environment Variables}
None

\subsubsection*{Assumptions}
\begin{itemize}
  \item The input file contains valid Python syntax.
  \item \texttt{pylint\_smell} provides valid occurrences of repeated calls with line numbers and call strings.
\end{itemize}

\subsubsection*{Access Routine Semantics}

\paragraph{\texttt{\_\_init\_\_(self, output\_dir: Path)}}
\begin{itemize}
  \item \textbf{transition}: Initializes the \texttt{temp\_dir} variable within \texttt{output\_dir}.
  \item \textbf{output}: None.
  \item \textbf{exception:} None.
\end{itemize}

\paragraph{\texttt{refactor(self, file\_path: Path, pylint\_smell: Smell, initial\_emissions: float)}}
\begin{itemize}
  \item \textbf{transition}: Parses \texttt{file\_path}, identifies repeated function calls, inserts a cached variable for the first call, updates subsequent calls to use the cached variable, and validates refactoring.
  \item \textbf{output}: None.
  \item \textbf{exception}: Raises \texttt{IOError} if input file cannot be read. Raises \texttt{TypeError} if source file cannot be parsed into an AST.
\end{itemize}

\subsubsection*{Local Functions}
\begin{itemize}
  \item \texttt{\_get\_indentation(lines, line\_number)}: Determines the indentation of a specific line.
  \item \texttt{\_replace\_call\_in\_line(line, call\_string, cached\_var\_name)}: Replaces repeated calls with the cached variable.
  \item \texttt{\_find\_valid\_parent(tree)}: Identifies the valid parent node containing all occurrences of the repeated call.
  \item \texttt{\_find\_insert\_line(parent\_node)}: Determines the line to insert the cached variable.
\end{itemize}


\bibliographystyle {plainnat}
\bibliography {../../../refs/References}

\newpage

\section{Appendix} \label{Appendix}

\wss{Extra information if required}

\newpage{}

\section*{Appendix --- Reflection}

\wss{Not required for CAS 741 projects}

The information in this section will be used to evaluate the team members on the
graduate attribute of Problem Analysis and Design.

\input{../../Reflection.tex}

\begin{enumerate}
  \item What went well while writing this deliverable? 
  \item What pain points did you experience during this deliverable, and how
    did you resolve them?
  \item Which of your design decisions stemmed from speaking to your client(s)
  or a proxy (e.g. your peers, stakeholders, potential users)? For those that
  were not, why, and where did they come from?
  \item While creating the design doc, what parts of your other documents (e.g.
  requirements, hazard analysis, etc), it any, needed to be changed, and why?
  \item What are the limitations of your solution?  Put another way, given
  unlimited resources, what could you do to make the project better? (LO\_ProbSolutions)
  \item Give a brief overview of other design solutions you considered.  What
  are the benefits and tradeoffs of those other designs compared with the chosen
  design?  From all the potential options, why did you select the documented design?
  (LO\_Explores)
\end{enumerate}


\end{document}