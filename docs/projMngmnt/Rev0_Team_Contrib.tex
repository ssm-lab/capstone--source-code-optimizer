\documentclass{article}

\usepackage{float}
\restylefloat{table}

\usepackage{booktabs}

\title{Team Contributions: Rev 0\\\progname}

\author{\authname}

\date{}

%% Comments

\usepackage{color}

% \newif\ifcomments\commentstrue %displays comments
\newif\ifcomments\commentsfalse %so that comments do not display

\ifcomments
\newcommand{\authornote}[3]{\textcolor{#1}{[#3 ---#2]}}
\newcommand{\todo}[1]{\textcolor{red}{[TODO: #1]}}
\else
\newcommand{\authornote}[3]{}
\newcommand{\todo}[1]{}
\fi

\newcommand{\wss}[1]{\authornote{blue}{SS}{#1}} 
\newcommand{\plt}[1]{\authornote{magenta}{TPLT}{#1}} %For explanation of the template
\newcommand{\an}[1]{\authornote{cyan}{Author}{#1}}

%% Common Parts

\newcommand{\progname}{Software Engineering} % PUT YOUR PROGRAM NAME HERE
\newcommand{\authname}{\textbf{Team 4, EcoOptimizers} \\
  \\ Nivetha Kuruparan
  \\ Sevhena Walker
  \\ Tanveer Brar
  \\ Mya Hussain
\\ Ayushi Amin} % AUTHOR NAMES

\usepackage{hyperref}
\hypersetup{colorlinks=true, linkcolor=blue, citecolor=blue, filecolor=blue,
urlcolor=blue, unicode=false}
\urlstyle{same}



\begin{document}

\maketitle

This document summarizes the contributions of each team member for the Rev 0
Demo.  The time period of interest is the time between the POC demo and the Rev
0 demo.

\section{Demo Plans}

For our \textbf{Rev 0 demonstration}, we will showcase the core
functionality and usability of our \textbf{energy-efficient Python
code refactoring tool}. This demonstration will highlight the
following key aspects:

\begin{enumerate}

  \item \textbf{Code Smell Detection:} We will demonstrate how our
    tool identifies inefficient coding patterns (\textit{code
    smells}) in Python source code that contribute to higher energy consumption.
  \item \textbf{Automated Refactoring:} Our tool will apply targeted
    refactorings to optimize the detected code smells, improving
    energy efficiency.
  \item \textbf{Energy Consumption Measurement:} Using
    \texttt{CodeCarbon}, we will measure and compare the energy
    consumption of the original code versus the refactored version,
    providing clear insights into energy savings.
  \item \textbf{Functionality Preservation:} To ensure that
    optimizations do not alter program behaviour, we will run the
    original test suite against both the unoptimized and refactored versions.
  \item \textbf{VS Code Integration:} We will showcase our \textbf{VS
    Code plugin}, where users can analyze a Python file for code
    smells, choose specific optimizations, and preview the refactored
    code with energy measurements. Users will have the option to
    accept or reject the proposed changes.

\end{enumerate}

\section{Team Meeting Attendance}

Our team stays in touch regularly through Discord and schedules a
meeting when detailed discussion is needed. Below is the attendance
of all format meetings that have taken place.

\wss{For each team member how many team meetings have they attended over the
  time period of interest.  This number should be determined from the meeting
  issues in the team's repo.  The first entry in the table should be the total
number of team meetings held by the team.}

\begin{table}[H]
  \centering
  \begin{tabular}{ll}
    \toprule
    \textbf{Student} & \textbf{Meetings}\\
    \midrule
    Total & 3\\
    Sevhena Walker & 3\\
    Mya Hussain & 3\\
    Ayushi Amin & 3\\
    Nivetha Kuruparan & 3\\
    Tanveer Brar & 3\\
    \bottomrule
  \end{tabular}
\end{table}

\wss{If needed, an explanation for the counts can be provided here.}

\section{Supervisor/Stakeholder Meeting Attendance}

\wss{For each team member how many supervisor/stakeholder team meetings have
  they attended over the time period of interest.  This number should
  be determined
  from the supervisor meeting issues in the team's repo.  The first entry in the
  table should be the total number of supervisor and team meetings held by the
  team.  If there is no supervisor, there will usually be meetings with
stakeholders (potential users) that can serve a similar purpose.}

We prioritize having in person meetings with our supervisor Dr David
to be able to effectively communicate our progress and concerns. We
have a weekly slot open with the supervisor, but in case there is
little to discuss, the meeting is cancelled for the week and our
progress is communicated via Discord.

\begin{table}[H]
  \centering
  \begin{tabular}{ll}
    \toprule
    \textbf{Student} & \textbf{Meetings}\\
    \midrule
    Total & 4\\
    Sevhena Walker & 4\\
    Mya Hussain & 4\\
    Ayushi Amin & 4\\
    Nivetha Kuruparan & 4\\
    Tanveer Brar & 4\\
    \bottomrule
  \end{tabular}
\end{table}

\wss{If needed, an explanation for the counts can be provided here.}

\section{Lecture Attendance}

We aim to have at least one person attend the lecture so the team is
upto date with all information. In the rare case that no one goes to
the lecture, we update through the lecture slides.

\wss{For each team member how many lectures have they attended over the time
  period of interest.  This number should be determined from the
  lecture issues in
  the team's repo.  The first entry in the table should be the total number of
lectures since the beginning of the term.}

\begin{table}[H]
  \centering
  \begin{tabular}{ll}
    \toprule
    \textbf{Student} & \textbf{Lectures}\\
    \midrule
    Total & 2\\
    Sevhena Walker & 1\\
    Mya Hussain & 0\\
    Ayushi Amin & 0\\
    Nivetha Kuruparan & 0\\
    Tanveer Brar & 0\\
    \bottomrule
  \end{tabular}
\end{table}

\wss{If needed, an explanation for the lecture attendance can be provided here.}
\section{TA Document Discussion Attendance}

\wss{For each team member how many of the informal document discussion meetings
with the TA were attended over the time period of interest.}

\begin{table}[H]
  \centering
  \begin{tabular}{ll}
    \toprule
    \textbf{Student} & \textbf{Lectures}\\
    \midrule
    Total & 2\\
    Sevhena Walker & 2 \\
    Mya Hussain & 2 \\
    Ayushi Amin & 2 \\
    Nivetha Kuruparan & 2 \\
    Tanveer Brar & 2 \\
    \bottomrule
  \end{tabular}
\end{table}

\wss{If needed, an explanation for the attendance can be provided here.}

\section{Commits}

\wss{For each team member how many commits to the main branch have been made
  over the time period of interest.  The total is the total number of
  commits for
  the entire team since the beginning of the term.  The percentage is the
percentage of the total commits made by each team member.}

\begin{table}[H]
  \centering
  \begin{tabular}{lll}
    \toprule
    \textbf{Student} & \textbf{Commits} & \textbf{Percent}\\
    \midrule
    Total & 662 & 100\% \\
    Sevhena Walker & 277 & 41.8429\% \\
    Mya Hussain & 78 & 11.782485\% \\
    Ayushi Amin & 129 & 19.4864\% \\
    Nivetha Kuruparan & 102 & 15.407855\% \\
    Tanveer Brar & 76 & 11.48036\% \\
    \bottomrule
  \end{tabular}
\end{table}

The number of commits does not necessarily reflect the amount of work
done by each team member. Some members prefer to make frequent small
commits while others make fewer but larger commits that include
multiple changes.

\section{Issue Tracker}

\wss{For each team member how many issues have they authored (including open and
  closed issues (O+C)) and how many have they been assigned (only
    counting closed
issues (C only)) over the time period of interest.}

\begin{table}[H]
  \centering
  \begin{tabular}{lll}
    \toprule
    \textbf{Student} & \textbf{Authored (O+C)} & \textbf{Assigned (C only)}\\
    \midrule
    Tanveer Brar & 1 & 47 \\
    Mya Hussain & 8 & 51 \\
    Ayushi Amin & 8 & 58 \\
    Sevhena Walker & 81 & 72 \\
    Nivetha Kuruparan & 17 & 53 \\
    \bottomrule
  \end{tabular}
\end{table}

Counts are considered by taking the total counts over the year and
subtracting the counts at the time of the POC Team Contribution. This
is the period of intrest.
We did a lot of work over the winter break to exclude that period and
only do "this term" would inaccurately represent the counts.
\section{CICD}

\wss{Say how CICD is used in your project}

CI/CD was implemented in the project to automate essential
development and documentation tasks, improving efficiency and
maintaining code quality. The project utilises GitHub Actions
workflows to enforce coding standards, validate changes, and ensure
documentation integrity.

A dedicated workflow formats and compiles \LaTeX{} documents whenever
changes are made to \texttt{.tex} files or their dependencies. This ensures that
only high quality documentation is pushed to the main branch.

Another workflow is responsible for maintaining code consistency and
adhering to best practices. It uses \texttt{ruff} to lint and format
\texttt{.py} files, identifying potential issues such as syntax
errors, unused imports, and style inconsistencies. This workflow
allows the python scripts to be written with a uniform coding style.

The final workflow ensures the correctness of the codebase. It runs
\texttt{pytest} to execute unit and integration tests, validating the
expected behaviour of the software. Additionally, it includes a code
coverage check, ensuring that newly introduced or modified code is
adequately tested. This helps prevent regressions and maintain the
reliability of the project.

\end{document}
