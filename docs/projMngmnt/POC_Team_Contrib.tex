\documentclass{article}

\usepackage{float}
\restylefloat{table}

\usepackage{booktabs}

\title{Team Contributions: POC\\\progname}

\author{\authname}

\date{}

\input{../Comments}
%% Common Parts

\newcommand{\progname}{Software Engineering} % PUT YOUR PROGRAM NAME HERE
\newcommand{\authname}{\textbf{Team 6, EcoOptimizers} \\
  \\ Nivetha Kuruparan
  \\ Sevhena Walker
  \\ Tanveer Brar
  \\ Mya Hussain
\\ Ayushi Amin} % AUTHOR NAMES

\usepackage{hyperref}
\hypersetup{colorlinks=true, linkcolor=blue, citecolor=blue, filecolor=blue,
urlcolor=blue, unicode=false}
\urlstyle{same}



\begin{document}

\maketitle

This document summarizes the contributions of each team member up to the POC
Demo.  The time period of interest is the time between the beginning of the term
and the POC demo.

\section{Demo Plans}

For our proof of concept demonstration, we will showcase the core functionality of our energy-efficient 
Python code refactoring tool. The demonstration will focus on the following key aspects:
\begin{enumerate}

    \item \textbf{Code Smell Detection:} We will show how we used Pylint to identify inefficient code 
    patterns (code smells) in Python source code that may lead to higher energy consumption.
    \item \textbf{Refactoring:} Using the Rope library, we'll demonstrate how our tool will 
    apply refactorings to address the detected code smells.
    \item \textbf{Energy Consumption Measurement:} We will show how we utilized CodeCarbon to measure 
    and compare the energy consumption of the original code versus the refactored version.
    \item \textbf{Functionality Preservation:} We will demonstrate that the refactored code 
    maintains its original functionality by running the original test suite against both versions of the code.
    \item \textbf{Performance Metrics:} We will display performance reports comparing the original and refactored 
    code, highlighting improvements in energy efficiency.
    
\end{enumerate}

\section{Team Meeting Attendance}

\wss{For each team member how many team meetings have they attended over the
time period of interest.  This number should be determined from the meeting
issues in the team's repo.  The first entry in the table should be the total
number of team meetings held by the team.}

\begin{table}[H]
\centering
\begin{tabular}{ll}
\toprule
\textbf{Student} & \textbf{Meetings}\\
\midrule
Total & Num\\
Name 1 & Num\\
Name 2 & Num\\
Name 3 & Num\\
Name 4 & Num\\
Name 5 & Num\\
\bottomrule
\end{tabular}
\end{table}

\wss{If needed, an explanation for the counts can be provided here.}

\section{Supervisor/Stakeholder Meeting Attendance}

\wss{For each team member how many supervisor/stakeholder team meetings have
they attended over the time period of interest.  This number should be determined
from the supervisor meeting issues in the team's repo.  The first entry in the
table should be the total number of supervisor and team meetings held by the
team.  If there is no supervisor, there will usually be meetings with
stakeholders (potential users) that can serve a similar purpose.}

\begin{table}[H]
\centering
\begin{tabular}{ll}
\toprule
\textbf{Student} & \textbf{Meetings}\\
\midrule
Total & Num\\
Name 1 & Num\\
Name 2 & Num\\
Name 3 & Num\\
Name 4 & Num\\
Name 5 & Num\\
\bottomrule
\end{tabular}
\end{table}

\wss{If needed, an explanation for the counts can be provided here.}

\section{Lecture Attendance}

\wss{For each team member how many lectures have they attended over the time
period of interest.  This number should be determined from the lecture issues in
the team's repo.  The first entry in the table should be the total number of
lectures since the beginning of the term.}

\begin{table}[H]
\centering
\begin{tabular}{ll}
\toprule
\textbf{Student} & \textbf{Lectures}\\
\midrule
Total & Num\\
Name 1 & Num\\
Name 2 & Num\\
Name 3 & Num\\
Name 4 & Num\\
Name 5 & Num\\
\bottomrule
\end{tabular}
\end{table}

\wss{If needed, an explanation for the lecture attendance can be provided here.}

\section{TA Document Discussion Attendance}

\wss{For each team member how many of the informal document discussion meetings
with the TA were attended over the time period of interest.}

\begin{table}[H]
\centering
\begin{tabular}{ll}
\toprule
\textbf{Student} & \textbf{Lectures}\\
\midrule
Total & Num\\
Name 1 & Num\\
Name 2 & Num\\
Name 3 & Num\\
Name 4 & Num\\
Name 5 & Num\\
\bottomrule
\end{tabular}
\end{table}

\wss{If needed, an explanation for the attendance can be provided here.}

\section{Commits}

\wss{For each team member how many commits to the main branch have been made
over the time period of interest.  The total is the total number of commits for
the entire team since the beginning of the term.  The percentage is the
percentage of the total commits made by each team member.}

\begin{table}[H]
\centering
\begin{tabular}{lll}
\toprule
\textbf{Student} & \textbf{Commits} & \textbf{Percent}\\
\midrule
Total & 396 & 100\% \\
Ayushi Amin & 89 & 23\% \\
Tanveer Brar & 44 & 11\% \\
Mya Hussain & 59 & 15\% \\
Sevhena Walker & 156 & 39\% \\
Nivetha Kuruparan & 48 & 12\% \\
\bottomrule
\end{tabular}
\end{table}

Some people might have higher commit counts because they commit more frequently or make smaller, more granular changes, whereas others might commit less often with larger, consolidated changes. Additionally some group members squash and merge when merging PRs and others forget sometimes.

\wss{If needed, an explanation for the counts can be provided here.  For
instance, if a team member has more commits to unmerged branches, these numbers
can be provided here.  If multiple people contribute to a commit, git allows for
multi-author commits.}

\section{Issue Tracker}

\wss{For each team member how many issues have they authored (including open and
closed issues (O+C)) and how many have they been assigned (only counting closed
issues (C only)) over the time period of interest.}

\begin{table}[H]
\centering
\begin{tabular}{lll}
\toprule
\textbf{Student} & \textbf{Authored (O+C)} & \textbf{Assigned (C only)}\\
\midrule
Sevhena Walker & 25 & 24 \\
Mya Hussain & 10 & 19 \\
Nivetha Kurparan & 18 & 23 \\
Tanveer Brar & 9 & 20 \\
Ayushi Amin & 12 & 21 \\
\bottomrule
\end{tabular}
\end{table}

The numbers in the \textbf{Assigned} column give a better picture of each team members contribution. Many commits were sometimes authored by the same person due to differences in team responsibilities (logistics and management). Furthermore, the issues here refer to what we can call ``work'' issues. Issues with the \texttt{lecture}, \texttt{team-meeting}, \texttt{sup-meeting}, and \texttt{ta-meeting} labels are not included in this tally.

\section{CICD}

The section outlines the plan to include CI/CD for this project. The plan will streamline development, testing and deployment processes, while ensuring consistent performance improvements.

\subsection{Source Control and Branching Strategy}
\begin{itemize}
    \item \textbf{Repository Setup}: Code is hosted on GitHub for version control and collaboration.
    \item \textbf{Branching Strategy}:
    \begin{itemize}
        \item \textbf{main}: Production-ready code.
        \item \textbf{dev}: Primary development branch.
        \item \textbf{docs}: Primary documentation branch.
    \end{itemize}
    Based on deliverables, temporary branches are created on team and individual level and discarded once merged into one of the above branches.
    \item \textbf{Merging Policy}: All pull requests should have at least two reviews before merging, as outlined in the Development Plan.
\end{itemize}

\subsection{Build and Testing Pipeline}
GitHub Actions will be used for CI/CD to automate testing and code analysis on pull requests. They will include the following:
\begin{itemize}
    \item \textbf{Build Steps}
    \begin{itemize}
        \item \textbf{Static Code Analysis \& Linting}: \texttt{PyLint} will be used to handle both code smells for static analysis and enforce PEP 8 style guide.
    \end{itemize}
    \item \textbf{Testing}:
    \begin{itemize}
        \item \textbf{Unit Tests}: Unit tests will be written using \texttt{PyTest}.
        \item \textbf{Code Coverage}: Test code coverage will be tracked using \texttt{coverage.py}.
        \item \textbf{Performance Testing}: Metrics such as memory usage and execution time will be tracked using \texttt{cProfile}.
    \end{itemize}
\end{itemize}

\subsection{Continuous Deployment}
With every stable version, the product will need to be continuously deployed.
\begin{itemize}
    \item \textbf{Environment Setup}: To standardize environment settings across platforms, Docker containers will be used.
    \item \textbf{Deployment}:
    \begin{itemize}
        \item \textbf{Refactoring Library}: The library will be rebuilt and updated on its public facing source.
        \item \textbf{VS Code Extension}: With each update to main branch, the VS Code extension will automatically be built and updated on its public facing link.
    \end{itemize}
\end{itemize}

\section{Additional Productivity Metrics}
The team does not have any additional metrics of productivity.
\end{document}