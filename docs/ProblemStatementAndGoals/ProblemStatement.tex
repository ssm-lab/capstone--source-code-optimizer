\documentclass{article}

\usepackage{tabularx}
\usepackage{booktabs}

\title{Problem Statement and Goals\\\progname}

\author{\authname}

\date{}

\input{../Comments}
%% Common Parts

\newcommand{\progname}{Software Engineering} % PUT YOUR PROGRAM NAME HERE
\newcommand{\authname}{\textbf{Team 6, EcoOptimizers} \\
  \\ Nivetha Kuruparan
  \\ Sevhena Walker
  \\ Tanveer Brar
  \\ Mya Hussain
\\ Ayushi Amin} % AUTHOR NAMES

\usepackage{hyperref}
\hypersetup{colorlinks=true, linkcolor=blue, citecolor=blue, filecolor=blue,
urlcolor=blue, unicode=false}
\urlstyle{same}



\begin{document}

\maketitle

\begin{table}[hp]
  \caption{Revision History} \label{TblRevisionHistory}
  \begin{tabularx}{\textwidth}{llX}
    \toprule
    \textbf{Date} & \textbf{Developer(s)} & \textbf{Change}\\
    \midrule
    Date1 & Name(s) & Description of changes\\
    Date2 & Name(s) & Description of changes\\
    ... & ... & ...\\
    \bottomrule
  \end{tabularx}
\end{table}

\section{Problem Statement}

\wss{You should check your problem statement with the
  \href{https://github.com/smiths/capTemplate/blob/main/docs/Checklists/ProbState-Checklist.pdf}
{problem statement checklist}.}

\wss{You can change the section headings, as long as you include the required
information.}

\subsection{Problem}

\subsection{Inputs and Outputs}

\wss{Characterize the problem in terms of ``high level'' inputs and outputs.
Use abstraction so that you can avoid details.}

\textbf{Inputs:} Source code that requires refactoring for energy efficiency.
\textbf{Outputs:} Refactored code with reduced energy consumption, along with performance and energy consumption reports.

\subsection{Stakeholders}
\subsubsection*{\color{blue}{Direct Stakeholders}}
\textbf{[Software Developers]}: They will be the primary users of the refactoring library as they will be the ones to integrate the library into their code for better refactoring.\\

\noindent
\textbf{[Business Sustainability Teams]}: These teams are responsible for considering how a companies practices affect the environment. They will especially be interested in viewing the metrics provided by the library on how it improves the energy efficiency of software over time, therefore decreasing the burden on hardware and minimizing the company's environmental footprint.

\subsubsection*{\color{blue}{Indirect Stakeholders}}
\textbf{[Business Leaders]}: They focus on reducing operational costs associated with energy consumption, especially in large-scale or cloud-hosted applications. Use of the library in their products allows them to better achieve those goals. \\

\noindent
\textbf{[End Users]}: While not directly affected by this refactoring library, end users of technology that use the library will benefit from more responsive and efficient software that consumes less power, especially in mobile, embedded, or battery-dependent applications. \\

\noindent
\textbf{[Regulatory Bodies]}: They enforce energy consumption and sustainability standards, and ensure that software adheres to environmental regulations and may certify tools that meet efficiency requirements. Their oversight promotes the adoption of energy-efficient software practices.

\subsection{Environment}
\textbf{Programming Language:} Python will be the main language for developing refactoring tools and performing energy optimization.
\textbf{Reinforcement Learning Library: } Stable Baselines will be the library to implement reinforcement learning techniques.
\textbf{Developement Frameworks and Tools: } 
\begin{enumerate}
    \item Github will be used for version control and for CI/CD integration to automate refactoring processes.
    \item Visual Studio Code will be the IDE used.
\end{enumerate} 
\textbf{Database: } MySQL will be the database used to store and retrieve data about refactoring and energy consumption metrics.

\wss{Hardware and software environment}

\section{Goals}

\section{Stretch Goals}

\section{Challenge Level and Extras}

The expected challenge level of our project as general. This is due to the 
relatively straightforward technical knowledge required for its completion. 
The project primarily involves applying known software optimization and 
refactoring techniques, which are well-documented and accessible. 
Additionally, the required programming knowledge is in Python which known 
by all of the team members and was, taught in our undergraduate courses. 
Although the project does involve substantial development and research components, 
we anticipate the overall scope and depth of work to be manageable within the 
given timeframe.

To further enhance the project and address any potential gaps in the challenge level, 
we propose two additional activities: User Documentation and Usability Testing. 
These extras will allow us to provide support for future users of the tool and 
ensure that the tool meets user expectations and accessibility standards. 

Approval of the challenge level and extras will be discussed with the instructor, 
and adjustments may be made as needed throughout the term.

\newpage{}

\section*{Appendix --- Reflection}

\wss{Not required for CAS 741}

\input{../Reflection.tex}

\subsubsection*{Ayushi Amin Reflection}

\begin{enumerate}
    \item \textit{What went well while writing this deliverable?}
    
    Writing the deliverable went quite well overall. The project itself was well defined and we were able to meet with the 
    industry supervisor which helped clarify most of the details. Our team worked well together which helped each memeber to clarify
    any concerns particularly about the database we decided to use. Our team had some good discussions regarding making decisions on 
    what languages and tools to use in the project such as our decision to use MySQL for our database.

    \item \textit{What pain points did you experience during this deliverable, and how did you resolve them?}
    
    Not all the information was clarified or given by the supervisor at an early stage so writing this document took some 
    time which caused us to delay working on this for a couple days. The team and I decided to consistently check in with the supervisor
    to encourage him to provide the required documents and information we needed as soon as possible. We used email as well as discord to 
    directly reach out to the supervisor in which we finally ended up getting the information 2 days before the deadline and then crammed to 
    finish the milstone documents.
    
\end{enumerate}  

\subsubsection*{Full Name Reflection}
\begin{enumerate}
  \item \textit{What went well while writing this deliverable?}
  \item \textit{What pain points did you experience during this deliverable, and how did you resolve them?}
  \item \textit{How did you and your team adjust the scope of your goals to ensure they are suitable for a Capstone project (not overly 
  ambitious but also of appropriate complexity for a senior design project)?}
\end{enumerate}

\subsubsection*{Group Answers Reflection}
\begin{enumerate}
  \item \textit{How did you and your team adjust the scope of your goals to ensure they are suitable for a Capstone project (not overly 
  ambitious but also of appropriate complexity for a senior design project)?}
  
  We carefully adjusted our project scope by focusing on the most important goals and prioritizing features that mattered most, while also 
  setting clear milestones to keep us on track. The feedback we got from our supervisor helped us refine our goals, making sure we were ambitious but
  also realistic about what we could achieve.
    
\end{enumerate}
    
\end{document}
