\documentclass{article}

\usepackage{tabularx}
\usepackage{booktabs}

\title{Reflection and Traceability Report on EcoOptimizer}

\author{Ayushi Amin \\ Tanveer Brar \\ Nivetha Kuruparan\\ Sevhena Walker\\ Mya Hussain}

\date{}

\input{../Comments}
%% Common Parts

\newcommand{\progname}{Software Engineering} % PUT YOUR PROGRAM NAME HERE
\newcommand{\authname}{\textbf{Team 6, EcoOptimizers} \\
  \\ Nivetha Kuruparan
  \\ Sevhena Walker
  \\ Tanveer Brar
  \\ Mya Hussain
\\ Ayushi Amin} % AUTHOR NAMES

\usepackage{hyperref}
\hypersetup{colorlinks=true, linkcolor=blue, citecolor=blue, filecolor=blue,
urlcolor=blue, unicode=false}
\urlstyle{same}



\begin{document}

\maketitle

\plt{Reflection is an important component of getting the full benefits from a
learning experience.  Besides the intrinsic benefits of reflection, this
document will be used to help the TAs grade how well your team responded to
feedback.  Therefore, traceability between Revision 0 and Revision 1 is and
important part of the reflection exercise.  In addition, several CEAB (Canadian
Engineering Accreditation Board) Learning Outcomes (LOs) will be assessed based
on your reflections.}

\section{Changes in Response to Feedback}

\plt{Summarize the changes made over the course of the project in response to
feedback from TAs, the instructor, teammates, other teams, the project
supervisor (if present), and from user testers.}

\plt{For those teams with an external supervisor, please highlight how the feedback 
from the supervisor shaped your project.  In particular, you should highlight the 
supervisor's response to your Rev 0 demonstration to them.}

\plt{Version control can make the summary relatively easy, if you used issues
and meaningful commits.  If you feedback is in an issue, and you responded in
the issue tracker, you can point to the issue as part of explaining your
changes.  If addressing the issue required changes to code or documentation, you
can point to the specific commit that made the changes.  Although the links are
helpful for the details, you should include a label for each item of feedback so
that the reader has an idea of what each item is about without the need to click
on everything to find out.}

\plt{If you were not organized with your commits, traceability between feedback
and commits will not be feasible to capture after the fact.  You will instead
need to spend time writing down a summary of the changes made in response to
each item of feedback.}

\plt{You should address EVERY item of feedback.  A table or itemized list is
recommended.  You should record every item of feedback, along with the source of
that feedback and the change you made in response to that feedback.  The
response can be a change to your documentation, code, or development process.
The response can also be the reason why no changes were made in response to the
feedback.  To make this information manageable, you will record the feedback and
response separately for each deliverable in the sections that follow.}

\plt{If the feedback is general or incomplete, the TA (or instructor) will not
be able to grade your response to feedback.  In that case your grade on this
document, and likely the Revision 1 versions of the other documents will be
low.} 

\subsection{SRS and Hazard Analysis}

\subsection{Design and Design Documentation}

\subsection{VnV Plan and Report}

\section{Challenge Level and Extras}

\subsection{Challenge Level}

\plt{State the challenge level (advanced, general, basic) for your project.  Your challenge level should exactly match what is included in your problem statement.  This should be the challenge level agreed on between you and the course instructor.}

\subsection{Extras}

\plt{Summarize the extras (if any) that were tackled by this project.  Extras
can include usability testing, code walkthroughs, user documentation, formal
proof, GenderMag personas, Design Thinking, etc.  Extras should have already
been approved by the course instructor as included in your problem statement.}

\section{Design Iteration (LO11 (PrototypeIterate))}

\plt{Explain how you arrived at your final design and implementation.  How did
the design evolve from the first version to the final version?} 

\plt{Don't just say what you changed, say why you changed it.  The needs of the
client should be part of the explanation.  For example, if you made changes in
response to usability testing, explain what the testing found and what changes
it led to.}

\section{Design Decisions (LO12)}

\plt{Reflect and justify your design decisions.  How did limitations,
 assumptions, and constraints influence your decisions?  Discuss each of these
 separately.}

\section{Economic Considerations (LO23)}

\plt{Is there a market for your product? What would be involved in marketing your 
product? What is your estimate of the cost to produce a version that you could 
sell?  What would you charge for your product?  How many units would you have to 
sell to make money? If your product isn't something that would be sold, like an 
open source project, how would you go about attracting users?  How many potential 
users currently exist?}

\section{Reflection on Project Management (LO24)}

\plt{This question focuses on processes and tools used for project management.}

\subsection{How Does Your Project Management Compare to Your Development Plan}

\plt{Did you follow your Development plan, with respect to the team meeting plan, 
team communication plan, team member roles and workflow plan.  Did you use the 
technology you planned on using?}

We largely followed our development plan regarding team meetings, communication, 
member roles, and workflow. We adhered to our planned meeting schedule, ensuring regular 
discussions to track progress, resolve blockers, and align our work with project goals. 
To keep everything organized, all types of meetings were documented as issues on GitHub. 
This allowed us to track progress effectively, link discussions to specific tasks, and 
maintain meeting notes for future reference.

Our communication plan also worked well—whether through scheduled check-ins or ad-hoc 
discussions, we maintained a steady flow of information via our chosen platforms. Each 
team member upheld their assigned roles, ensuring a balanced distribution of tasks. Our 
workflow remained structured, with clear milestones and responsibilities that kept the 
project on track. While there were some natural adjustments along the way to optimize 
efficiency, we remained aligned with the overall plan.

Regarding technology, we successfully used the tools and frameworks outlined in our 
development plan. Our refactoring library was developed in Python, leveraging Rope for 
refactoring, PyLint for inefficient code pattern detection, and Code Carbon for energy 
analysis. For code quality, we enforced PEP 8 standards and used Ruff for linting and 
Pyright for static type checking. Additionally, we incorporated PySmells for detecting 
code smells.

To ensure robust testing, we wrote unit tests using pytest, integrated with coverage.py 
to measure code coverage. We also implemented performance benchmarking using Python’s 
built-in benchmarking tools to measure execution time across different file sizes.

For version control and CI/CD, we relied on GitHub and GitHub Actions, streamlining 
our development process through automated testing and integration. Our VS Code plugin 
was built using TypeScript, aligning with the VS Code architecture.

Overall, we effectively used the planned technologies and tools, making only necessary 
refinements to optimize development.

\subsection{What Went Well?}

\plt{What went well for your project management in terms of processes and 
technology?}

One of the biggest strengths was how we stayed organized. We used GitHub Issues to track 
tasks, discussions, and even meeting notes, which made it easy to monitor progress and 
ensure accountability. This approach helped keep everything transparent and well-documented.

Communication was another strong point. We stuck to our planned check-ins but also had 
the flexibility to reach out whenever needed. This balance kept things moving without 
feeling rigid. We also made sure to follow PEP 8 coding standards and used linters, 
which kept our code consistent and easy to read.

The tools we used made a big difference in keeping things smooth. GitHub Actions handled 
our automated testing and integration, so we didn’t have to worry about manually running 
tests every time. PyTest and Coverage.py helped us stay on top of testing, while Ruff and 
PyLint made sure our code was clean and error-free.

For the actual project, Code Carbon gave us energy consumption insights (even if it 
wasn’t always perfectly accurate), and Rope made refactoring much easier, saving us a lot of time.

Overall, the combination of good teamwork, clear processes, and the right tools kept us 
organized and made development a lot more efficient.

\subsection{What Went Wrong?}

\plt{What went wrong in terms of processes and technology?}

One of the main challenges we faced in the project was moving away from using 
reinforcement learning. Initially, it seemed like a great way to optimize energy 
consumption, but as we decided not to use it, the direction of the project had to 
shift. This required us to adjust our approach, which took time and created some 
uncertainty within the team. Keeping everyone aligned and maintaining clear 
communication was essential during this transition, especially as we explored 
alternative methods to achieve our goals.

On the technology side, developing the refactoring library itself turned out to 
be more complicated than we initially anticipated. Creating a tool that could not 
only identify energy-saving opportunities but also refactor the code while preserving 
its intent was a delicate balance. Working with Python's unique performance and 
dynamic features added another layer of complexity. Despite these challenges, they 
provided valuable learning experiences that have shaped how we approach the project now.

\subsection{What Would you Do Differently Next Time?}

\plt{What will you do differently for your next project?}

For our next project, one thing we’d do differently is spend more time upfront 
validating the technical approach before diving into development. With the shift 
away from reinforcement learning, we had to make adjustments mid-way through, which 
slowed down progress. In the future, we’d prioritize a more thorough exploration of 
different technologies and approaches early on to avoid these kinds of pivots. We’d 
also make sure to keep a closer eye on scope to prevent unnecessary shifts that could 
affect the timeline.

Additionally, we’d focus on improving team coordination from the start, ensuring that 
everyone is aligned not just on the technical goals but also on the project’s overall 
direction. Regular, structured check-ins would be helpful to track progress and address 
roadblocks quickly. On the technical side, we would invest more time in setting up a robust
DevOps pipeline earlier in the process, ensuring that continuous integration and testing are 
seamless from the beginning. Overall, we believe these changes would help streamline both the 
technical and team dynamics for a smoother project experience.

\section{Reflection on Capstone}

\plt{This question focuses on what you learned during the course of the capstone project.}

\subsection{Which Courses Were Relevant}

\plt{Which of the courses you have taken were relevant for the capstone project?}
Several courses we’ve taken were highly relevant to our capstone project:

\begin{itemize}
    \item \textbf{Software Architecture}: This course gave us a solid foundation in designing 
    scalable, maintainable, and efficient software systems, which was essential when building 
    the refactoring library and ensuring the overall structure of our tool was optimal.
    
    \item \textbf{Data Structures and Algorithms}: The principles from this course helped us 
    design more efficient algorithms for analyzing and refactoring source code to optimize 
    energy consumption. Understanding how to work with data structures effectively was key 
    in handling different code patterns and optimization techniques.
    
    \item \textbf{Database Systems}: Although we didn’t directly deal with complex databases 
    in the capstone, understanding how to efficiently manage and query large datasets was 
    useful when we considered tracking energy usage metrics or managing configurations within 
    the system.
    
    \item \textbf{Real-Time Systems and Control Applications}: This course provided insights 
    into managing time-sensitive tasks and the real-time performance of systems, which was 
    helpful when considering the energy optimization of software execution, particularly in 
    the context of how different coding choices could impact performance in real-time.

    \item \textbf{Intro to Software Development}: This course was crucial for learning best 
    practices in documentation and understanding design patterns, which helped us structure 
    our code and communicate our approach effectively throughout the project.
    
    \item \textbf{Object-Oriented Programming}: The concepts from this course were directly 
    applied when designing our system, especially when managing the relationships between the 
    different components of our refactoring library.
    
    \item \textbf{Human Computer Interfaces}: This course helped us a lot with usability testing 
    and designing the frontend of our tool in VS Code, ensuring that it was user-friendly and 
    intuitive for anyone using the software.
    
    \item \textbf{Software Engineering Practice and Experience: Binding Theory to Practice}: This 
    course was instrumental in teaching us how to approach open-ended design problems and apply 
    both theoretical and practical knowledge to real-world scenarios. It was particularly helpful 
    in guiding us through the experiential approach to solving computational problems, especially 
    when considering embedded systems and assembly programming.
    
\end{itemize}

These courses equipped us with the necessary technical background to approach the project’s 
challenges, from system design to algorithm optimization, and they directly informed the decisions 
we made while building the energy optimization tool.

\subsection{Knowledge/Skills Outside of Courses}

\plt{What skills/knowledge did you need to acquire for your capstone project
that was outside of the courses you took?}

For our capstone project, we had to acquire several skills and knowledge areas that were not directly 
covered in our coursework:

\begin{itemize}
    \item \textbf{Energy-Efficient Software Development}: We had to research and understand techniques 
    for reducing energy consumption in software, including best practices for writing energy-efficient 
    code and tools for measuring energy usage.
    
    \item \textbf{Static Code Analysis}: Since our project involved analyzing and refactoring code, 
    we had to learn about static analysis techniques and how to extract meaningful insights from 
    source code without executing it.
    
    \item \textbf{Refactoring Strategies for Energy Optimization}: While we had learned about code 
    refactoring in some courses, optimizing for energy efficiency was a new challenge. We had to 
    explore strategies that improve performance while minimizing power consumption.
    
    \item \textbf{GitHub DevOps and CI/CD Pipelines}: Although we had experience with version control,
    setting up automated testing and continuous integration workflows in GitHub required additional 
    learning.
    
    \item \textbf{VS Code Extension Development}: Since our tool was designed to work within VS Code, 
    we had to learn how to develop and integrate extensions, which was not covered in our coursework.
    
    \item \textbf{User Research and Usability Testing}: While our Human-Computer Interfaces course 
    covered usability principles, we had to go deeper into conducting user research, gathering 
    feedback, and refining our tool's user experience.
    
    \item \textbf{Performance Profiling and Benchmarking}: Measuring the energy impact of different 
    coding techniques required us to explore profiling tools and benchmarking methods to ensure our 
    refactorings were actually improving efficiency.
    
    \item \textbf{Advanced Python Optimization Techniques}: Since our refactoring library is 
    Python-based, we needed to learn about Python-specific optimizations, including memory management, 
    just-in-time compilation techniques, and efficient data structures.
\end{itemize}

These additional skills allowed us to successfully design and implement our energy optimization tool, 
bridging the gap between our academic knowledge and the real-world challenges of software efficiency. 

\end{document}