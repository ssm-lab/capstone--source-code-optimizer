% THIS DOCUMENT IS FOLLOWS THE VOLERE TEMPLATE BY Suzanne Robertson and James Robertson
% ONLY THE SECTION HEADINGS ARE PROVIDED
%
% Initial draft from https://github.com/Dieblich/volere
%
% Risks are removed because they are covered by the Hazard Analysis
\documentclass[12pt]{article}

\usepackage{booktabs}
\usepackage{tabularx}
\usepackage{hyperref}
\hypersetup{
    bookmarks=true,         % show bookmarks bar?
      colorlinks=true,      % false: boxed links; true: colored links
    linkcolor=red,          % color of internal links (change box color with linkbordercolor)
    citecolor=green,        % color of links to bibliography
    filecolor=magenta,      % color of file links
    urlcolor=cyan           % color of external links
}

\newcommand{\lips}{\textit{Insert your content here.}}

\input{../Comments}
%% Common Parts

\newcommand{\progname}{Software Engineering} % PUT YOUR PROGRAM NAME HERE
\newcommand{\authname}{\textbf{Team 6, EcoOptimizers} \\
  \\ Nivetha Kuruparan
  \\ Sevhena Walker
  \\ Tanveer Brar
  \\ Mya Hussain
\\ Ayushi Amin} % AUTHOR NAMES

\usepackage{hyperref}
\hypersetup{colorlinks=true, linkcolor=blue, citecolor=blue, filecolor=blue,
urlcolor=blue, unicode=false}
\urlstyle{same}



\begin{document}

\title{Software Requirements Specification for \progname: subtitle describing software} 
\author{\authname}
\date{\today}
	
\maketitle

~\newpage

\pagenumbering{roman}

\tableofcontents

~\newpage

\section*{Revision History}

\begin{tabularx}{\textwidth}{p{3cm}p{2cm}X}
\toprule {\textbf{Date}} & {\textbf{Version}} & {\textbf{Notes}}\\
\midrule
Date 1 & 1.0 & Notes\\
Date 2 & 1.1 & Notes\\
\bottomrule
\end{tabularx}

~\\

~\newpage
\section{Purpose of the Project}
\subsection{User Business}
\lips
\subsection{Goals of the Project}
\lips
\section{Stakeholders}
\subsection{Client}
\lips
\subsection{Customer}
\lips
\subsection{Other Stakeholders}
\lips
\subsection{Hands-On Users of the Project}
\lips
\subsection{Personas}
\lips
\subsection{Priorities Assigned to Users}
\lips
\subsection{User Participation}
\lips
\subsection{Maintenance Users and Service Technicians}
\lips

\section{Mandated Constraints}
\subsection{Solution Constraints}
\lips
\subsection{Implementation Environment of the Current System}
\lips
\subsection{Partner or Collaborative Applications}
\lips
\subsection{Off-the-Shelf Software}
\lips
\subsection{Anticipated Workplace Environment}
\lips
\subsection{Schedule Constraints}
\lips
\subsection{Budget Constraints}
\lips
\subsection{Enterprise Constraints}
\lips

\section{Naming Conventions and Terminology}
\subsection{Glossary of All Terms, Including Acronyms, Used by Stakeholders
involved in the Project}
\lips

\section{Relevant Facts And Assumptions}
\subsection{Relevant Facts}
\lips
\subsection{Business Rules}
\lips
\subsection{Assumptions}
\lips

\section{The Scope of the Work}
\subsection{The Current Situation}
\lips
\subsection{The Context of the Work}
\lips
\subsection{Work Partitioning}
\lips
\subsection{Specifying a Business Use Case (BUC)}
\lips

\section{Business Data Model and Data Dictionary}
\subsection{Business Data Model}
\lips
\subsection{Data Dictionary}
\lips

\section{The Scope of the Product}
\subsection{Product Boundary}
\lips
\subsection{Product Use Case Table}
\lips
\subsection{Individual Product Use Cases (PUC's)}
\lips

\section{Functional Requirements}
\subsection{Functional Requirements}
\lips

\section{Look and Feel Requirements}
\subsection{Appearance Requirements}
\lips
\subsection{Style Requirements}
\lips

\section{Usability and Humanity Requirements}
\subsection{Ease of Use Requirements}
\lips
\subsection{Personalization and Internationalization Requirements}
\lips
\subsection{Learning Requirements}
\lips
\subsection{Understandability and Politeness Requirements}
\lips
\subsection{Accessibility Requirements}
\lips

\section{Performance Requirements}
\subsection{Speed and Latency Requirements}
\lips
\subsection{Safety-Critical Requirements}
\lips
\subsection{Precision or Accuracy Requirements}
\lips
\subsection{Robustness or Fault-Tolerance Requirements}
\lips
\subsection{Capacity Requirements}
\lips
\subsection{Scalability or Extensibility Requirements}
\lips
\subsection{Longevity Requirements}
\lips

\section{Operational and Environmental Requirements}
\subsection{Expected Physical Environment}
\lips
\subsection{Wider Environment Requirements}
\lips
\subsection{Requirements for Interfacing with Adjacent Systems}
\lips
\subsection{Productization Requirements}
\lips
\subsection{Release Requirements}
\lips

\section{Maintainability and Support Requirements}
\subsection{Maintenance Requirements}
\lips
\subsection{Supportability Requirements}
\lips
\subsection{Adaptability Requirements}
\lips

\section{Security Requirements}
\subsection{Access Requirements}
\lips
\subsection{Integrity Requirements}
\lips
\subsection{Privacy Requirements}
\lips
\subsection{Audit Requirements}
\lips
\subsection{Immunity Requirements}
\lips

\section{Cultural Requirements}
\subsection{Cultural Requirements}
\lips

\section{Compliance Requirements}
\subsection{Legal Requirements}
\lips
\subsection{Standards Compliance Requirements}
\lips

\section{Open Issues}
\lips

\section{Off-the-Shelf Solutions}
\subsection{Ready-Made Products}
\lips
\subsection{Reusable Components}
\lips
\subsection{Products That Can Be Copied}
\lips

\section{New Problems}
\subsection{Effects on the Current Environment}
\lips
\subsection{Effects on the Installed Systems}
\lips
\subsection{Potential User Problems}
\lips
\subsection{Limitations in the Anticipated Implementation Environment That May
Inhibit the New Product}
\lips
\subsection{Follow-Up Problems}
\lips

\wss{The team will meet multiple times a week, once during Monday tutorial time and throughout the week as issues or concerns arise. The meetings will be conducted either online through a Teams meeting or in-person on campus. The team
will hold an official meeting with the industry advisor once a week yet the time has not been decided. The meeting with the advisor will be online on Teams for the first 3 weeks, followed by in-person meetings later on. Meetings itself will be strutured as follows: 
1. Each member will take turns giving a short recap of work they have accomplished throughout the week.
2. Members will voice any concerns or issues they may be facing.
3. Team will form a discussion and make decisions for the project.
4. Any/all questions will be doucmented for the next meeting with the advisor.

The team will go ahead and use Issues on Github to add anything they may want to talk about in the next meeting as the week progresses in order to have some form of agenda for the next meeting.}

\section{Team Communication Plan}

Issues on GitHub, Microsoft Teams for meetings, Discord for meetings with the advisor online, WhatsApp for informal project discussion.
\section{Tasks}
\subsection{Project Planning}

The project planning includes the following key elements:
\begin{enumerate}

  \item \textbf{Project Structure}
  \begin{itemize}
    \item The team consists of five members: Nivetha Kuruparan, Sevhena Walker, Tanveer Brar, Mya Hussain, and Ayushi Amin.
    \item Team members will share responsibilities for various tasks, including coding, testing, and documentation.
  \end{itemize}
    \item \textbf{Project Scope}
  \begin{itemize}
    \item Develop a refactoring library to optimize Python code for energy efficiency. 
    \item Create a plugin that utilizes the refactoring library within an IDE.
    \item Implement a reinforcement learning model to evolve refactoring recommendations over time.
  \end{itemize}
  \item \textbf{Stakeholder Management}
  \begin{itemize}
    \item Direct stakeholders include software developers, Dr. Istvan David (supervisor), and business sustainability teams.
    \item Indirect stakeholders include business leaders, end users, and regulatory bodies.
  \end{itemize}
  \item \textbf{Environment}
  \begin{itemize}
    \item Utilize Stable Baselines for reinforcement learning techniques.
    \item Use GitHub for version control and CI/CD integration.
    \item Develop using Visual Studio Code as the primary IDE.
    \item Implement a database for storing and retrieving refactoring and energy consumption metrics.
  \end{itemize}

\end{enumerate}

\subsection{Planning of the Development Phases}

The planning of the development phases is based on the deliverables submissions as follows:

\begin{enumerate}

    \item \textbf{Requirements Phase}
    \begin{itemize}
        \item Deliverable: Requirements Document (Revision 0)
        \item Due Date: October 9th, 2024
    \end{itemize}
    
    \item \textbf{Risk Assessment Phase}
    \begin{itemize}
        \item Deliverable: Hazard Analysis (Revision 0)
        \item Due Date: October 23rd, 2024
    \end{itemize}
    
    \item \textbf{Verification and Validation Planning}
    \begin{itemize}
        \item Deliverable: Verification \& Validation Plan (Revision 0)
        \item Due Date: November 1st, 2024
    \end{itemize}
    
    \item \textbf{Proof of Concept Implementation}
    \begin{itemize}
        \item Period: November 11th-22nd, 2024
    \end{itemize}
    
    \item \textbf{Design Phase}
    \begin{itemize}
        \item Deliverable: Design Document (Revision 0)
        \item Due Date: January 15th, 2025
    \end{itemize}
    
    \item \textbf{Initial Implementation and Demo}
    \begin{itemize}
        \item Deliverable: Project Demo (Revision 0)
        \item Period: February 3rd-14th, 2025
    \end{itemize}
    
    \item \textbf{Final Implementation and Testing}
    \begin{itemize}
        \item Deliverable: Final Demonstration
        \item Period: March 17th-30th, 2025
    \end{itemize}
    
    \item \textbf{Project Closure}
    \begin{itemize}
        \item Deliverable: Final Documentation
        \item Due Date: April 2nd, 2025
    \end{itemize}
    
    \item \textbf{Project Presentation}
    \begin{itemize}
        \item Event: Capstone EXPO
        \item Date: TBD
    \end{itemize}
\end{enumerate}

\section{Migration to the New Product}
\subsection{Requirements for Migration to the New Product}
\lips
\subsection{Data That Has to be Modified or Translated for the New System}
\lips

\section{Costs}
\lips
\section{User Documentation and Training}
\subsection{User Documentation Requirements}
\lips
\subsection{Training Requirements}
\lips

\section{Waiting Room}
\lips

\section{Ideas for Solution}
\lips

\newpage{}
\section*{Appendix --- Reflection}

The information in this section will be used to evaluate the team members on the
graduate attribute of Lifelong Learning.  Please answer the following questions:

\begin{enumerate}
  \item What knowledge and skills will the team collectively need to acquire to
  successfully complete this capstone project?  Examples of possible knowledge
  to acquire include domain specific knowledge from the domain of your
  application, or software engineering knowledge, mechatronics knowledge or
  computer science knowledge.  Skills may be related to technology, or writing,
  or presentation, or team management, etc.  You should look to identify at
  least one item for each team member.
  \item For each of the knowledge areas and skills identified in the previous
  question, what are at least two approaches to acquiring the knowledge or
  mastering the skill?  Of the identified approaches, which will each team
  member pursue, and why did they make this choice?
\end{enumerate}

\end{document}