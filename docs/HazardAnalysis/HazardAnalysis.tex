\documentclass{article}

\usepackage{booktabs}
\usepackage{tabularx}
\usepackage{hyperref}

\usepackage[round]{natbib}
\usepackage{tabularx}
\usepackage{booktabs}
\usepackage{xcolor}
\usepackage{blindtext}

\hypersetup{
    colorlinks=true,       % false: boxed links; true: colored links
    linkcolor=red,          % color of internal links (change box color with linkbordercolor)
    citecolor=green,        % color of links to bibliography
    filecolor=magenta,      % color of file links
    urlcolor=cyan           % color of external links
}

\title{Hazard Analysis\\\progname}

\author{\authname}

\date{}

%% Comments

\usepackage{color}

% \newif\ifcomments\commentstrue %displays comments
\newif\ifcomments\commentsfalse %so that comments do not display

\ifcomments
\newcommand{\authornote}[3]{\textcolor{#1}{[#3 ---#2]}}
\newcommand{\todo}[1]{\textcolor{red}{[TODO: #1]}}
\else
\newcommand{\authornote}[3]{}
\newcommand{\todo}[1]{}
\fi

\newcommand{\wss}[1]{\authornote{blue}{SS}{#1}} 
\newcommand{\plt}[1]{\authornote{magenta}{TPLT}{#1}} %For explanation of the template
\newcommand{\an}[1]{\authornote{cyan}{Author}{#1}}

%% Common Parts

\newcommand{\progname}{Software Engineering} % PUT YOUR PROGRAM NAME HERE
\newcommand{\authname}{\textbf{Team 4, EcoOptimizers} \\
  \\ Nivetha Kuruparan
  \\ Sevhena Walker
  \\ Tanveer Brar
  \\ Mya Hussain
\\ Ayushi Amin} % AUTHOR NAMES

\usepackage{hyperref}
\hypersetup{colorlinks=true, linkcolor=blue, citecolor=blue, filecolor=blue,
urlcolor=blue, unicode=false}
\urlstyle{same}



\begin{document}

\maketitle
\thispagestyle{empty}

~\newpage

\pagenumbering{roman}

\begin{table}[hp]
\caption{Revision History} \label{TblRevisionHistory}
\begin{tabularx}{\textwidth}{llX}
\toprule
\textbf{Date} & \textbf{Developer(s)} & \textbf{Change}\\
\midrule
Date1 & Name(s) & Description of changes\\
Date2 & Name(s) & Description of changes\\
... & ... & ...\\
\bottomrule
\end{tabularx}
\end{table}

~\newpage

\tableofcontents

~\newpage

\pagenumbering{arabic}



\section{Introduction}

\subsection{Problem Statement}
The Information and Communications Technology (ICT) sector is currently responsible
for approximately 2-4\% of global CO2 emissions, a figure projected to rise to 14\% 
by 2040 without intervention ~\citep{BelkhirAndElmeligi2018}. To align with broader 
economic sustainability goals, the ICT industry must reduce its CO2 emissions by 72\% 
by 2040 ~\citep{FreitagAndBernersLee2021}. Optimizing energy consumption in software 
systems is a complex task that cannot rely solely on software engineers, who often 
face strict deadlines and busy schedules. This creates a pressing need for supporting 
technologies that help automate this process. This project aims to develop a tool that 
applies automated refactoring techniques to optimize Python code for energy efficiency 
while preserving its original functionality. 

\subsection{Hazard Analysis Introduction}

A hazard is defined as a property or condition in the system, 
combined with a condition in the environment, that has the potential to cause harm 
or damage—referred to as loss ~\citep{Leveson2021}. In software development, hazards can take various 
forms beyond just safety hazards, including security risks, usability challenges, 
incorrect inputs, or technical limitations like lack of internet connectivity.
\\

This project focuses on developing an automated tool to refactor Python code 
for energy efficiency while preserving its original functionality. While this 
initiative holds significant potential for reducing CO2 emissions in the 
Information and Communications Technology (ICT) sector, it also introduces 
various hazards. These hazards could arise from technical shortcomings, ethical 
challenges, or the inadvertent introduction of new problems during the refactoring 
process. This hazard analysis aims to identify and assess these risks to ensure 
the successful development and adoption of the tool.

\section{Scope and Purpose of Hazard Analysis}

The scope of this hazard analysis covers the potential risks and losses associated 
with the automated refactoring tool throughout its lifecycle. The primary hazards include:

\begin{itemize}

    \item \textbf{Technical Failures}: Inaccurate refactorings, undetected code 
    smells, or energy optimization that does not meet its intended goals could 
    result in performance issues or loss of functionality. 

    \item \textbf{Security Risks}: The automated nature of the tool may introduce 
    security vulnerabilities, particularly if the refactorings unintentionally 
    affect the security posture of the original code.

    \item \textbf{User Insensitivity}: If the tool is not designed with the users 
    in mind, it could disrupt developer workflows or lead to the rejection of the 
    tool. This can result in loss of productivity or missed opportunities for 
    energy efficiency.

    \item \textbf{External Conditions}: The tool’s dependency on environmental 
    factors, such as the availability of internet connection or access to 
    third-party libraries, could limit its usefulness in certain scenarios. 
    This can lead to delays or failures in the refactoring process.

\end{itemize}

The purpose of this analysis is to identify these hazards, assess their potential 
impact, and outline strategies for mitigating them. By doing so, we aim to prevent 
losses related to time, resources, security, and the overall effectiveness of the 
tool, ensuring that it contributes positively to reducing the ICT sector's energy 
consumption and CO2 emissions.

\section{System Boundaries and Components}

\wss{Dividing the system into components will help you brainstorm the hazards.
You shouldn't do a full design of the components, just get a feel for the major
ones.  For projects that involve hardware, the components will typically include
each individual piece of hardware.  If your software will have a database, or an
important library, these are also potential components.}

\section{Critical Assumptions}

\wss{These assumptions that are made about the software or system.  You should
minimize the number of assumptions that remove potential hazards.  For instance,
you could assume a part will never fail, but it is generally better to include
this potential failure mode.}

\section{Failure Mode and Effect Analysis}

\wss{Include your FMEA table here. This is the most important part of this document.}
\wss{The safety requirements in the table do not have to have the prefix SR.
The most important thing is to show traceability to your SRS. You might trace to
requirements you have already written, or you might need to add new
requirements.}
\wss{If no safety requirement can be devised, other mitigation strategies can be
entered in the table, including strategies involving providing additional
documentation, and/or test cases.}

\section{Safety and Security Requirements}

\wss{Newly discovered requirements.  These should also be added to the SRS.  (A
rationale design process how and why to fake it.)}

\section{Roadmap}

\wss{Which safety requirements will be implemented as part of the capstone timeline?
Which requirements will be implemented in the future?}

\newpage{}

\section*{Appendix --- Reflection}

\wss{Not required for CAS 741}

\input{../Reflection.tex}

\begin{enumerate}
    \item What went well while writing this deliverable? 
    \item What pain points did you experience during this deliverable, and how
    did you resolve them?
    \item Which of your listed risks had your team thought of before this
    deliverable, and which did you think of while doing this deliverable? For
    the latter ones (ones you thought of while doing the Hazard Analysis), how
    did they come about?
    \item Other than the risk of physical harm (some projects may not have any
    appreciable risks of this form), list at least 2 other types of risk in
    software products. Why are they important to consider?
\end{enumerate}

\bibliographystyle {plainnat}
\bibliography{../../refs/References}
\end{document}