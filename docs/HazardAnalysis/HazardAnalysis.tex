\documentclass{article}

\usepackage{booktabs}
\usepackage{tabularx}
\usepackage{hyperref}
\usepackage[letterpaper, portrait, margin=1in]{geometry}

\hypersetup{
    colorlinks=true,       % false: boxed links; true: colored links
    linkcolor=red,          % color of internal links (change box color with linkbordercolor)
    citecolor=green,        % color of links to bibliography
    filecolor=magenta,      % color of file links
    urlcolor=cyan           % color of external links
}

\title{Hazard Analysis\\\progname}

\author{\authname}

\date{}

\input{../Comments}
%% Common Parts

\newcommand{\progname}{Software Engineering} % PUT YOUR PROGRAM NAME HERE
\newcommand{\authname}{\textbf{Team 6, EcoOptimizers} \\
  \\ Nivetha Kuruparan
  \\ Sevhena Walker
  \\ Tanveer Brar
  \\ Mya Hussain
\\ Ayushi Amin} % AUTHOR NAMES

\usepackage{hyperref}
\hypersetup{colorlinks=true, linkcolor=blue, citecolor=blue, filecolor=blue,
urlcolor=blue, unicode=false}
\urlstyle{same}



\begin{document}

\maketitle
\thispagestyle{empty}

~\newpage

\pagenumbering{roman}

\begin{table}[hp]
\caption{Revision History} \label{TblRevisionHistory}
\begin{tabularx}{\textwidth}{llX}
\toprule
\textbf{Date} & \textbf{Developer(s)} & \textbf{Change}\\
\midrule
Date1 & Name(s) & Description of changes\\
Date2 & Name(s) & Description of changes\\
... & ... & ...\\
\bottomrule
\end{tabularx}
\end{table}

~\newpage

\tableofcontents

~\newpage

\pagenumbering{arabic}

\wss{You are free to modify this template.}

\section{Introduction}

\wss{You can include your definition of what a hazard is here.}

\section{Scope and Purpose of Hazard Analysis}

\wss{You should say what \textbf{loss} could be incurred because of the
hazards.}

\section{System Boundaries and Components}

The system boundary refers to the library being developed and its core constituent modules, as well as peripheral tools that serve to expand on the system's utility and usability. \\

It is also important to make note of elements not controlled by the development team such as the database, the physical computer that will run the system, and any cloud hosting platforms used to utilize the system on a greater scale.

\subsection{Core Modules}

\subsubsection*{Energy Measurement Module}
This module tracks and analyzes the energy consumption of the software being refactored, providing detailed metrics that help assess the efficiency of the code before and after refactoring. External tools will be used to implement this module with the primary one being pyJoules\footnote{A python library that energies the energy footprint of a host machine.}. Other libraries may be added should the need arise. 

\subsubsection*{Testing Module}
The testing module runs automated tests on the refactored code to ensure that the functionality remains intact and that no errors are introduced during the refactoring process. Testing will be done through AST\footnote{Abstract Syntax Tree: a data structure used to represent the structure of a program.} parsing and any provided tests from the user.

\subsubsection*{Refactoring module}
This core module identifies code smells and inefficiencies in the original code and suggests or applies appropriate refactorings to optimize the code for better performance and energy efficiency. Refactoring will be done through a mix of custom-made refactoring strategies and with the help of the python library, Rope.

\subsubsection*{Reinforcement Learning Model}
The reinforcement learning model uses data from previous refactorings to improve its suggestions, helping the system learn and refine its refactoring strategies based on outcomes and energy consumption metrics. The model shall be built with the help of the machine learning library, PyTorch.

\subsection{Peripherals}

\subsubsection*{Visual Studio Code (VS Code) Extension}
The VS Code extension provides a user-friendly interface within the IDE\footnote{Integrated Development Environment}, allowing developers to interact with the refactoring tool, view energy consumption metrics, and apply refactorings suggestions directly from their development environment.

\subsubsection*{GitHub Action}
The GitHub Action automates the refactoring process within CI/CD workflows, applying refactoring suggestions and running energy consumption analyses during code integration, ensuring consistent energy-efficient practices.

\subsubsection*{Web Client}
The web client offers a user interface that allows users to interact with the refactoring system remotely, enabling them to view energy consumption reports, and track performance metrics from a browser.

\section{Critical Assumptions}

\wss{These assumptions that are made about the software or system.  You should
minimize the number of assumptions that remove potential hazards.  For instance,
you could assume a part will never fail, but it is generally better to include
this potential failure mode.}

\section{Failure Mode and Effect Analysis}

\wss{Include your FMEA table here. This is the most important part of this document.}
\wss{The safety requirements in the table do not have to have the prefix SR.
The most important thing is to show traceability to your SRS. You might trace to
requirements you have already written, or you might need to add new
requirements.}
\wss{If no safety requirement can be devised, other mitigation strategies can be
entered in the table, including strategies involving providing additional
documentation, and/or test cases.}

\section{Safety and Security Requirements}

\wss{Newly discovered requirements.  These should also be added to the SRS.  (A
rationale design process how and why to fake it.)}

\section{Roadmap}

\wss{Which safety requirements will be implemented as part of the capstone timeline?
Which requirements will be implemented in the future?}

\newpage{}

\section*{Appendix --- Reflection}

\wss{Not required for CAS 741}

\input{../Reflection.tex}

\begin{enumerate}
    \item What went well while writing this deliverable? 
    \item What pain points did you experience during this deliverable, and how
    did you resolve them?
    \item Which of your listed risks had your team thought of before this
    deliverable, and which did you think of while doing this deliverable? For
    the latter ones (ones you thought of while doing the Hazard Analysis), how
    did they come about?
    \item Other than the risk of physical harm (some projects may not have any
    appreciable risks of this form), list at least 2 other types of risk in
    software products. Why are they important to consider?
\end{enumerate}

\end{document}