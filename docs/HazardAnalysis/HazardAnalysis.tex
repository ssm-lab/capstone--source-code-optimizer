\documentclass{article}

\usepackage{booktabs}
\usepackage{tabularx}
\usepackage{hyperref}
\usepackage[letterpaper, portrait, margin=1in]{geometry}

\usepackage[round]{natbib}
\usepackage{tabularx}
\usepackage{booktabs}
\usepackage{xcolor}
\usepackage{blindtext}
\usepackage{enumitem}

\hypersetup{
    colorlinks=true,       % false: boxed links; true: colored links
    linkcolor=red,          % color of internal links (change box color with linkbordercolor)
    citecolor=green,        % color of links to bibliography
    filecolor=magenta,      % color of file links
    urlcolor=cyan           % color of external links
}

\title{Hazard Analysis\\\progname}

\author{\authname}

\date{}

\newcommand{\lips}{\textit{Insert your content here.}}

%% Comments

\usepackage{color}

% \newif\ifcomments\commentstrue %displays comments
\newif\ifcomments\commentsfalse %so that comments do not display

\ifcomments
\newcommand{\authornote}[3]{\textcolor{#1}{[#3 ---#2]}}
\newcommand{\todo}[1]{\textcolor{red}{[TODO: #1]}}
\else
\newcommand{\authornote}[3]{}
\newcommand{\todo}[1]{}
\fi

\newcommand{\wss}[1]{\authornote{blue}{SS}{#1}} 
\newcommand{\plt}[1]{\authornote{magenta}{TPLT}{#1}} %For explanation of the template
\newcommand{\an}[1]{\authornote{cyan}{Author}{#1}}

%% Common Parts

\newcommand{\progname}{Software Engineering} % PUT YOUR PROGRAM NAME HERE
\newcommand{\authname}{\textbf{Team 4, EcoOptimizers} \\
  \\ Nivetha Kuruparan
  \\ Sevhena Walker
  \\ Tanveer Brar
  \\ Mya Hussain
\\ Ayushi Amin} % AUTHOR NAMES

\usepackage{hyperref}
\hypersetup{colorlinks=true, linkcolor=blue, citecolor=blue, filecolor=blue,
urlcolor=blue, unicode=false}
\urlstyle{same}



\begin{document}

\maketitle
\thispagestyle{empty}

~\newpage

\pagenumbering{roman}

\begin{table}[hp]
\caption{Revision History} \label{TblRevisionHistory}
\begin{tabularx}{\textwidth}{llX}
\toprule
\textbf{Date} & \textbf{Developer(s)} & \textbf{Change}\\
\midrule
Date1 & Name(s) & Description of changes\\
Date2 & Name(s) & Description of changes\\
... & ... & ...\\
\bottomrule
\end{tabularx}
\end{table}

~\newpage

\tableofcontents

~\newpage

\pagenumbering{arabic}



\section{Introduction}

\subsection{Problem Statement}
The Information and Communications Technology (ICT) sector is currently responsible
for approximately 2-4\% of global CO2 emissions, a figure projected to rise to 14\% 
by 2040 without intervention ~\citep{BelkhirAndElmeligi2018}. To align with broader 
economic sustainability goals, the ICT industry must reduce its CO2 emissions by 72\% 
by 2040 ~\citep{FreitagAndBernersLee2021}. Optimizing energy consumption in software 
systems is a complex task that cannot rely solely on software engineers, who often 
face strict deadlines and busy schedules. This creates a pressing need for supporting 
technologies that help automate this process. This project aims to develop a tool that 
applies automated refactoring techniques to optimize Python code for energy efficiency 
while preserving its original functionality. 

\subsection{Hazard Analysis Introduction}

A hazard is defined as a property or condition in the system, 
combined with a condition in the environment, that has the potential to cause harm 
or damage—referred to as loss ~\citep{Leveson2021}. In software development, hazards can take various 
forms beyond just safety hazards, including security risks, usability challenges, 
incorrect inputs, or technical limitations like lack of internet connectivity.
\\

This project focuses on developing an automated tool to refactor Python code 
for energy efficiency while preserving its original functionality. While this 
initiative holds significant potential for reducing CO2 emissions in the 
Information and Communications Technology (ICT) sector, it also introduces 
various hazards. These hazards could arise from technical shortcomings, ethical 
challenges, or the inadvertent introduction of new problems during the refactoring 
process. This hazard analysis aims to identify and assess these risks to ensure 
the successful development and adoption of the tool.

\section{Scope and Purpose of Hazard Analysis}

The scope of this hazard analysis covers the potential risks and losses associated 
with the automated refactoring tool throughout its lifecycle. The primary hazards include:

\begin{itemize}

    \item \textbf{Technical Failures}: Inaccurate refactorings, undetected code 
    smells, or energy optimization that does not meet its intended goals could 
    result in performance issues or loss of functionality. 

    \item \textbf{Security Risks}: The automated nature of the tool may introduce 
    security vulnerabilities, particularly if the refactorings unintentionally 
    affect the security posture of the original code.

    \item \textbf{User Insensitivity}: If the tool is not designed with the users 
    in mind, it could disrupt developer workflows or lead to the rejection of the 
    tool. This can result in loss of productivity or missed opportunities for 
    energy efficiency.

    \item \textbf{External Conditions}: The tool’s dependency on environmental 
    factors, such as the availability of internet connection or access to 
    third-party libraries, could limit its usefulness in certain scenarios. 
    This can lead to delays or failures in the refactoring process.

\end{itemize}

The purpose of this analysis is to identify these hazards, assess their potential 
impact, and outline strategies for mitigating them. By doing so, we aim to prevent 
losses related to time, resources, security, and the overall effectiveness of the 
tool, ensuring that it contributes positively to reducing the ICT sector's energy 
consumption and CO2 emissions.

\section{System Boundaries and Components}

The system boundary refers to the library being developed and its core constituent modules, as well as peripheral tools that serve to expand on the system's utility and usability. \\

It is also important to make note of elements not controlled by the development team such as the database, the physical computer that will run the system, and any cloud hosting platforms used to utilize the system on a greater scale.

\subsection{Core Modules}

\subsubsection*{Energy Measurement Module}
This module tracks and analyzes the energy consumption of the software being refactored, providing detailed metrics that help assess the efficiency of the code before and after refactoring. External tools will be used to implement this module with the primary one being pyJoules\footnote{A python library that energies the energy footprint of a host machine.}. Other libraries may be added should the need arise. 

\subsubsection*{Testing Module}
The testing module runs automated tests on the refactored code to ensure that the functionality remains intact and that no errors are introduced during the refactoring process. Testing will be done through AST\footnote{Abstract Syntax Tree: a data structure used to represent the structure of a program.} parsing and any provided tests from the user.

\subsubsection*{Refactoring module}
This core module identifies code smells and inefficiencies in the original code and suggests or applies appropriate refactorings to optimize the code for better performance and energy efficiency. Refactoring will be done through a mix of custom-made refactoring strategies and with the help of the python library, Rope.

\subsubsection*{Reinforcement Learning Model}
The reinforcement learning model uses data from previous refactorings to improve its suggestions, helping the system learn and refine its refactoring strategies based on outcomes and energy consumption metrics. The model shall be built with the help of the machine learning library, PyTorch.

\subsection{Peripherals}

\subsubsection*{Visual Studio Code (VS Code) Extension}
The VS Code extension provides a user-friendly interface within the IDE\footnote{Integrated Development Environment}, allowing developers to interact with the refactoring tool, view energy consumption metrics, and apply refactorings suggestions directly from their development environment.

\subsubsection*{GitHub Action}
The GitHub Action automates the refactoring process within CI/CD workflows, applying refactoring suggestions and running energy consumption analyses during code integration, ensuring consistent energy-efficient practices.

\subsubsection*{Web Client}
The web client offers a user interface that allows users to interact with the refactoring system remotely, enabling them to view energy consumption reports, and track performance metrics from a browser.

\section{Critical Assumptions}

\begin{itemize}
    \item The Energy Measurement Model will provide accurate and consistent energy consumption metrics across different platforms (Windows, macOS, Linux). There are no discrepancies in measurements due to platform differences that could result in ineffective refactoring.
    \item The Testing Module is provided with automated tests that have enough coverage to detect post refactoring bugs, functionality regressions, etc.
    \item Code smells identified by the Refactoring Module always involve code that could be more energy efficient.
    \item Custom-made refactoring strategies and Rope are capable of generating effective and correct refactoring.
    \item Sufficient data sets are available for the reinforcement learning model to provide increasingly accurate and efficient refactoring suggestions over time.
    \item GitHub Actions, which is a third-party dependency for the DevOps integration, is not suspended for a prolonged period of time.
\end{itemize}

\section{Failure Mode and Effect Analysis}

\wss{Include your FMEA table here. This is the most important part of this document.}
\wss{The safety requirements in the table do not have to have the prefix SR.
The most important thing is to show traceability to your SRS. You might trace to
requirements you have already written, or you might need to add new
requirements.}
\wss{If no safety requirement can be devised, other mitigation strategies can be
entered in the table, including strategies involving providing additional
documentation, and/or test cases.}

\section{Safety and Security Requirements}

\begin{enumerate}[label=SCR \arabic*., wide=0pt, leftmargin=*]

    \item \emph{The system shall log all energy consumption measurements with timestamps and indicate which processes were measured to aid in future analysis and troubleshooting.}\\
    {\bf Rationale:} Detailed logging with timestamps and process attribution ensures accurate energy data and helps identify delays or misattributions.\\
    {\bf Fit Criterion:} 100\% of energy analysis logs must include timestamps and process-level breakdowns of all measured processes.\\
    {\bf Associated Hazards:} HZ-1, HZ-2, HZ-3\\
    {\bf Priority:} High

    \item \emph{The system shall ensure that all refactored code has comprehensive test coverage and passes performance metrics such as energy efficiency, speed, and memory usage.}\\
    {\bf Rationale:} Proper test coverage and performance checks prevent faulty code from being introduced and ensure refactorings improve or maintain performance.\\
    {\bf Fit Criterion:} 100\% of refactorings must pass tests covering all code paths, and performance must remain within a 5\% tolerance across energy, speed, and memory metrics.\\
    {\bf Associated Hazards:} HZ-5, HZ-6, HZ-7, HZ-10\\
    {\bf Priority:} High

    \item \emph{The system shall check for necessary system-level permissions to access energy consumption data and alert users if permissions are missing.}\\
    {\bf Rationale:} Lack of access may lead to failure in energy data retrieval, which can hinder the accuracy of analysis.\\
    {\bf Fit Criterion:} 100\% of runs shall check for and request permissions if required, and alert the user in case of failures.\\
    {\bf Associated Hazards:} HZ-3\\
    {\bf Priority:} High

    \item \emph{The system shall ensure version control for each refactoring, allowing changes to be reverted in case of errors.}\\
    {\bf Rationale:} Version control helps prevent loss of code or data and allows developers to revert refactorings if necessary.\\
    {\bf Fit Criterion:} 100\% of changes shall be recorded, allowing full reversion with no data loss.\\
    {\bf Associated Hazards:} HZ-9\\
    {\bf Priority:} High

    \item \emph{The system shall detect and exempt external library dependencies from refactorings to avoid compatibility issues.}\\
    {\bf Rationale:} Modifying external dependencies could lead to system instability or incompatibility with other tools or frameworks.\\
    {\bf Fit Criterion:} 100\% detection accuracy for external library code during refactoring.\\
    {\bf Associated Hazards:} HZ-11\\
    {\bf Priority:} Medium

    \item \emph{The system shall not refactor or alter code containing sensitive information (noted by user), ensuring security is maintained.}\\
    {\bf Associated Rationale:} Refactoring sensitive code may introduce vulnerabilities and compromise security.\\
    {\bf Fit Criterion:} 100\% of refactorings must pass a security check to avoid tampering with sensitive information.\\
    {\bf Associated Hazards:} HZ-12\\
    {\bf Priority:} High

    \item \emph{The reinforcement learning model shall be trained on diverse datasets and periodically audited to avoid bias and prevent degradation.}\\
    {\bf Rationale:} Overfitting or model degradation can lead to suboptimal or biased refactorings, impacting the system's effectiveness.\\
    {\bf Fit Criterion:} 95\% of refactorings should be equally effective across different types of projects, and model audits should occur at least quarterly.\\
    {\bf Associated Hazards:} HZ-13, HZ-14, HZ-15\\
    {\bf Priority:} Medium

    \item \emph{The system shall implement memory leak detection during refactoring and alert users if any issues are detected.}\\
    {\bf Rationale:} Memory leaks may cause system crashes and reduce performance.\\
    {\bf Fit Criterion:} 100\% of memory leak incidents should trigger an error alert and resolution process.\\
    {\bf Associated Hazards:} HZ-8\\
    {\bf Priority:} Medium

    \item \emph{The system shall require user approval for high-impact refactorings or those with low confidence, providing visibility and oversight for critical changes.}\\
    {\bf Rationale:} Automated decisions could introduce errors without human oversight, and users should be aware of significant changes.\\
    {\bf Fit Criterion:} 100\% of high-risk or low-confidence refactorings must require user approval before proceeding.\\
    {\bf Associated Hazards:} HZ-16\\
    {\bf Priority:} High

    \item \emph{The system shall alert users to any delays or failures in reporting energy consumption, ensuring transparency in reporting.}\\
    {\bf Rationale:} Users need to be aware of any issues in energy reporting to troubleshoot and resolve potential problems.\\
    {\bf Fit Criterion:} 100\% of energy measurement delays or failures must trigger a user alert.\\
    {\bf Associated Hazards:} HZ-2, HZ-3\\
    {\bf Priority:} High

\end{enumerate}

\section{Roadmap}

\wss{Requirements that will be implemented during the capstone timeline:}
\begin{itemize}
    \item SCR 1
    \item SCR 2
    \item SCR 3
    \item SCR 4
    \item SCR 5
    \item SCR 6
    \item SCR 9
    \item SCR 10
\end{itemize}

\wss{Requirements implemented in the future:}
\begin{itemize}
    \item SCR 7: This will be audited on a regular basis which will be a future implementation.
    \item SCR 8: This can be implemented in the future as it is not a high priority and not the biggest concern to this project.
\end{itemize}

\newpage{}

\section*{Appendix --- Reflection}

\subsubsection*{Nivetha Kuruparan}

\begin{enumerate}
  \item \textit{What went well while writing this deliverable?}
  
  \lips

  \item \textit{What pain points did you experience during this deliverable, and how did you resolve them?}
  
  \lips

\end{enumerate}

\subsubsection*{Sevhena Walker}

\begin{enumerate}
  \item \textit{What went well while writing this deliverable?}
  
  One thing that went really well during the hazard analysis was how it helped me catch issues I’d originally missed. The structured process made it easier to step back and look at our project from a different perspective, which helped highlight potential risks I hadn’t thought of before. 

  \item \textit{What pain points did you experience during this deliverable, and how did you resolve them?}
  
  I'll be honest the worst part of this deliverable was formatting the FMEA table in latex. It doesn't seem right to talk about pain points without mentioning the one thing that truly had me pulling my hair out. In terms of the actual content of the deliverable, brainstorming hazards was challenging, but not exactly a pain. The challenging part was coming up with solution or mitigating actions to counter those hazards. Some components, like the reinforcement model, I have truly no experience with and its pretty hard to come up with solutions to risks you have never even experienced, let alone thought of. 
  
\end{enumerate}

\subsubsection*{Tanveer Brar}

\begin{enumerate}
    \item \textit{What went well while writing this deliverable?}

    This deliverable was pretty short compared to previous ones but we were still on top of our toes when it came to planning it within the team. I like that we allowed everyone to pick up topics that interested them the most and left the key piece of work(FMEA table) to be worked on collaboratively in Overleaf by everyone. Timely spiliting of the work gave us ample time to finish individual assignments as well as review other people's contributions.

    \item \textit{What pain points did you experience during this deliverable, and how did you resolve them?}

    The main challenge that I faced was mapping the Failure Modes to appropriate security requirements. Some of the failure modes that I came up with aligned with the security requirements written previously, but more content needed to be added to those. To resolve this, I added the additional description needed for these security requirements for some hazards and created new requirements for others.

\end{enumerate}

\subsubsection*{Mya Hussain}

\begin{enumerate}
  \item \textit{What went well while writing this deliverable?}
  
  We divided up the work early and were all able to complete sections at our own 
  pace or ahead of time depending on our midterm schedules. This week was 
  particularly busy because all of us had midterms so I was able to complete 
  my section during reading week to reduce the capstone workload during the week. 
  Although I will say it's a little disappointing that every time we are done on 
  time (which has been every time so far) the deliverable is extended last minute. 
  I don't want to complain too much though because I have a feeling that if I do 
  complain it won't be extended next time I'd actually like it to be. So far team 
  dynamics and morale have been good. I appreciate the level of organization we've 
  been able to have so far as it made collaborating so much smoother and has helped 
  everyone stay on track with our tasks.

  \item \textit{What pain points did you experience during this deliverable, and how did you resolve them?}
  
  Determining which factors qualify as hazards for our analysis was somewhat unclear.
  A hazard is defined as anything with the potential to cause harm or loss, yet certain    
  risks may emerge from poor design, complicating our decision on whether to include them.    
  For example, user interface hazards like "the tool does not provide clear feedback    
  to the user after refactoring" can technically be classified as a hazard. While we    
  aim to mitigate team-imposed hazards, it raises the question of whether we should    
  simply avoid designing a flawed product in the first place, and not include these    
  hazards in the analysis or if we should do a worst-case analysis and include every    
  possible pitfall. The same argument could be made for some security hazards for example    
  "while parsing user input code, the software encounters malware and executes it,"    
  avoiding this is something a good tool should already have built in, so it begs    
  the question of "how bad do we envision our final product when analyzing hazards?"    
  We were able to get some clarification on this in our TA 1-1 meeting but ultimately    
  tried to keep it high level so our report didn't end up being too long.

\end{enumerate}

\subsubsection*{Ayushi Amin}

\begin{enumerate}
  \item \textit{What went well while writing this deliverable?}
  
  I think one of the best things about writing this deliverable was how well we collaborated using Overleaf. It made it super easy to 
  work together on the FMEA table. We divided up the work, so everyone had their own sections to focus on, but we also helped each 
  other out when needed. This teamwork really made a difference because we could share ideas and give feedback in real time.
  Even though we had midterms this week, which delayed our progress a bit, everything ended up working out. We managed our time well, 
  and I was impressed with how we all stayed on track despite the busy schedule. It felt good to see how our combined efforts came 
  together in the final product. Overall, I think our collaboration really strengthened the quality of our work.

  \item \textit{What pain points did you experience during this deliverable, and how did you resolve them?}
  
  One big challenge I faced was figuring out the difference between general risks and specific hazards for our project. At first, it was 
  a bit confusing, and we spent some time debating whether certain issues were specific enough. To resolve this, I looked up examples from
  other projects, which helped clarify things for everyone. Overall, even though there were some bumps along the way, working through these 
  challenges taught me a lot about hazard analysis and teamwork in software development.
  
\end{enumerate}

\subsubsection*{Group Answer}

\begin{enumerate}
    \item[3.] \textit{Which of your listed risks had your team thought of before this deliverable, and which did you think of while doing this deliverable? For the latter ones (ones you thought of while doing the Hazard Analysis), how did they come about?}
    
    \lips

    \item[4.] \textit{Other than the risk of physical harm (some projects may not have any appreciable risks of this form),
    list at least 2 other types of risk in software products. Why are they important to consider?}

    \lips

\end{enumerate}

\bibliographystyle {plainnat}
\bibliography{../../refs/References}

\end{document}