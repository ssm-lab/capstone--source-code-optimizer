\documentclass[12pt, titlepage]{article}
\usepackage[letterpaper, portrait, margin=1in]{geometry}
\usepackage{booktabs}
\usepackage{tabularx}
\usepackage{hyperref}
\hypersetup{
  colorlinks,
  citecolor=black,
  filecolor=black,
  linkcolor=red,
  urlcolor=blue
}
\usepackage{longtable}
\usepackage{colortbl}
\usepackage{graphicx}
\usepackage{placeins}
\usepackage{array}
\usepackage{float}
\usepackage{colortbl}
\usepackage{caption}
\usepackage{graphicx}
\captionsetup[table]{width=0.9\textwidth}

\usepackage{listings}
\lstset{
  basicstyle=\ttfamily, % Use monospaced font
  breaklines=true,      % Enable line breaks
}
\usepackage{csvsimple}

\input{../Comments}
%% Common Parts

\newcommand{\progname}{Software Engineering} % PUT YOUR PROGRAM NAME HERE
\newcommand{\authname}{\textbf{Team 6, EcoOptimizers} \\
  \\ Nivetha Kuruparan
  \\ Sevhena Walker
  \\ Tanveer Brar
  \\ Mya Hussain
\\ Ayushi Amin} % AUTHOR NAMES

\usepackage{hyperref}
\hypersetup{colorlinks=true, linkcolor=blue, citecolor=blue, filecolor=blue,
urlcolor=blue, unicode=false}
\urlstyle{same}



%

\begin{document}

\title{Verification and Validation Report: \progname}
\author{\authname}
\date{\today}

\maketitle

\pagenumbering{roman}

\section*{Revision History}

\begin{tabularx}{\textwidth}{p{3cm}p{2cm}X}
  \toprule {\bf Date} & {\bf Version} & {\bf Notes}\\
  \midrule
  March 8th, 2025 & 0.0 & Started VnV Report\\
  \bottomrule
\end{tabularx}

~\newpage

\section*{Symbols, Abbreviations and Acronyms}

\renewcommand{\arraystretch}{1.2}
\begin{tabular}{l l}
  \toprule
  \textbf{symbol} & \textbf{description}\\
  \midrule
  T & Test\\
  TC & Test Case\\
  VSCode & Visual Studio Code\\
  \bottomrule
\end{tabular}\\

\wss{symbols, abbreviations or acronyms -- you can reference the SRS
tables if needed}

\newpage

\tableofcontents

\listoftables %if appropriate

\listoffigures %if appropriate

\newpage

\pagenumbering{arabic}

This Verification and Validation (V\&V) report outlines the testing
process used to ensure the accuracy, reliability, and performance of
our system. It details our verification approach, test cases, and
validation results, demonstrating that the system meets its
requirements and functions as intended. Key findings and resolutions
are also discussed.

\section{Functional Requirements Evaluation}
\subsection{Code Input Acceptance Tests}
\begin{enumerate}

  \item \textbf{test-FR-1A Valid Python File Acceptance} \\[2mm]
    The \textbf{valid Python file acceptance test} ensures that the
    system correctly processes a syntactically valid Python file
    without errors. A correctly formatted Python file was provided as
    input, and the expected result was that the system should accept
    the file without issue. The \textbf{actual result} confirmed that
    the system successfully processed the valid file without
    generating any errors.

  \item \textbf{test-FR-1A-2 Feedback for Python File with Bad Syntax} \\[2mm]
    This test verifies that the system correctly handles Python files
    containing deliberate syntax errors. A Python file with syntax
    errors was fed into the system, and the expected result was that
    the system should reject the file and provide an appropriate
    error message indicating the syntax issue. The \textbf{actual
    result} confirmed that the system correctly identified the syntax
    errors and displayed the expected error message.

  \item \textbf{test-FR-1A-3 Feedback for Non-Python File} \\[2mm]
    The \textbf{non-Python file test} ensures that the system
    correctly rejects unsupported file types and provides clear
    feedback. A document file (\texttt{document.txt}) and a script
    with an incorrect file extension (\texttt{script.js}) were
    tested. The expected result was that the system should reject the
    files and return an error message indicating that the file format
    is not supported. The \textbf{actual result} confirmed that the
    system correctly flagged the non-Python files and provided the
    appropriate error message.

\end{enumerate}

\subsection{Code Smell Detection Tests and Refactoring Suggestion (RS) Tests}

This area includes tests to verify the detection and refactoring of specified code smells that impact energy efficiency. These tests will be done through unit testing.

\begin{enumerate}
  \item \textbf{test-FR-IA-1 Successful Refactoring Execution} \\[2mm]
    \textbf{Control:} Automated \\
    \textbf{Initial State:} Tool is idle in the VS Code environment. \\
    \textbf{Input:} A valid Python file with a detectable code smell. \\
    \textbf{Output:} The system applies the appropriate refactoring and updates the code view. \\
    \textbf{Test Case Derivation:} Ensures the tool correctly identifies a smell (e.g., LEC001), chooses an applicable refactoring, and applies it successfully, per FR2 and FR3. \\
    \textbf{How test will be performed:} Provide a valid Python file containing a known smell, trigger refactoring via the VS Code interface, and confirm the output includes refactored code as expected.

  \item \textbf{test-FR-IA-2 No Available Refactorer Handling} \\[2mm]
    \textbf{Control:} Automated \\
    \textbf{Initial State:} Tool is idle. \\
    \textbf{Input:} A valid Python file containing a code smell that does not yet have a supported refactorer. \\
    \textbf{Output:} The system does not apply changes and logs or displays an informative message. \\
    \textbf{Test Case Derivation:} Verifies that unsupported code smells are gracefully handled without errors, per FR2. \\
    \textbf{How test will be performed:} Provide a valid Python file with an unsupported smell and observe that the system notifies the user without attempting modification.

  \item \textbf{test-FR-IA-3 Multiple Refactoring Calls on Same File} \\[2mm]
    \textbf{Control:} Automated \\
    \textbf{Initial State:} Tool is idle. \\
    \textbf{Input:} A valid Python file with a detectable code smell, refactored more than once. \\
    \textbf{Output:} The tool processes the file repeatedly and applies changes incrementally. \\
    \textbf{Test Case Derivation:} Confirms the system can handle repeated invocations and re-apply applicable refactorings, per FR3. \\
    \textbf{How test will be performed:} Refactor a file containing a supported smell multiple times and verify that each run performs valid operations and results in updated outputs.

  \item \textbf{test-FR-IA-4 Handling Empty Modified Files List} \\[2mm]
    \textbf{Control:} Automated \\
    \textbf{Initial State:} Tool is idle. \\
    \textbf{Input:} A valid Python file where the code smell is detected, but the refactorer makes no modifications. \\
    \textbf{Output:} The system does not generate output files and notifies the user appropriately. \\
    \textbf{Test Case Derivation:} Confirms the tool handles no-op refactorers correctly, per FR4. \\
    \textbf{How test will be performed:} Supply a file where the refactorer returns an unchanged version of the code and verify that no new files are created and that appropriate feedback is displayed or logged.
\end{enumerate}

\subsection{Output Validation Tests}
\begin{enumerate}
  \item \textbf{test-FR-OV-1 Verification of Valid Python Output} \\[2mm]
    The \textbf{output validation test} ensures that refactored
    Python code remains syntactically correct and compliant with
    Python standards. This validation is crucial for maintaining
    \textbf{functional requirement FR3}, as it confirms that the
    refactored code behaves identically to the original but with
    improved efficiency.

    A Python file with detected code smells was refactored, and the
    expected result was that the optimized code should pass a syntax
    check and retain its original functionality. The \textbf{actual
    result} confirmed that the refactored code was valid, passed
    linting checks, and maintained correctness.
\end{enumerate}

\subsection{Tests for Reporting Functionality}
The reporting functionality of the tool is a critical feature that
provides comprehensive insights into the refactoring process,
including detected code smells, applied refactorings, energy
consumption measurements, and test results. These tests ensure that
the reporting feature operates correctly and delivers accurate,
well-structured information as specified in \textbf{functional
requirement FR9}.\\

\noindent At this stage, the reporting functionality is still under
development, and testing has not yet been conducted. The tests
outlined below will be performed in \textbf{Revision 1} once the
reporting feature is fully implemented.

\begin{enumerate}
  \item \textbf{test-FR-RP-1 A Report With All Components Is Generated} \\[2mm]
    This test ensures that the tool generates a comprehensive report
    that includes all necessary information required by \textbf{FR9}.
    The system should produce a structured summary of the refactoring
    process, displaying detected code smells, applied refactorings,
    and energy consumption metrics.

    \textbf{Planned Test Execution:} After refactoring, the tool will
    invoke the report generation feature, and a user will validate
    that the output meets the structure and content specifications.

  \item \textbf{test-FR-RP-2 Validation of Code Smell and Refactoring
    Data in Report} \\[2mm]
    This test will verify that the report correctly includes details
    on detected code smells and refactorings, ensuring compliance
    with \textbf{FR9}.

    \textbf{Planned Test Execution:} The tool will generate a report,
    and its contents will be compared with the detected code smells
    and refactorings to confirm accuracy.

  \item \textbf{test-FR-RP-3 Energy Consumption Metrics Included in
    Report} \\[2mm]
    This test will validate that the reporting feature correctly
    includes energy consumption measurements before and after
    refactoring, aligning with \textbf{FR9}.

    \textbf{Planned Test Execution:} A user will analyze the energy
    consumption metrics in the generated report to ensure they
    accurately reflect the measurements taken during the refactoring process.

  \item \textbf{test-FR-RP-4 Functionality Test Results Included in
    Report} \\[2mm]
    This test will ensure that the reporting functionality accurately
    reflects the results of the test suite, summarizing test
    pass/fail outcomes after refactoring.

    \textbf{Planned Test Execution:} The tool will generate a report,
    and validation will be conducted to confirm that it includes a
    summary of test results matching the actual execution outcomes.
\end{enumerate}

\subsection{Documentation Availability Tests}
The following tests will ensure that the necessary documentation is
available as per \textbf{FR10}. Since documentation is still under
development, these tests have not yet been conducted and will be
included in \textbf{Revision 1}.

\begin{enumerate}
  \item \textbf{test-FR-DA-1 Test for Documentation Availability} \\[2mm]
    This test verifies that the system provides proper documentation
    covering installation, usage, and troubleshooting.

    \textbf{Planned Test Execution:} Review the documentation for
    completeness, clarity, and accuracy, ensuring it meets \textbf{FR10}.
\end{enumerate}

\subsection{IDE Extension Tests}
The following tests ensure that users can integrate the tool as a VS
Code extension in compliance with \textbf{FR11}. Local testing has
been conducted successfully, confirming the extension's ability to
function within the development environment. Once all features are
implemented, the extension will be packaged and tested in a deployed
environment.

\begin{enumerate}
  \item \textbf{test-FR-IE-1 Installation of Extension in Visual
    Studio Code} \\[2mm]
    This test ensures that the refactoring tool extension can be
    installed from the Visual Studio Marketplace.

    \textbf{Test Execution:} The extension was installed locally, and
    its presence in the Extensions View was confirmed.

    \textbf{Future Testing:} Once all features are implemented, the
    extension will be zipped, packaged, and tested as a published extension.

  \item \textbf{test-FR-IE-2 Running the Extension in Visual Studio
    Code} \\[2mm]
    This test validates that the extension functions correctly within
    the development environment, detecting code smells and suggesting
    refactorings.

    \textbf{Test Execution:} Local tests confirmed that activating
    the extension successfully detects code smells and applies refactorings.

    \textbf{Future Testing:} Once the extension is packaged,
    additional tests will be conducted to confirm functionality in a
    deployed environment.
\end{enumerate}

\section{Nonfunctional Requirements Evaluation}

\subsection{Usability}

\subsection*{Key Findings}
\begin{itemize}
  \item The extension demonstrated strong functionality in detecting
    code smells and providing refactoring suggestions.
  \item Participants appreciated the \textbf{preview feature} and
    \textbf{energy savings feedback}.
  \item Major usability issues included \textbf{sidebar visibility},
    \textbf{refactoring speed}, and \textbf{UI clarity}.
\end{itemize}

\section*{Methodology}
The usability test involved 5 student developers familiar with VSCode
but with no prior experience using the extension. Participants
performed tasks such as detecting code smells, refactoring single and
multi-file smells, and customizing settings. Metrics included task
completion rate, error rate, and user satisfaction scores. Additional
qualitative data was collected using surveys that gathered background
information of the participants as well as their opinions post
testing (\ref{appendix:usability}).

\section*{Results}

The following is an overview of the most significant task that the
test participants performed. Information on the tasks themselves can
be found in the Appendix (\ref{appendix:usability}).

\subsection*{Quantitative Results}
\begin{itemize}
  \item \textbf{Task Completion Rate:}
    \begin{itemize}
      \item \textbf{Task 1-3 (Smell Detection):} 100\% success rate.
      \item \textbf{Task 4 (Initiate Refactoring):} 100\% success rate.
      \item \textbf{Task 6 (Multi-File Refactoring):} 60\% success
        rate (participants struggled with identifying clickable file names).
      \item \textbf{Task 7 (Smell Settings):} 100\% success rate.
    \end{itemize}
  \item \textbf{User Satisfaction:}
    \begin{itemize}
      \item Confidence in Using the Tool: \textbf{4.2/5}.
      \item Satisfaction with UI Design: \textbf{4.0/5}.
      \item Trust in Refactoring Suggestions: \textbf{4.5/5}.
    \end{itemize}
\end{itemize}

\begin{figure}[H]
  \centering
  \includegraphics[width=0.7\textwidth]{../Images/usability-satisfaction-graph.png}
  \label{img:usability-satisfaction}
  \caption{User Satisfaction Survey Data}
\end{figure}

\subsection*{Qualitative Results}
Participants found the code smell detection intuitive and accurate,
and they appreciated the preview feature and Accept/Reject buttons.
However, they struggled with sidebar visibility, refactoring speed,
and UI clarity. Hover descriptions were overwhelming, and some
elements (e.g., ``(6/3)'') were unclear.

\section*{Discussion}
The usability test revealed that the extension performs well in
detecting code smells and providing refactoring suggestions.
Participants appreciated the energy savings feedback but requested
clearer explanations of how refactoring improves energy efficiency.
The sidebar and refactoring process were identified as major pain
points, requiring immediate attention.\\

The extension met its core functionality objectives but fell short in
UI clarity and performance reliability. Participants expressed
interest in using the extension in the future, provided the
identified issues are addressed. The test highlighted the need for
better onboarding, clearer documentation, and performance
optimizations to enhance user satisfaction and adoption.

\section*{Feedback and Implementation Plan}
The following table summarizes participant feedback and whether the
suggested changes will be implemented:

\begin{table}[H]
  \centering
  \begin{tabular}{>{\raggedright\arraybackslash}p{6cm}p{3.2cm}>{\raggedright\arraybackslash}p{5cm}}
    \toprule \textbf{Feedback} & \textbf{Implementation Decision} &
    \textbf{Reason} \\
    \midrule
    Relocate the sidebar or change its colour for better visibility.
    & Partial & The relocation of the sidebar is not something that
    is in scope during the development period. \\
    Make Accept/Reject buttons more prominent and visually distinct.
    & Yes & High user frustration. \\
    Allow users to customize colours for different types of smells. &
    Yes & Enhances user experience. \\
    Optimize the refactoring process to reduce wait times. & No &
    This is a time intensive ask that is not in scope. \\
    Add progress bars or loading messages to manage user
    expectations. & Yes & Additional messages will be added to the UI.\\
    Provide step-by-step instructions and a tutorial for new users. &
    Yes & This was already planned and will be implemented for revision 1. \\
    Simplify hover descriptions and provide examples or links to
    documentation. & Yes & The hover content will be improved for revision 1. \\
    Explain how refactoring saves energy, possibly with
    visualizations. & Partial & No visualizations will be added, but
    better explanation of smells will be provided. \\
    \bottomrule
  \end{tabular}
  \caption{Participant Feedback and Implementation Decisions}
\end{table}

\subsection{Performance}

This testing benchmarks the performance of ecooptimizer across
files of varying sizes (250, 1000, and 3000 lines). The data includes
detection times,
refactoring times for specific smells, and energy measurement times.
The goal is to
identify scalability patterns, performance bottlenecks, and
opportunities for optimization.\\

\textbf{Related Performance Requirement:} PR-1\\

\noindent The test cases for this module can be found
\href{https://github.com/ssm-lab/capstone--source-code-optimizer/blob/new-poc/tests/benchmarking/benchmark.py}{here}\\

This script benchmarks the following components:

\begin{enumerate}
  \item \textbf{Detection/Analyzer Runtime} (via
    \texttt{AnalyzerController.run\_analysis})
  \item \textbf{Refactoring Runtime} (via
    \texttt{RefactorerController.run\_refactorer})
  \item \textbf{Energy Measurement Time} (via
    \texttt{CodeCarbonEnergyMeter.measure\_energy})
\end{enumerate}

For each detected smell (grouped by smell type), refactoring is run
10 times to compute average times.\\

\noindent The following is for your reference: \\

\begin{tabular}{|l|l|l|}
  \hline
  \textbf{Type of Smell} & \textbf{Code} & \textbf{Smell Name} \\
  \hline
  Pylint & R0913 & Long Parameter List \\
  Pylint & R6301 & No Self Use \\
  Pylint & R1729 & Use a Generator \\
  \hline
  Custom & LMC001 & Long Message Chain \\
  Custom & UVA001 & Unused Variable or Attribute \\
  Custom & LEC001 & Long Element Chain \\
  Custom & LLE001 & Long Lambda Expression \\
  Custom & SCL001 & String Concatenation in Loop \\
  Custom & CRC001 & Cache Repeated Calls \\
  \hline
\end{tabular}

\subsection*{1. Detection Time vs File Size}
\begin{figure}[H]
  \centering
  \includegraphics[width=\textwidth]{{../Images/detectionTimeVsFileSize.png}}
  \caption{Detection Time vs File Size}
\end{figure}

\noindent \textbf{What}: Linear plot showing code smell detection
time growth with file size\\

\noindent \textbf{Why}: Understand scalability of detection mechanism\\

The detection time grows non-linearly with file size, suggesting a
potential \(O(n^2)\) complexity.
For a 250-line file, detection takes 0.38 seconds, while a 1000-line
file takes 0.90 seconds (a 2.4×
increase). At 3000 lines, the detection time jumps to 2.58 seconds (a
2.9× increase from 1000 lines).
This indicates that the detection algorithm scales poorly for larger
files, which could become
problematic for very large codebases. However, the absolute times
remain reasonable, with detection
completing in under 3 seconds even for 3000-line files making this
not a current critical bottleneck.

\subsection*{2. Refactoring Times by Smell Type (Log Scale)}
\begin{figure}[H]
  \centering
  \includegraphics[width=\textwidth]{../Images/refactoring\_times\_log\_scale.png}
  \caption{Refactoring Times by Smell Type (Log Scale)}
\end{figure}

\noindent \textbf{What}: Logarithmic plot of refactoring times per
smell across file sizes\\

\noindent \textbf{Why}: Identify most expensive refactorings and
scalability patterns\\

The logarithmic plot reveals a clear hierarchy of refactoring costs.
The most expensive smells are \texttt{R6301} and \texttt{R0913},
which take 6.13 seconds and 5.65 seconds, respectively, for a
3000-line file. These smells show exponential growth, with
\texttt{R6301} increasing by 14.6× from 250 to 3000 lines. In
contrast, low-cost smells like \texttt{LLE001} and \texttt{LMC001}
remain consistently fast (0.03 seconds) across all file sizes. This
suggests that optimizing \texttt{R6301} and \texttt{R0913} should be
a priority, as they dominate the refactoring time for larger files.

\subsection*{3. Refactoring Times Heatmap}
\begin{figure}[H]
  \centering
  \includegraphics[width=\textwidth]{../Images/refactoring\_times\_heatmap.png}
  \caption{Refactoring Times Heatmap}
\end{figure}

\noindent \textbf{What}: Color-coded matrix of refactoring times by
smell/file size\\

\noindent \textbf{Why}: Quick visual identification of hot spots\\

The heatmap provides a quick visual summary of refactoring times
across smells and file sizes. The darkest cells correspond to
\texttt{R6301} and \texttt{R0913} at 3000 lines, confirming their
status as the most expensive operations. In contrast, \texttt{LLE001}
and \texttt{LMC001} remain light-colored across all sizes, indicating
consistently low costs. The heatmap also highlights the dramatic
variation in refactoring times: at 3000 lines, the fastest smell
(\texttt{LLE001}) is 200× faster than the slowest (\texttt{R6301}).

\subsection*{4. Energy Measurement Times Distribution}
\begin{figure}[H]
  \centering
  \includegraphics[width=\textwidth]{../Images/energy\_measurement\_boxplot.png}
  \caption{Energy Measurement Times Distribution}
\end{figure}

\noindent \textbf{What}: Box plot of energy measurement durations\\

\noindent \textbf{Why}: Verify measurement consistency across operations\\

Energy measurement times are remarkably consistent, ranging from 5.54
to 6.14 seconds across all operations and file sizes.
The box plot shows no significant variation with file size,
suggesting that energy measurement is operation-specific
rather than dependent on the size of the file. This stability could
indicate that the energy measurement process has a fixed
overhead, which could simplify efforts in the future if we were to
create our own energy measurement module.

\subsection*{5. Comparative Refactoring Times per File Size}
\begin{figure}[H]
  \centering
  \includegraphics[width=\textwidth]{../Images/refactoring\_times\_comparison.png}
  \caption{Comparative Refactoring Times per File Size}
\end{figure}

\noindent \textbf{What}: Side-by-side bar charts per file size\\

\noindent \textbf{Why}: Direct comparison of refactoring costs at
different scales\\

The side-by-side bar charts reveal consistent dominance patterns
across file sizes. \texttt{R6301} and \texttt{R0913} are always the
top two most expensive smells, while \texttt{LLE001} and
\texttt{LMC001} remain the cheapest. Notably, the relative cost
difference between the most and least expensive smells increases with
file size: at 250 lines, the ratio is 100:1, but at 3000 lines, it
grows to 200:1. This suggests that the scalability of refactoring
operations varies significantly by smell type.

\subsection*{6. Energy vs Refactoring Time Correlation}

\begin{figure}[H]
  \centering
  \includegraphics[width=\textwidth]{../Images/energy\_refactoring\_correlation.png}
  \caption{Energy vs Refactoring Time Correlation}
\end{figure}

\noindent \textbf{What}: Scatter plot comparing refactoring and
energy measurement times\\

\noindent \textbf{Why}: Identify potential relationships between
effort and energy impact\\

The scatter plot shows no clear correlation between refactoring time
and energy measurement time. Fast refactorings like \texttt{LLE001}
and slow refactorings like \texttt{R6301} both result in energy
measurement times clustered between 5.5 and 6.1 seconds. This makes
perfect sense as the refactoring operations and energy measurement
are disjoint functionalities in the code.

\subsection*{Key Insights and Recommendations}
\begin{itemize}
  \item \textbf{Bottleneck Identification:} The smells \texttt{R6301}
    and \texttt{R0913} are the primary bottlenecks, consuming over
    50\% of the total refactoring time for 3000-line files.
    Optimizing these operations should be a top priority.
  \item \textbf{Scalability Concerns:} Both detection and refactoring
    times scale poorly with file size, suggesting \(O(n^2)\)
    complexity. This could become problematic for very large codebases.
  \item \textbf{Low-Hanging Fruit:} Smells like \texttt{LLE001} and
    \texttt{LMC001} are consistently fast to refactor, making them
    ideal candidates for early refactoring efforts.
  \item \textbf{Energy Measurement Stability:} Energy measurement
    times seem consistent across operations and file sizes,
    indicating a fixed overhead. This simplifies efforts to correlate
    refactoring with energy savings.
  \item \textbf{Disproportionate Costs:} The cost difference between
    the most and least expensive smells grows with file size,
    highlighting the need for targeted optimization.
\end{itemize}

The analysis reveals significant scalability challenges for both
detection and refactoring, particularly for smells like
\texttt{R6301} and \texttt{R0913}. While energy measurement times are
stable, their lack of correlation with refactoring time suggests that
additional metrics may be needed to accurately assess energy savings.
Future work should focus on optimizing high-cost operations and
improving the scalability of the detection algorithm.

\subsection{Maintainability and Support}
\begin{enumerate}

  \item \textbf{test-MS-1: Extensibility for New Code Smells and
    Refactorings} \\[2mm]
    To validate the extensibility of our tool, we structured the
    codebase using a modular design, where new code smell detection
    and refactoring functions can
    be easily added as separate components. In simpler terms, each
    refactoring and each custom detection is placed in its own file.
    A code walkthrough confirmed that
    existing modules remain unaffected when adding new detection
    logic. We successfully integrated a sample code smell and its
    corresponding refactoring method with
    minimal changes, ensuring that the new function was accessible
    through the main interface. This demonstrated that our
    architecture supports future expansions without
    disrupting core functionality.

  \item \textbf{test-MS-2: Maintainable and Adaptable Codebase} \\[2mm]
    We conducted a static analysis and documentation walkthrough to
    assess the maintainability of our codebase. The code was reviewed
    for modular organization, clear separation
    of concerns, and adherence to coding standards. All modules were
    correctly seperated and organzied. Documentation will be updated
    to include detailed descriptions of functions
    and configuration files, ensuring clarity for future developers.
    Code comments were refined to enhance readability, and function
    naming conventions were standardized for
    consistency. These efforts ensured that the tool remains
    adaptable to new Python versions and evolving best practices.

  \item \textbf{test-MS-3: Easy rollback of updates in case of errors} \\[2mm]
    Once releases are made, each release will be properly tagged and
    versioned to ensure smooth rollbacks through version control.
    This will all be handled with Git. done, but
    this approach guarantees that users will be able to revert to a
    previous stable version if needed, maintaining system integrity
    and minimizing disruptions.
\end{enumerate}

\subsection{Look and Feel}
\begin{enumerate}
  \item \textbf{test-LF-1 Side-by-Side Code Comparison in IDE Plugin} \\[2mm]
    The side-by-side code comparison feature in the IDE plugin was
    tested manually to verify that users can clearly view the
    original and refactored code within the VS Code interface. The
    test followed the procedure outlined in Test-LF-1, which
    specifies that upon initiating a refactoring operation, the
    plugin should display the original and modified versions of the
    code in parallel, allowing users to compare changes effectively.

    The tester performed the test dynamically by opening a sample
    code file within the plugin and applying various refactoring
    operations across all detected code smells. The expected result
    was that the IDE plugin would correctly display the two versions
    side by side, with clear options for users to accept or reject
    each change. The actual result confirmed that the functionality
    operates as expected: refactored code was displayed adjacent to
    the original code, ensuring an intuitive comparison process. The
    tester was also able to interact with the accept/reject buttons,
    verifying their usability and correctness.

    A screenshot of the successful test execution is provided in
    Figure \ref{fig:lf1_test}, illustrating the side-by-side code
    comparison functionality within the IDE plugin.

    \FloatBarrier
    \begin{figure}[h]
      \centering
      \includegraphics[width=0.8\linewidth]{../Images/test-LF-1-image.png}
      \caption{Side-by-Side Code Comparison in VS Code Plugin}
      \label{fig:lf1_test}
    \end{figure}
    \FloatBarrier

  \item \textbf{test-LF-2 Theme Adaptation in VS Code} \\[2mm]
    The theme adaptation feature in the IDE plugin was tested
    manually to confirm that the plugin correctly adjusts to VS
    Code’s light and dark themes without requiring manual
    configuration. The tester performed the test by opening the
    plugin in both themes and switching between them using VS Code’s settings.

    The expected result was that the plugin’s interface should
    automatically adjust when the theme is changed. The actual result
    confirmed that the plugin seamlessly transitioned between light
    and dark themes while maintaining a consistent interface. The
    images in Figures \ref{fig:lf2_light} and \ref{fig:lf2_dark}
    illustrate the side-by-side refactoring panel in both light mode
    and dark mode.

    \FloatBarrier
    \begin{figure}[h]
      \centering
      \includegraphics[width=0.8\linewidth]{../Images/test-LF-2-image-light.png}
      \caption{Side-by-Side Refactoring Panel in Light Mode}
      \label{fig:lf2_light}
    \end{figure}
    \FloatBarrier

    \FloatBarrier
    \begin{figure}[h]
      \centering
      \includegraphics[width=0.8\linewidth]{../Images/test-LF-2-image-dark.png}
      \caption{Side-by-Side Refactoring Panel in Dark Mode}
      \label{fig:lf2_dark}
    \end{figure}
    \FloatBarrier

  \item \textbf{test-LF-3 Design Acceptance} \\[2mm]
    The design acceptance test was conducted as part of the usability
    testing session, where developers and testers interacted with the
    plugin and provided feedback. This test evaluated user
    experience, ease of navigation, and overall satisfaction with the
    plugin’s interface.

    The expected result was that users would be able to interact with
    the plugin smoothly and provide structured feedback. The actual
    result confirmed that users were able to navigate and use the
    plugin effectively. The feedback collected during this session
    was used to assess the overall usability of the plugin. More
    details regarding this evaluation can be found in the Usability
    Testing section.

\end{enumerate}

\subsection{Operational \& Environmental}

\textbf{test-OPE-1} will be tested once the extension is officially
launched.\\[2mm]

\noindent
\textbf{test-OPE-2} tests a feature that is yet to be implemented. \\[2mm]

\noindent
\textbf{test-OPE-3} will be tested once the python package is published.

\subsection{Security}

\textbf{test-SRT-1: Audit Logs for Refactoring Processes} \\[2mm]
We conducted a combination of code walkthroughs and static analysis
of logging mechanisms to validate that the tool maintains a secure
log of all refactoring processes, including pattern analysis, energy
analysis, and report generation. The objective was to do so while
covering the logging mechanisms for refactoring events, ensuring that
logs are complete and immutable.\\

\noindent The development team reviewed the codebase to confirm that
each refactoring event (pattern analysis, energy analysis, report
generation) is logged with accurate timestamps and event description.
Missing log entries and/or insufficient details were identified and
added to the logging process.\\

\noindent Through this process, all major refactoring processes were
correctly logged with accurate timestamps. Logs are stored locally on
the user's device, ensuring that unauthorized modifications are
prevented by restricting external access.\\

\noindent The team was able to confirm that the tool maintains a
secure logging system for refactoring processes, with logs being
tamper-resistant due to their local storage on user devices.

\subsection{Compliance}
\begin{enumerate}

  \item \textbf{test-CPL-1: Compliance with PIPEDA and CASL} \\[2mm]
    This process was applied to all processes related to data
    handling and user communication within the local API framework
    with the objective of assesing whether the tool’s data handling
    and communication mechanisms align with PIPEDA and CASL
    requirements, ensuring that no personal information is stored,
    all processing is local, and communication practices meet
    regulatory standards.\\
    \noindent Through code review, the team confirmed that all data
    processing remains local and does not involve external storage.
    During this time, internal API functionality was also reviewed to
    ensure that user interactions are transient and not logged
    externally. By going through the different workflows, the team
    verified that no personal data collection occurs, eliminating the
    need for explicit consent mechanisms.\\
    \noindent As a result of this process, it was concluded that the
    tool does not store any user data. The tool also does not send
    unsolicited communications, aligning with CASL requirements.

  \item \textbf{test-CPL-2: Compliance with ISO 9001 and SSADM
    Standards} \\[2mm]
    This process evaluated development workflows, documentation
    practices, and adherence to structured methodologies with the
    object of assessing whether the tool’s quality management and
    software development processes align with ISO 9001 standards for
    quality management and SSADM for structured software development.\\
    \noindent Through an unbiased approach, the team verified the
    presence of structured documentation, feedback mechanisms, and
    version tracking. It was also confirmed that a combination of
    unit testing, informal testing and iteration processes were
    applied during development. After code review, adherence to
    structured programming and modular design principles was also confirmed.
    \noindent Our goal was to take a third perspective check on
    whether these set of practices were applied to our development
    workflows. Development follows reasonable structured processes
    and also includes formal documentation of testing and quality
    assurance procedures. Version control system is present including
    change tracking and basic project management.

\end{enumerate}

\section{Comparison to Existing Implementation}

Not applicable.

\section{Unit Testing}

The following section outlines the unit tests created for the python
backend modules and the vscode extension.

\newcounter{testcase}
\newcommand{\testcount}{\stepcounter{testcase}\thetestcase}
\renewcommand{\arraystretch}{1.2} % Adjust row height for better readability

\subsection{API Endpoints}

\subsubsection{Smell Detection Endpoint}

\begin{longtable}{c
    >{\raggedright\arraybackslash}p{1.5cm}
    >{\raggedright\arraybackslash}p{4.5cm}
    >{\raggedright\arraybackslash}p{4cm}
  >{\raggedright\arraybackslash}p{3cm} c}
  \toprule
  \textbf{ID} & \textbf{Ref. Req.} & \textbf{Action} &
  \textbf{Expected Result} & \textbf{Actual Result} & \textbf{Result} \\
  \midrule
  \endfirsthead

  \multicolumn{6}{l}{\textit{(Continued from previous page)}} \\
  \toprule
  \textbf{ID} & \textbf{Ref. Req.} & \textbf{Action} &
  \textbf{Expected Result} & \textbf{Actual Result} & \textbf{Result} \\
  \midrule
  \endhead

  \multicolumn{6}{r}{\textit{Continued on next page}} \\
  \endfoot

  \bottomrule
  \caption{Smell Detection Endpoint Test Cases}
  \label{table:detection_endpoint_tests}
  \endlastfoot

  TC\testcount & FR10, OER-IAS1 & User requests to detect smells in a
  valid file. & Status code is 200. Response contains 2 smells. & All
  assertions pass. & \cellcolor{green} Pass \\ \midrule
  TC\testcount & FR10, OER-IAS1 & User requests to detect smells in a
  non-existent file. & Status code is 404. Error message indicates
  file not found. & All assertions pass. & \cellcolor{green} Pass \\ \midrule
  TC\testcount & FR10, OER-IAS1 & Internal server error occurs during
  smell detection. & Status code is 500. Error message indicates
  internal server error. & All assertions pass. & \cellcolor{green} Pass \\
\end{longtable}

\subsubsection{Refactor Endpoint}

\begin{longtable}{c
    >{\raggedright\arraybackslash}p{1.5cm}
    >{\raggedright\arraybackslash}p{4.5cm}
    >{\raggedright\arraybackslash}p{4cm}
  >{\raggedright\arraybackslash}p{3cm} c}
  \toprule
  \textbf{ID} & \textbf{Ref. Req.} & \textbf{Action} &
  \textbf{Expected Result} & \textbf{Actual Result} & \textbf{Result} \\
  \midrule
  \endfirsthead

  \multicolumn{6}{l}{\textit{(Continued from previous page)}} \\
  \toprule
  \textbf{ID} & \textbf{Ref. Req.} & \textbf{Action} &
  \textbf{Expected Result} & \textbf{Actual Result} & \textbf{Result} \\
  \midrule
  \endhead

  \multicolumn{6}{r}{\textit{Continued on next page}} \\
  \endfoot

  \bottomrule
  \caption{Refactor Endpoint Test Cases}
  \label{table:refactor_endpoint_tests}
  \endlastfoot

  TC\testcount & FR10, OER-IAS1 & User requests to refactor a valid
  source directory. & Status code is 200. Response contains
  refactored data and updated smells. & All assertions pass. &
  \cellcolor{green} Pass \\ \midrule
  TC\testcount & FR10, OER-IAS1 & User requests to refactor a
  non-existent source directory. & Status code is 404. Error message
  indicates directory not found. & All assertions pass. &
  \cellcolor{green} Pass \\ \midrule
  TC\testcount & FR10, OER-IAS1 & Energy is not saved after
  refactoring. & Status code is 400. Error message indicates energy
  was not saved. & All assertions pass. & \cellcolor{green} Pass \\ \midrule
  TC\testcount & FR10, OER-IAS1 & Initial energy measurement fails. &
  Status code is 400. Error message indicates initial emissions could
  not be retrieved. & All assertions pass. & \cellcolor{green} Pass \\ \midrule
  TC\testcount & FR10, OER-IAS1 & Final energy measurement fails. &
  Status code is 400. Error message indicates final emissions could
  not be retrieved. & All assertions pass. & \cellcolor{green} Pass \\ \midrule
  TC\testcount & FR10, OER-IAS1 & Unexpected error occurs during
  refactoring. & Status code is 400. Error message contains the
  exception details. & All assertions pass. & \cellcolor{green} Pass \\
\end{longtable}

\subsection{Analyzer Controller Module}

\begin{longtable}{c
    >{\raggedright\arraybackslash}p{1.5cm}
    >{\raggedright\arraybackslash}p{4.5cm}
    >{\raggedright\arraybackslash}p{4cm}
  >{\raggedright\arraybackslash}p{3cm} c}
  \toprule
  \textbf{ID} & \textbf{Ref. Req.} & \textbf{Action} &
  \textbf{Expected Result} & \textbf{Actual Result} & \textbf{Result} \\
  \midrule
  \endfirsthead

  \multicolumn{6}{l}{\textit{(Continued from previous page)}} \\
  \toprule
  \textbf{ID} & \textbf{Ref. Req.} & \textbf{Action} &
  \textbf{Expected Result} & \textbf{Actual Result} & \textbf{Result} \\
  \midrule
  \endhead

  \multicolumn{6}{r}{\textit{Continued on next page}} \\
  \endfoot

  \bottomrule
  \caption{Analyzer Controller Module Test Cases}
  \label{table:analyzer_controller_tests}
  \endlastfoot

  TC\testcount & FR2, FR5, PR-PAR3 & Test detection of repeated
  function calls in AST-based analysis. & One repeated function call
  should be detected. & All assertions pass. & \cellcolor{green} Pass
  \\ \midrule
  TC\testcount & FR2, FR5, PR-PAR3 & Test detection of repeated
  method calls on the same object instance. & One repeated method
  call should be detected. & All assertions pass. & \cellcolor{green}
  Pass \\ \midrule
  TC\testcount & FR2 & Test that no code smells are detected in a
  clean file. & The system should return an empty list of smells. &
  All assertions pass. & \cellcolor{green} Pass \\ \midrule
  TC\testcount & FR2, PR-PAR2 & Test filtering of smells by analysis
  method. & The function should return only smells matching the
  specified method (AST, Pylint, Astroid). & All assertions pass. &
  \cellcolor{green} Pass \\ \midrule
  TC\testcount & FR2, PR-PAR2 & Test generating custom analysis
  options for AST-based analysis. & The generated options should
  include callable detection functions. & All assertions pass. &
  \cellcolor{green} Pass \\ \midrule
  TC\testcount & FR2, FR5, PR-PAR3 & Test correct logging of detected
  code smells. & Detected smells should be logged with correct
  details. & All assertions pass. & \cellcolor{green} Pass \\ \midrule
  TC\testcount & FR2, FR5 & Test handling of an empty registry when
  filtering smells. & The function should return an empty dictionary.
  & All assertions pass. & \cellcolor{green} Pass \\ \midrule
  TC\testcount & FR2, PR-PAR2 & Test that smells remain unchanged if
  no modifications occur. & The function should not modify existing
  smells if no changes are detected. & All assertions pass. &
  \cellcolor{green} Pass \\
\end{longtable}

\subsection{CodeCarbon Measurement}

\begin{longtable}{c
    >{\raggedright\arraybackslash}p{1.5cm}
    >{\raggedright\arraybackslash}p{4.5cm}
    >{\raggedright\arraybackslash}p{4cm}
  >{\raggedright\arraybackslash}p{3cm} c}
  \toprule
  \textbf{ID} & \textbf{Ref. Req.} & \textbf{Action} &
  \textbf{Expected Result} & \textbf{Actual Result} & \textbf{Result} \\
  \midrule
  \endfirsthead

  \multicolumn{6}{l}{\textit{(Continued from previous page)}} \\
  \toprule
  \textbf{ID} & \textbf{Ref. Req.} & \textbf{Action} &
  \textbf{Expected Result} & \textbf{Actual Result} & \textbf{Result} \\
  \midrule
  \endhead

  \multicolumn{6}{r}{\textit{Continued on next page}} \\
  \endfoot

  \bottomrule
  \caption{CodeCarbon Measurement Test Cases}
  \label{table:measurement_tests}
  \endlastfoot

  TC\testcount & PR-RFT1, FR6 & Trigger CodeCarbon measurements with
  a valid file path. & CodeCarbon subprocess for the file should be
  invoked at least once. \texttt{EmissionsTracker. start} and
  \texttt{stop} API endpoints should be called. Success message
  ``CodeCarbon measurement completed successfully.'' should be
  logged. & All assertions pass. & \cellcolor{green} Pass \\
  \midrule
  TC\testcount & PR-RFT1 & Trigger CodeCarbon function with a valid
  file path that causes a subprocess failure. & CodeCarbon subprocess
  run should still be invoked. \texttt{EmissionsTracker. start} and
  \texttt{stop} API endpoints should be called. An error message
  ``Error executing file'' should be logged. Returned emissions data
  should be \texttt{None} since the execution failed.& All assertions
  pass. & \cellcolor{green} Pass \\
  \midrule
  TC\testcount & FR5, PR-SCR1 & Results produced by CodeCarbon run
  are at a valid CSV file path and can be read. & Emissions data
  should be read successfully from the CSV file. The function should
  return the last row of emissions data. & All assertions pass. &
  \cellcolor{green} Pass \\
  \midrule
  TC\testcount & PR-RFT1, FR6 & Results produced by CodeCarbon run
  are at a valid CSV file path but the file cannot be read. & An
  error message ``Error reading file'' should be logged. The function
  should return \texttt{None} because the file reading failed. & All
  assertions pass. & \cellcolor{green} Pass \\
  \midrule
  TC\testcount & PR-RFT1, FR5 & Given CSV Path for results produced
  by CodeCarbon does not have a file. & An error message ``File
  \texttt{file path} does not exist.'' should be logged.The function
  should return \texttt{None} since the file does not exist. & All
  assertions pass. & \cellcolor{green} Pass \\
\end{longtable}

\subsection{Smell Analyzers}

\subsubsection{String Concatenation in Loop}

\begin{longtable}{c
    >{\raggedright\arraybackslash}p{1.5cm}
    >{\raggedright\arraybackslash}p{4.5cm}
    >{\raggedright\arraybackslash}p{4cm}
  >{\raggedright\arraybackslash}p{3cm} c}
  \toprule
  \textbf{ID} & \textbf{Ref. Req.} & \textbf{Action} &
  \textbf{Expected Result} & \textbf{Actual Result} & \textbf{Result} \\
  \midrule
  \endfirsthead

  \multicolumn{6}{l}{\textit{(Continued from previous page)}} \\
  \toprule
  \textbf{ID} & \textbf{Ref. Req.} & \textbf{Action} &
  \textbf{Expected Result} & \textbf{Actual Result} & \textbf{Result} \\
  \midrule
  \endhead

  \multicolumn{6}{r}{\textit{Continued on next page}} \\
  \endfoot

  \bottomrule
  \caption{String Concatenation in Loop Detection Test Cases}
  \label{table:string_concat_detection_tests}
  \endlastfoot

  TC\testcount & FR2 & Detects \texttt{+=} string concatenation
  inside a \texttt{for} loop. & One smell detected with target
  \texttt{result} and line 4. & All assertions pass. & \cellcolor{green} Pass \\
  \midrule
  TC\testcount & FR2 & Detects \texttt{<var = var + ...>} string
  concatenation inside a loop. & One smell detected with target
  \texttt{result} and line 4. & All assertions pass. & \cellcolor{green} Pass \\
  \midrule
  TC\testcount & FR2 & Detects \texttt{+=} string concatenation
  inside a \texttt{while} loop. & One smell detected with target
  \texttt{result} and line 4. & All assertions pass. & \cellcolor{green} Pass \\
  \midrule
  TC\testcount & FR2 & Detects \texttt{+=} modifying a list item
  inside a loop. & One smell detected with target
  \lstinline|self.text[0]| and line 6. & All assertions pass. &
  \cellcolor{green} Pass \\
  \midrule
  TC\testcount & FR2 & Detects \texttt{+=} modifying an object
  attribute inside a loop. & One smell detected with target
  \lstinline|self.text| and line 6. & All assertions pass. &
  \cellcolor{green} Pass \\
  \midrule
  TC\testcount & FR2 & Detects \texttt{+=} modifying a dictionary
  value inside a loop. & One smell detected with target
  \texttt{data['key']} and line 4. & All assertions pass. &
  \cellcolor{green} Pass \\
  \midrule
  TC\testcount & FR2 & Detects multiple separate string
  concatenations in a loop. & Two smells detected with targets
  \texttt{result} and \texttt{logs[0]} on line 5. & All assertions
  pass. & \cellcolor{green} Pass \\
  \midrule
  TC\testcount & FR2 & Detects string concatenations with
  re-assignments inside the loop. & One smell detected with target
  \texttt{result} and line 4. & All assertions pass. & \cellcolor{green} Pass \\
  \midrule
  TC\testcount & FR2 & Detects concatenation inside nested loops. &
  One smell detected with target \texttt{result} and line 5. & All
  assertions pass. & \cellcolor{green} Pass \\
  \midrule
  TC\testcount & FR2 & Detects multi-level concatenations belonging
  to the same smell. & One smell detected with target \texttt{result}
  and two occurrences on lines 4 and 5. & All assertions pass. &
  \cellcolor{green} Pass \\
  \midrule
  TC\testcount & FR2 & Detects \texttt{+=} inside an \texttt{if-else}
  condition within a loop. & One smell detected with target
  \texttt{result} and two occurrences on line 4. & All assertions
  pass. & \cellcolor{green} Pass \\
  \midrule
  TC\testcount & FR2 & Detects \texttt{+=} using f-strings inside a
  loop. & One smell detected with target \texttt{result} and line 4.
  & All assertions pass. & \cellcolor{green} Pass \\
  \midrule
  TC\testcount & FR2 & Detects \texttt{+=} using \texttt{\%}
  formatting inside a loop. & One smell detected with target
  \texttt{result} and line 4. & All assertions pass. & \cellcolor{green} Pass \\
  \midrule
  TC\testcount & FR2 & Detects \texttt{+=} using \texttt{.format()}
  inside a loop. & One smell detected with target \texttt{result} and
  line 4. & All assertions pass. & \cellcolor{green} Pass \\
  \midrule
  TC\testcount & FR2 & Ensures accessing the concatenation variable
  inside the loop is NOT flagged. & No smells detected. & All
  assertions pass. & \cellcolor{green} Pass \\
  \midrule
  TC\testcount & FR2 & Ensures regular string assignments are NOT
  flagged. & No smells detected. & All assertions pass. &
  \cellcolor{green} Pass \\
  \midrule
  TC\testcount & FR2 & Ensures number operations with \texttt{+=} are
  NOT flagged. & No smells detected. & All assertions pass. &
  \cellcolor{green} Pass \\
  \midrule
  TC\testcount & FR2 & Ensures string concatenation OUTSIDE a loop is
  NOT flagged. & No smells detected. & All assertions pass. &
  \cellcolor{green} Pass \\
  \midrule
  TC\testcount & FR2 & Detects a variable concatenated multiple times
  in the same loop iteration. & One smell detected with target
  \texttt{result} and two occurrences on line 4. & All assertions
  pass. & \cellcolor{green} Pass \\
  \midrule
  TC\testcount & FR2 & Detects concatenation where both prefix and
  suffix are added. & One smell detected with target \texttt{result}
  and line 4. & All assertions pass. & \cellcolor{green} Pass \\
  \midrule
  TC\testcount & FR2 & Detects \texttt{+=} where new values are
  inserted at the beginning instead of the end. & One smell detected
  with target \texttt{result} and line 4. & All assertions pass. &
  \cellcolor{green} Pass \\
  \midrule
  TC\testcount & FR2 & Ignores potential smells where type cannot be
  confirmed as a string. & No smells detected. & All assertions pass.
  & \cellcolor{green} Pass \\
  \midrule
  TC\testcount & FR2 & Detects string concatenation where type is
  inferred from function type hints. & One smell detected with target
  \texttt{result} and line 4. & All assertions pass. & \cellcolor{green} Pass \\
  \midrule
  TC\testcount & FR2 & Detects string concatenation where type is
  inferred from variable type hints. & One smell detected with target
  \texttt{result} and line 4. & All assertions pass. & \cellcolor{green} Pass \\
  \midrule
  TC\testcount & FR2 & Detects string concatenation where type is
  inferred from class attributes. & One smell detected with target
  \texttt{result} and line 9. & All assertions pass. & \cellcolor{green} Pass \\
  \midrule
  TC\testcount & FR2 & Detects string concatenation where type is
  inferred from the initial value assigned. & One smell detected with
  target \texttt{result} and line 4. & All assertions pass. &
  \cellcolor{green} Pass \\
\end{longtable}

\subsubsection{Long Element Chain Detector Module}

\begin{longtable}{c
    >{\raggedright\arraybackslash}p{1.5cm}
    >{\raggedright\arraybackslash}p{4.5cm}
    >{\raggedright\arraybackslash}p{4cm}
  >{\raggedright\arraybackslash}p{3cm} c}
  \toprule
  \textbf{ID} & \textbf{Ref. Req.} & \textbf{Action} &
  \textbf{Expected Result} & \textbf{Actual Result} & \textbf{Result} \\
  \midrule
  \endfirsthead

  \multicolumn{6}{l}{\textit{(Continued from previous page)}} \\
  \toprule
  \textbf{ID} & \textbf{Ref. Req.} & \textbf{Action} &
  \textbf{Expected Result} & \textbf{Actual Result} & \textbf{Result} \\
  \midrule
  \endhead

  \multicolumn{6}{r}{\textit{Continued on next page}} \\
  \endfoot

  \bottomrule
  \caption{Long Element Chain Detector Module Test Cases}
  \label{table:lec_tests}
  \endlastfoot

  TC\testcount & FR2 & Test with code that has no chains. & No chains
  should be detected. & All assertions pass. & \cellcolor{green} Pass
  \\ \midrule
  TC\testcount & FR2 & Test with chains shorter than threshold. & No
  chains should be detected for threshold of 5. & All assertions
  pass. & \cellcolor{green} Pass \\ \midrule
  TC\testcount & FR2  & Test with chains exactly at threshold. & One
  chain should be detected at line 3. & All assertions pass. &
  \cellcolor{green} Pass \\ \midrule
  TC\testcount & FR2 & Test with chains longer than threshold. & One
  chain should be detected with message ``Dictionary chain too long
  (4/3)''. & All assertions pass. & \cellcolor{green} Pass \\ \midrule
  TC\testcount & FR2 & Test with multiple chains in the same file. &
  Two chains should be detected at different lines. & All assertions
  pass. & \cellcolor{green} Pass \\ \midrule
  TC\testcount & FR2 & Test chains inside nested functions and
  classes. & Two chains should be detected, one inside a function,
  one inside a class. & All assertions pass. & \cellcolor{green} Pass
  \\ \midrule
  TC\testcount & FR2 & Test that chains on the same line are reported
  only once. & One chain should be detected at line 4. & All
  assertions pass. & \cellcolor{green} Pass \\ \midrule
  TC\testcount & FR2 & Test chains with different variable types. &
  Two chains should be detected, one in a list and one in a tuple. &
  All assertions pass. & \cellcolor{green} Pass \\ \midrule
  TC\testcount & FR2 & Test with a custom threshold value. & No
  chains detected with threshold 4. One chain detected with threshold
  2. & All assertions pass. & \cellcolor{green} Pass \\ \midrule
  TC\testcount & FR2 & Test the structure of the returned LECSmell
  object. & Object should have correct type, path, module, symbol,
  and occurrence details. & All assertions pass. & \cellcolor{green}
  Pass \\ \midrule
  TC\testcount & FR2 & Test chains within complex expressions. &
  Three chains should be detected in different contexts. & All
  assertions pass. & \cellcolor{green} Pass \\ \midrule
  TC\testcount & FR2 & Test with an empty file. & No chains should be
  detected. & All assertions pass. & \cellcolor{green} Pass \\ \midrule
  TC\testcount & FR2 & Test with threshold of 1 (every subscript
  reported). & One chain should be detected with message ``Dictionary
  chain too long (5/5)''. & All assertions pass. & \cellcolor{green} Pass \\
\end{longtable}

\subsubsection{Repeated Calls Detection Module}

\begin{longtable}{c
    >{\raggedright\arraybackslash}p{1.5cm}
    >{\raggedright\arraybackslash}p{4.5cm}
    >{\raggedright\arraybackslash}p{4cm}
  >{\raggedright\arraybackslash}p{3cm} c}
  \toprule
  \textbf{ID} & \textbf{Ref. Req.} & \textbf{Action} &
  \textbf{Expected Result} & \textbf{Actual Result} & \textbf{Result} \\
  \midrule
  \endfirsthead

  \multicolumn{6}{l}{\textit{(Continued from previous page)}} \\
  \toprule
  \textbf{ID} & \textbf{Ref. Req.} & \textbf{Action} &
  \textbf{Expected Result} & \textbf{Actual Result} & \textbf{Result} \\
  \midrule
  \endhead

  \multicolumn{6}{r}{\textit{Continued on next page}} \\
  \endfoot

  \bottomrule
  \caption{Repeated Calls Detection Module Test Cases}
  \label{table:crc_tests}
  \endlastfoot

  TC\testcount & FR2, PR-PAR2 & Test detection of repeated function
  calls within the same scope. & One repeated call detected with two
  occurrences. & All assertions pass. & \cellcolor{green} Pass \\ \midrule
  TC\testcount & FR2, PR-PAR2 & Test detection of repeated method
  calls on the same object instance. & One repeated method call
  detected with two occurrences. & All assertions pass. &
  \cellcolor{green} Pass \\ \midrule
  TC\testcount & FR2 & Test that function calls with different
  arguments are not flagged. & No repeated calls should be detected.
  & All assertions pass. & \cellcolor{green} Pass \\ \midrule
  TC\testcount & FR2 & Test that function calls on modified objects
  are not flagged. & No repeated calls should be detected due to
  object state change. & All assertions pass. & \cellcolor{green}
  Pass \\ \midrule
  TC\testcount & FR2, PR-PAR3 & Test detection of repeated external
  function calls. & One repeated function call detected with two
  occurrences. & All assertions pass. & \cellcolor{green} Pass \\ \midrule
  TC\testcount & FR2, PR-PAR3 & Test detection of repeated calls to
  expensive built-in functions. & One repeated function call detected
  with two occurrences. & All assertions pass. & \cellcolor{green}
  Pass \\ \midrule
  TC\testcount & FR2, PR-PAR3 & Test that built-in functions with
  primitive arguments are not flagged. & No repeated calls should be
  detected. & All assertions pass. & \cellcolor{green} Pass \\ \midrule
  TC\testcount & FR2 & Test that method calls on different object
  instances are not flagged. & No repeated calls should be detected.
  & All assertions pass. & \cellcolor{green} Pass \\
\end{longtable}

\subsubsection{Long Lambda Element Detection Module}

\begin{longtable}{c
    >{\raggedright\arraybackslash}p{1.5cm}
    >{\raggedright\arraybackslash}p{4.5cm}
    >{\raggedright\arraybackslash}p{4cm}
  >{\raggedright\arraybackslash}p{3cm} c}
  \toprule
  \textbf{ID} & \textbf{Ref. Req.} & \textbf{Action} &
  \textbf{Expected Result} & \textbf{Actual Result} & \textbf{Result} \\
  \midrule
  \endfirsthead

  \multicolumn{6}{l}{\textit{(Continued from previous page)}} \\
  \toprule
  \textbf{ID} & \textbf{Ref. Req.} & \textbf{Action} &
  \textbf{Expected Result} & \textbf{Actual Result} & \textbf{Result} \\
  \midrule
  \endhead

  \multicolumn{6}{r}{\textit{Continued on next page}} \\
  \endfoot

  \bottomrule
  \caption{Long Lambda Element Detector Module Test Cases}
  \label{table:lle_tests}
  \endlastfoot

  TC\testcount & FR2 & Test code with no lambdas. & No smells should
  be detected. & All assertions pass. & \cellcolor{green} Pass \\ \midrule
  TC\testcount & FR2 & Test short single lambda (under thresholds). &
  No smells should be detected. & All assertions pass. &
  \cellcolor{green} Pass \\ \midrule
  TC\testcount & FR2 & Test lambda exceeding expression count
  threshold. & One smell should be detected. & All assertions pass. &
  \cellcolor{green} Pass \\ \midrule
  TC\testcount & FR2 & Test lambda exceeding character length
  threshold (100). & One smell should be detected. & All assertions
  pass. & \cellcolor{green} Pass \\ \midrule
  TC\testcount & FR2 & Test lambda exceeding both expression and
  length thresholds. & At least one smell should be detected. & All
  assertions pass. & \cellcolor{green} Pass \\ \midrule
  TC\testcount & FR2 & Test nested lambdas. & Two smells should be
  detected. & All assertions pass. & \cellcolor{green} Pass \\ \midrule
  TC\testcount & FR2 & Test inline lambdas passed to functions. & Two
  smells should be detected. & All assertions pass. &
  \cellcolor{green} Pass \\ \midrule
  TC\testcount & FR2 & Test trivial lambda with no body. & No smells
  should be detected. & All assertions pass. & \cellcolor{green} Pass \\
\end{longtable}

\subsubsection{Long Message Chain Detector Module}

\begin{longtable}{c
    >{\raggedright\arraybackslash}p{1.5cm}
    >{\raggedright\arraybackslash}p{4.5cm}
    >{\raggedright\arraybackslash}p{4cm}
  >{\raggedright\arraybackslash}p{3cm} c}
  \toprule
  \textbf{ID} & \textbf{Ref. Req.} & \textbf{Action} &
  \textbf{Expected Result} & \textbf{Actual Result} & \textbf{Result} \\
  \midrule
  \endfirsthead

  \multicolumn{6}{l}{\textit{(Continued from previous page)}} \\
  \toprule
  \textbf{ID} & \textbf{Ref. Req.} & \textbf{Action} &
  \textbf{Expected Result} & \textbf{Actual Result} & \textbf{Result} \\
  \midrule
  \endhead

  \multicolumn{6}{r}{\textit{Continued on next page}} \\
  \endfoot

  \bottomrule
  \caption{Long Message Chain Detector Module Test Cases}
  \label{table:lmc_tests}
  \endlastfoot

  TC\testcount & FR2 & Test chain with exactly five method calls. &
  One smell should be detected. & All assertions pass. &
  \cellcolor{green} Pass \\ \midrule
  TC\testcount & FR2 & Test chain with six method calls. & One smell
  should be detected. & All assertions pass. & \cellcolor{green} Pass
  \\ \midrule
  TC\testcount & FR2 & Test chain with four method calls. & No smells
  should be detected. & All assertions pass. & \cellcolor{green} Pass
  \\ \midrule
  TC\testcount & FR2 & Test chain with both attribute and method
  calls. & One smell should be detected. & All assertions pass. &
  \cellcolor{green} Pass \\ \midrule
  TC\testcount & FR2 & Test chain inside a loop. & One smell should
  be detected. & All assertions pass. & \cellcolor{green} Pass \\ \midrule
  TC\testcount & FR2 & Test multiple chains on the same line. & One
  smell should be detected. & All assertions pass. &
  \cellcolor{green} Pass \\ \midrule
  TC\testcount & FR2 & Test separate statements with fewer calls. &
  No smells should be detected. & All assertions pass. &
  \cellcolor{green} Pass \\ \midrule
  TC\testcount & FR2 & Test short chain in a comprehension. & No
  smells should be detected. & All assertions pass. &
  \cellcolor{green} Pass \\ \midrule
  TC\testcount & FR2 & Test long chain in a comprehension. & One
  smell should be detected. & All assertions pass. &
  \cellcolor{green} Pass \\ \midrule
  TC\testcount & FR2 & Test five separate long chains in one
  function. & Five smells should be detected. & All assertions pass.
  & \cellcolor{green} Pass \\ \midrule
  TC\testcount & FR2 & Test chain with attribute and index lookups
  (no calls). & No smells should be detected. & All assertions pass.
  & \cellcolor{green} Pass \\ \midrule
  TC\testcount & FR2 & Test chain with slicing. & One smell should be
  detected. & All assertions pass. & \cellcolor{green} Pass \\ \midrule
  TC\testcount & FR2 & Test multiline chain. & One smell should be
  detected. & All assertions pass. & \cellcolor{green} Pass \\ \midrule
  TC\testcount & FR2 & Test chain inside a lambda. & One smell should
  be detected. & All assertions pass. & \cellcolor{green} Pass \\ \midrule
  TC\testcount & FR2 & Test chain with mixed return types. & One
  smell should be detected. & All assertions pass. &
  \cellcolor{green} Pass \\ \midrule
  TC\testcount & FR2 & Test multiple short chains on the same line. &
  No smells should be detected. & All assertions pass. &
  \cellcolor{green} Pass \\ \midrule
  TC\testcount & FR2 & Test chain inside a conditional (ternary). &
  No smells should be detected. & All assertions pass. &
  \cellcolor{green} Pass \\
\end{longtable}

\subsection{Refactorer Controller Module}

\begin{longtable}{c
    >{\raggedright\arraybackslash}p{1.5cm}
    >{\raggedright\arraybackslash}p{4.5cm}
    >{\raggedright\arraybackslash}p{4cm}
  >{\raggedright\arraybackslash}p{3cm} c}
  \toprule
  \textbf{ID} & \textbf{Ref. Req.} & \textbf{Action} &
  \textbf{Expected Result} & \textbf{Actual Result} & \textbf{Result} \\
  \midrule
  \endfirsthead

  \multicolumn{6}{l}{\textit{(Continued from previous page)}} \\
  \toprule
  \textbf{ID} & \textbf{Ref. Req.} & \textbf{Action} &
  \textbf{Expected Result} & \textbf{Actual Result} & \textbf{Result} \\
  \midrule
  \endhead

  \multicolumn{6}{r}{\textit{Continued on next page}} \\
  \endfoot

  \bottomrule
  \caption{Refactorer Controller Module Test Cases}
  \label{table:refactorer_controller_tests}
  \endlastfoot

  TC\testcount & FR5 & User requests to refactor a smell. & Correct
  smell is identified. Logger logs ``Running refactoring for
  long-element-chain using TestRefactorer.'' Correct refactorer is
  called once with correct arguments. Output path is
  \texttt{test\_path.LEC001\_1.py}. & All assertions pass. &
  \cellcolor{green} Pass \\ \midrule
  TC\testcount & UHR-UPLD1 & System handles missing refactorer. &
  Raises \texttt{NotImplementedError} with message ``No refactorer
  implemented for smell: long-element-chain.'' Logger logs error. &
  All assertions pass. & \cellcolor{green} Pass \\ \midrule
  TC\testcount & FR5 & Multiple refactorer calls are handled
  correctly. & Correct smell counter incremented. Refactorer is
  called twice. First output: \texttt{test\_path.LEC001\_1.py}.
  Second output: \texttt{test\_path.LEC001\_2.py}. & All assertions
  pass. & \cellcolor{green} Pass \\ \midrule
  TC\testcount & FR5 & Refactorer runs with overwrite set to False. &
  Refactorer is called once. Overwrite argument is set to False. &
  All assertions pass. & \cellcolor{green} Pass \\ \midrule
  TC\testcount & PR-RFT 1, FR5 & System handles empty modified files
  correctly. & Modified files list remains empty (\texttt{[]} in
  output). & All assertions pass. & \cellcolor{green} Pass \\
\end{longtable}

\subsection{Smell Refactorers}

\subsubsection{String Concatenation in Loop}

\begin{longtable}{c
    >{\raggedright\arraybackslash}p{1.5cm}
    >{\raggedright\arraybackslash}p{4.5cm}
    >{\raggedright\arraybackslash}p{4cm}
  >{\raggedright\arraybackslash}p{3cm} c}
  \toprule
  \textbf{ID} & \textbf{Ref. Req.} & \textbf{Action} &
  \textbf{Expected Result} & \textbf{Actual Result} & \textbf{Result} \\
  \midrule
  \endfirsthead

  \multicolumn{6}{l}{\textit{(Continued from previous page)}} \\
  \toprule
  \textbf{ID} & \textbf{Ref. Req.} & \textbf{Action} &
  \textbf{Expected Result} & \textbf{Actual Result} & \textbf{Result} \\
  \midrule
  \endhead

  \multicolumn{6}{r}{\textit{Continued on next page}} \\
  \endfoot

  \bottomrule
  \caption{String Concatenation in Loop Refactoring Test Cases}
  \label{table:string_concat_refactoring_tests}
  \endlastfoot

  TC\testcount & FR3, FR6 & Refactors empty initial concatenation
  variable (e.g., \lstinline|result = ""|). & Code is refactored to
  use a list and \texttt{join()}. & All assertions pass. &
  \cellcolor{green} Pass \\
  \midrule
  TC\testcount & FR3, FR6 & Refactors non-empty initial concatenation
  variable not referenced before the loop. & Code is refactored to
  use a list and \texttt{join()}. & All assertions pass. &
  \cellcolor{green} Pass \\
  \midrule
  TC\testcount & FR3, FR6 & Refactors non-empty initial concatenation
  variable referenced before the loop. & Code is refactored to use a
  list and \texttt{join()}. & All assertions pass. & \cellcolor{green} Pass \\
  \midrule
  TC\testcount & FR3, FR6 & Refactors concatenation where the target
  is not a simple variable (e.g., \texttt{result["key"]}). & Code is
  refactored to use a temporary list and \texttt{join()}. & All
  assertions pass. & \cellcolor{green} Pass \\
  \midrule
  TC\testcount & FR3, FR6 & Refactors concatenation where the
  variable is not initialized in the same scope. & Code is refactored
  to use a list and \texttt{join()}. & All assertions pass. &
  \cellcolor{green} Pass \\
  \midrule
  TC\testcount & FR3, FR6 & Refactors prefix concatenation (e.g.,
  \lstinline|result = str(i) + result|). & Code uses
  \lstinline|insert(0, ...)| for prefix concatenation. & All
  assertions pass. & \cellcolor{green} Pass \\
  \midrule
  TC\testcount & FR3, FR6 & Refactors concatenation with both prefix
  and suffix. & Code uses both \lstinline|insert(0, ...)| and
  \texttt{append(...)}. & All assertions pass. & \cellcolor{green} Pass \\
  \midrule
  TC\testcount & FR3, FR6 & Refactors multiple concatenations in the
  same loop. & Code uses \texttt{append(...)} and \texttt{insert(0,
  ...)} as needed. & All assertions pass. & \cellcolor{green} Pass \\
  \midrule
  TC\testcount & FR3, FR6 & Refactors nested concatenation in loops.
  & Code uses \texttt{append(...)} and \texttt{insert(0, ...)} for
  nested loops. & All assertions pass. & \cellcolor{green} Pass \\
  \midrule
  TC\testcount & FR3, FR6 & Refactors multiple occurrences of
  concatenation at different loop levels. & Code uses
  \texttt{append(...)} for all occurrences. & All assertions pass. &
  \cellcolor{green} Pass \\
  \midrule
  TC\testcount & FR3, FR6 & Handles reassignment of the concatenation
  variable inside the loop. & Code resets the list to the new value.
  & All assertions pass. & \cellcolor{green} Pass \\
  \midrule
  TC\testcount & FR3, FR6 & Handles reassignment of the concatenation
  variable to an empty value. & Code clears the list using
  \texttt{clear()}. & All assertions pass. & \cellcolor{green} Pass \\
  \midrule
  TC\testcount & FR3, FR6 & Ensures unrelated code and comments are
  preserved during refactoring. & Unrelated lines and comments remain
  unchanged. & All assertions pass. & \cellcolor{green} Pass \\
\end{longtable}

\subsubsection{Member Ignoring Method}

\begin{longtable}{c
    >{\raggedright\arraybackslash}p{1.5cm}
    >{\raggedright\arraybackslash}p{4.5cm}
    >{\raggedright\arraybackslash}p{4cm}
  >{\raggedright\arraybackslash}p{3cm} c}
  \toprule
  \textbf{ID} & \textbf{Ref. Req.} & \textbf{Action} &
  \textbf{Expected Result} & \textbf{Actual Result} & \textbf{Result} \\
  \midrule
  \endfirsthead

  \multicolumn{6}{l}{\textit{(Continued from previous page)}} \\
  \toprule
  \textbf{ID} & \textbf{Ref. Req.} & \textbf{Action} &
  \textbf{Expected Result} & \textbf{Actual Result} & \textbf{Result} \\
  \midrule
  \endhead

  \multicolumn{6}{r}{\textit{Continued on next page}} \\
  \endfoot

  \bottomrule
  \caption{Member Ignoring Method Refactoring Test Cases}
  \label{table:member_ignoring_method_tests}
  \endlastfoot

  TC\testcount & FR3, FR6 & Refactors a basic member-ignoring method.
  & Adds \texttt{@staticmethod}, removes \texttt{self}, and updates
  calls. & All assertions pass. & \cellcolor{green} Pass \\
  \midrule
  TC\testcount & FR3, FR6 & Refactors a member-ignoring method with
  inheritance. & Updates calls from subclass instances. & All
  assertions pass. & \cellcolor{green} Pass \\
  \midrule
  TC\testcount & FR3, FR6 & Refactors a member-ignoring method with
  subclass in a separate file. & Updates calls from subclass
  instances in external files. & All assertions pass. &
  \cellcolor{green} Pass \\
  \midrule
  TC\testcount & FR3, FR6 & Refactors a member-ignoring method with
  subclass method override. & Does not update calls to overridden
  methods. & All assertions pass. & \cellcolor{green} Pass \\
  \midrule
  TC\testcount & FR3, FR6 & Refactors a member-ignoring method with
  type hints. & Updates calls using type hints to infer instance
  type. & All assertions pass. & \cellcolor{green} Pass \\
\end{longtable}

\subsubsection{Long Element Chain Refactorer Module}

\begin{longtable}{c
    >{\raggedright\arraybackslash}p{1.5cm}
    >{\raggedright\arraybackslash}p{4.5cm}
    >{\raggedright\arraybackslash}p{4cm}
  >{\raggedright\arraybackslash}p{3cm} c}
  \toprule
  \textbf{ID} & \textbf{Ref. Req.} & \textbf{Action} &
  \textbf{Expected Result} & \textbf{Actual Result} & \textbf{Result} \\
  \midrule
  \endfirsthead

  \multicolumn{6}{l}{\textit{(Continued from previous page)}} \\
  \toprule
  \textbf{ID} & \textbf{Ref. Req.} & \textbf{Action} &
  \textbf{Expected Result} & \textbf{Actual Result} & \textbf{Result} \\
  \midrule
  \endhead

  \multicolumn{6}{r}{\textit{Continued on next page}} \\
  \endfoot

  \bottomrule
  \caption{Long Element Chain Refactorer Test Cases}
  \label{table:lec_refactorer_tests}
  \endlastfoot

  TC\testcount & PR-PAR3, FR6, FR3 & Test the long element chain
  refactorer on basic nested dictionary access & Dictionary should be
  flattened, and access updated & Refactoring applied successfully,
  dictionary access updated & \cellcolor{green} Pass \\ \midrule
  TC\testcount & PR-PAR3, FR6, FR3 & Test the long element chain
  refactorer across multiple files & Dictionary access across
  multiple files should be updated & Refactoring applied successfully
  across multiple files & \cellcolor{yellow} TBD \\ \midrule
  TC\testcount & PR-PAR3, FR6, FR3 & Test the refactorer on
  dictionary access via class attributes & Class attributes should be
  flattened and access updated & Refactoring applied successfully on
  class attribute accesses. All accesses changed correctly. &
  \cellcolor{green} Pass \\ \midrule
  TC\testcount & PR-PAR3, FR6, FR3 & Ensure the refactorer skips
  shallow dictionary access & Refactoring should be skipped for
  shallow access & Refactoring correctly skipped for shallow access &
  \cellcolor{green} Pass \\ \midrule
  TC\testcount & PR-PAR3, FR6, FR3 & Test the refactorer on
  dictionary access with mixed depths & Flatten the dictionary up to
  the minimum access depth & All dictionary access chains flattened
  to minimum access depth and dictionary flattened successfully. &
  \cellcolor{green} Pass \\
\end{longtable}

\subsubsection{Repeated Calls Refactoring Module}

\begin{longtable}{c
    >{\raggedright\arraybackslash}p{1.5cm}
    >{\raggedright\arraybackslash}p{4.5cm}
    >{\raggedright\arraybackslash}p{4cm}
  >{\raggedright\arraybackslash}p{3cm} c}
  \toprule
  \textbf{ID} & \textbf{Ref. Req.} & \textbf{Action} &
  \textbf{Expected Result} & \textbf{Actual Result} & \textbf{Result} \\
  \midrule
  \endfirsthead

  \multicolumn{6}{l}{\textit{(Continued from previous page)}} \\
  \toprule
  \textbf{ID} & \textbf{Ref. Req.} & \textbf{Action} &
  \textbf{Expected Result} & \textbf{Actual Result} & \textbf{Result} \\
  \midrule
  \endhead

  \multicolumn{6}{r}{\textit{Continued on next page}} \\
  \endfoot

  \bottomrule
  \caption{Cache Repeated Calls Refactoring Module Test Cases}
  \label{table:crc_refactor_tests}
  \endlastfoot

  TC\testcount & FR3, FR5, PR-PAR3 & Test that repeated function
  calls are cached properly. & The function calls should be replaced
  with a cached variable. & All assertions pass. & \cellcolor{green}
  Pass \\ \midrule
  TC\testcount & FR3, FR5, PR-PAR3 & Test that repeated method calls
  on the same object are cached. & Method calls should be replaced
  with a cached result stored in a variable. & All assertions pass. &
  \cellcolor{green} Pass \\ \midrule
  TC\testcount & FR3, FR5, PR-PAR2 & Test that repeated method calls
  on different object instances are not cached. & Calls on different
  object instances should remain unchanged. & All assertions pass. &
  \cellcolor{green} Pass \\ \midrule
  TC\testcount & FR3, FR5 & Test that caching is applied even with
  multiple identical function calls. & The repeated function calls
  should be replaced with a cached variable. & All assertions pass. &
  \cellcolor{green} Pass \\ \midrule
  TC\testcount & FR3, FR5 & Test caching when refactoring function
  calls that appear in a docstring. & Function calls inside the
  docstring should not be modified. & All assertions pass. &
  \cellcolor{green} Pass \\ \midrule
  TC\testcount & FR3, FR5, PR-PAR3 & Test caching of method calls
  inside a class with an unchanged instance state. & Repeated method
  calls should be cached correctly. & All assertions pass. &
  \cellcolor{green} Pass \\ \midrule
  TC\testcount & FR3, FR5 & Test that functions with varying
  arguments are not cached. & Calls with different arguments should
  remain unchanged. & All assertions pass. & \cellcolor{green} Pass \\ \midrule
  TC\testcount & FR3, FR5, PR-PAR2 & Test that caching does not
  interfere with scope and closures. & The cached value should remain
  valid within the correct scope. & All assertions pass. &
  \cellcolor{green} Pass \\
\end{longtable}

\subsubsection{Use a Generator Refactoring Module}

\begin{longtable}{c
    >{\raggedright\arraybackslash}p{1.5cm}
    >{\raggedright\arraybackslash}p{4.5cm}
    >{\raggedright\arraybackslash}p{4cm}
  >{\raggedright\arraybackslash}p{3cm} c}
  \toprule
  \textbf{ID} & \textbf{Ref. Req.} & \textbf{Action} &
  \textbf{Expected Result} & \textbf{Actual Result} & \textbf{Result} \\
  \midrule
  \endfirsthead

  \multicolumn{6}{l}{\textit{(Continued from previous page)}} \\
  \toprule
  \textbf{ID} & \textbf{Ref. Req.} & \textbf{Action} &
  \textbf{Expected Result} & \textbf{Actual Result} & \textbf{Result} \\
  \midrule
  \endhead

  \multicolumn{6}{r}{\textit{Continued on next page}} \\
  \endfoot

  \bottomrule
  \caption{Use a Generator Refactoring Module Test Cases}
  \label{table:ugen_refactor_tests}
  \endlastfoot

  TC\testcount & FR3, FR5, PR-PAR3 & Test refactoring of list
  comprehensions in `all()` calls. & The list comprehension should be
  converted into a generator expression. & All assertions pass. &
  \cellcolor{green} Pass \\ \midrule
  TC\testcount & FR3, FR5, PR-PAR3 & Test refactoring of list
  comprehensions in `any()` calls. & The list comprehension should be
  converted into a generator expression. & All assertions pass. &
  \cellcolor{green} Pass \\ \midrule
  TC\testcount & FR3, FR5, PR-PAR3 & Test refactoring of multi-line
  list comprehensions. & The multi-line comprehension should be
  refactored correctly while preserving indentation. & All assertions
  pass. & \cellcolor{green} Pass \\ \midrule
  TC\testcount & FR3, FR5, PR-PAR3 & Test refactoring of complex
  conditions within `any()` and `all()`. & The refactored generator
  expression should maintain logical correctness. & All assertions
  pass. & \cellcolor{green} Pass \\ \midrule
  TC\testcount & FR3, FR5 & Test that improperly formatted list
  comprehensions are handled correctly. & No unintended modifications
  should be applied to non-standard formats. & All assertions pass. &
  \cellcolor{green} Pass \\ \midrule
  TC\testcount & FR3, FR5 & Test that readability is preserved in
  refactored code. & The refactored code should be clear,
  well-formatted, and maintain original intent. & All assertions
  pass. & \cellcolor{green} Pass \\ \midrule
  TC\testcount & FR3, FR5 & Test that list comprehensions outside of
  `all()` and `any()` remain unchanged. & The refactorer should not
  modify list comprehensions used in other contexts. & All assertions
  pass. & \cellcolor{green} Pass \\ \midrule
  TC\testcount & FR3, FR5 & Test refactoring when `all()` or `any()`
  calls are nested. & The refactored code should handle nested
  expressions correctly. & All assertions pass. & \cellcolor{green} Pass \\
\end{longtable}

\subsubsection{Long Lambda Element Refactorer}

\begin{longtable}{c
    >{\raggedright\arraybackslash}p{1.5cm}
    >{\raggedright\arraybackslash}p{4.5cm}
    >{\raggedright\arraybackslash}p{4cm}
  >{\raggedright\arraybackslash}p{3cm} c}
  \toprule
  \textbf{ID} & \textbf{Ref. Req.} & \textbf{Action} &
  \textbf{Expected Result} & \textbf{Actual Result} & \textbf{Result} \\
  \midrule
  \endfirsthead

  \multicolumn{6}{l}{\textit{(Continued from previous page)}} \\
  \toprule
  \textbf{ID} & \textbf{Ref. Req.} & \textbf{Action} &
  \textbf{Expected Result} & \textbf{Actual Result} & \textbf{Result} \\
  \midrule
  \endhead

  \multicolumn{6}{r}{\textit{Continued on next page}} \\
  \endfoot

  \bottomrule
  \caption{Long Lambda Element Refactorer Test Cases}
  \label{table:long_lambda_refactorer_tests}
  \endlastfoot

  TC\testcount & FR1, FR2, FR3, FR5, FR6 & Refactor a basic
  single-line lambda. & Lambda is converted to a named function. &
  All assertions pass. & \cellcolor{green} Pass \\
  \midrule
  TC\testcount & FR1, FR2, FR3, FR5, FR6 & Ensure no print statements
  are added unnecessarily. & Refactored code contains no print
  statements. & All assertions pass. & \cellcolor{green} Pass \\
  \midrule
  TC\testcount & FR1, FR2, FR3, FR5, FR6 & Refactor a lambda passed
  as an argument to another function. & Lambda is converted to a
  named function and used correctly. & All assertions pass. &
  \cellcolor{green} Pass \\
  \midrule
  TC\testcount & FR1, FR2, FR3, FR5, FR6 & Refactor a lambda with
  multiple parameters. & Lambda is converted to a named function with
  multiple parameters. & All assertions pass. & \cellcolor{green} Pass \\
  \midrule
  TC\testcount & FR1, FR2, FR3, FR5, FR6 & Refactor a lambda used
  with keyword arguments. & Lambda is converted to a named function
  and used correctly with keyword arguments. & All assertions pass. &
  \cellcolor{green} Pass \\
  \midrule
  TC\testcount & FR1, FR2, FR3, FR5, FR6 & Refactor a very long
  lambda spanning multiple lines. & Lambda is converted to a named
  function preserving the logic. & All assertions pass. &
  \cellcolor{green} Pass \\
\end{longtable}

\subsubsection{Long Message Chain Refactorer}

\begin{longtable}{c
    >{\raggedright\arraybackslash}p{1.5cm}
    >{\raggedright\arraybackslash}p{4.5cm}
    >{\raggedright\arraybackslash}p{4cm}
  >{\raggedright\arraybackslash}p{3cm} c}
  \toprule
  \textbf{ID} & \textbf{Ref. Req.} & \textbf{Action} &
  \textbf{Expected Result} & \textbf{Actual Result} & \textbf{Result} \\
  \midrule
  \endfirsthead

  \multicolumn{6}{l}{\textit{(Continued from previous page)}} \\
  \toprule
  \textbf{ID} & \textbf{Ref. Req.} & \textbf{Action} &
  \textbf{Expected Result} & \textbf{Actual Result} & \textbf{Result} \\
  \midrule
  \endhead

  \multicolumn{6}{r}{\textit{Continued on next page}} \\
  \endfoot

  \bottomrule
  \caption{Long Message Chain Refactorer Test Cases}
  \label{table:long_message_chain_refactorer_tests}
  \endlastfoot

  TC\testcount & FR1, FR2, FR3, FR5, FR6 & Refactor a basic method
  chain. & Method chain is split into intermediate variables. & All
  assertions pass. & \cellcolor{green} Pass \\
  \midrule
  TC\testcount & FR1, FR2, FR3, FR5, FR6 & Refactor a long message
  chain with an f-string. & F-string chain is split into intermediate
  variables. & All assertions pass. & \cellcolor{green} Pass \\
  \midrule
  TC\testcount & FR1, FR2, FR3, FR5, FR6 & Ensure modifications occur
  even if the method chain isn't long. & Short method chain is split
  into intermediate variables. & All assertions pass. &
  \cellcolor{green} Pass \\
  \midrule
  TC\testcount & FR1, FR2, FR3, FR5, FR6 & Ensure indentation is
  preserved after refactoring. & Refactored code maintains proper
  indentation. & All assertions pass. & \cellcolor{green} Pass \\
  \midrule
  TC\testcount & FR1, FR2, FR3, FR5, FR6 & Refactor method chains
  containing method arguments. & Method chain with arguments is split
  into intermediate variables. & All assertions pass. &
  \cellcolor{green} Pass \\
  \midrule
  TC\testcount & FR1, FR2, FR3, FR5, FR6 & Refactor print statements
  with method chains. & Print statement with method chain is split
  into intermediate variables. & All assertions pass. &
  \cellcolor{green} Pass \\
  \midrule
  TC\testcount & FR1, FR2, FR3, FR5, FR6 & Refactor nested method
  chains. & Nested method chain is split into intermediate variables.
  & All assertions pass. & \cellcolor{green} Pass \\
\end{longtable}

\subsubsection{Long Parameter List}

\begin{longtable}{c
    >{\raggedright\arraybackslash}p{1.5cm}
    >{\raggedright\arraybackslash}p{4.5cm}
    >{\raggedright\arraybackslash}p{4cm}
  >{\raggedright\arraybackslash}p{3cm} c}
  \toprule
  \textbf{ID} & \textbf{Ref. Req.} & \textbf{Action} &
  \textbf{Expected Result} & \textbf{Actual Result} & \textbf{Result} \\
  \midrule
  \endfirsthead

  \multicolumn{6}{l}{\textit{(Continued from previous page)}} \\
  \toprule
  \textbf{ID} & \textbf{Ref. Req.} & \textbf{Action} &
  \textbf{Expected Result} & \textbf{Actual Result} & \textbf{Result} \\
  \midrule
  \endhead

  \multicolumn{6}{r}{\textit{Continued on next page}} \\
  \endfoot

  \bottomrule
  \caption{Long Parameter List Refactoring Test Cases}
  \label{table:long_parameter_list_tests}
  \endlastfoot

  TC\testcount & FR3, FR6 & Refactors a constructor definition with 8
  parameters, and class initialization with positional arguments. &
  Declares grouping classes. Updates constructor call with grouped
  instantiations. Also updates function signature and body to reflect
  new parameters. & All assertions pass. & \cellcolor{green} Pass \\
  \midrule
  TC\testcount & FR3, FR6 & Refactors a constructor definition with 8
  parameters with one unused in body, as well as class initialization
  with positional arguments. & Declares grouping classes. Updates
  constructor call with grouped instantiations. Also updates function
  signature and body to reflect new used parameters. & All assertions
  pass. & \cellcolor{green} Pass \\
  \midrule
  TC\testcount & FR3, FR6 & Refactors an instance method with 8
  parameters (two default values) and the call made to it (1
  positional argument). & Declares grouping classes with default
  values preserved. Updates method call with grouped instantiations.
  Also updates method signature and body to reflect new parameters.
  & All assertions pass. & \cellcolor{green} Pass \\
  \midrule
  TC\testcount & FR3, FR6 & Refactors a static method with 8
  parameters (1 with default value, 4 unused in body) and the call
  made to it (2 positional arguments)& Declares grouping classes with
  default values preserved. Updates method call with grouped
  instantiations. Also updates method signature and body to reflect
  new used parameters. & All assertions pass. & \cellcolor{green} Pass \\
  \midrule
  TC\testcount & FR3, FR6 & Refactors a standalone function with 8
  parameters (1 with default value that is also unused in body) and
  the call made to it (1 positional arguments) & Declares grouping
  classes. Updates method call with grouped instantiations. Also
  updates method signature and body to reflect new used parameters. &
  All assertions pass. & \cellcolor{green} Pass \\
\end{longtable}

\subsection{VS Code Extension}

\subsubsection{Detect Smells Command}

\begin{longtable}{c
    >{\raggedright\arraybackslash}p{1.5cm}
    >{\raggedright\arraybackslash}p{4.5cm}
    >{\raggedright\arraybackslash}p{4cm}
  >{\raggedright\arraybackslash}p{3cm} c}
  \toprule
  \textbf{ID} & \textbf{Ref. Req.} & \textbf{Action} &
  \textbf{Expected Result} & \textbf{Actual Result} & \textbf{Result} \\
  \midrule
  \endfirsthead

  \multicolumn{6}{l}{\textit{(Continued from previous page)}} \\
  \toprule
  \textbf{ID} & \textbf{Ref. Req.} & \textbf{Action} &
  \textbf{Expected Result} & \textbf{Actual Result} & \textbf{Result} \\
  \midrule
  \endhead

  \multicolumn{6}{r}{\textit{Continued on next page}} \\
  \endfoot

  \bottomrule
  \caption{Detect Smells Command Test Cases}
  \label{table:plugin_detect_command_tests}
  \endlastfoot

  TC\testcount & FR10, OER-IAS1 & No active editor is found. & Shows
  error message: ``Eco: No active editor found.'' & All assertions
  pass. & \cellcolor{green} Pass \\
  \midrule
  TC\testcount & FR10, OER-IAS1 & Active editor has no valid file
  path. & Shows error message: ``Eco: Active editor has no valid file
  path.'' & All assertions pass. & \cellcolor{green} Pass \\
  \midrule
  TC\testcount & FR10, OER-IAS1 & No smells are enabled. & Shows
  warning message: ``Eco: No smells are enabled! Detection skipped.''
  & All assertions pass. & \cellcolor{green} Pass \\
  \midrule
  TC\testcount & FR10, OER-IAS1 & Uses cached smells when hash is
  unchanged and same smells are enabled. & Shows info message: ``Eco:
  Using cached smells for fake.path'' & All assertions pass. &
  \cellcolor{green} Pass \\
  \midrule
  TC\testcount & FR10, OER-IAS1 & Fetches new smells when enabled
  smells change. & Calls \texttt{wipeWorkCache}, \texttt{updateHash},
  and \texttt{fetchSmells}. Updates workspace data. & All assertions
  pass. & \cellcolor{green} Pass \\
  \midrule
  TC\testcount & FR10, OER-IAS1 & Fetches new smells when hash
  changes but enabled smells remain the same. & Calls
  \texttt{updateHash} and \texttt{fetchSmells}. Updates workspace
  data. & All assertions pass. & \cellcolor{green} Pass \\
  \midrule
  TC\testcount & FR10, OER-IAS1 & No cached smells and server is
  down. & Shows warning message: ``Action blocked: Server is down and
  no cached smells exist for this file version.'' & All assertions
  pass. & \cellcolor{green} Pass \\
  \midrule
  TC\testcount & FR10, OER-IAS1 & Highlights smells when smells are
  found. & Shows info messages and calls \texttt{highlightSmells}. &
  All assertions pass. & \cellcolor{green} Pass \\
\end{longtable}

\subsubsection{Refactor Smell Command}

\begin{longtable}{c
    >{\raggedright\arraybackslash}p{1.5cm}
    >{\raggedright\arraybackslash}p{4.5cm}
    >{\raggedright\arraybackslash}p{4cm}
  >{\raggedright\arraybackslash}p{3cm} c}
  \toprule
  \textbf{ID} & \textbf{Ref. Req.} & \textbf{Action} &
  \textbf{Expected Result} & \textbf{Actual Result} & \textbf{Result} \\
  \midrule
  \endfirsthead

  \multicolumn{6}{l}{\textit{(Continued from previous page)}} \\
  \toprule
  \textbf{ID} & \textbf{Ref. Req.} & \textbf{Action} &
  \textbf{Expected Result} & \textbf{Actual Result} & \textbf{Result} \\
  \midrule
  \endhead

  \multicolumn{6}{r}{\textit{Continued on next page}} \\
  \endfoot

  \bottomrule
  \caption{Refactor Smell Command Test Cases}
  \label{table:plugin_refactor_command_tests}
  \endlastfoot

  TC\testcount & PR-RFT1 & No active editor is found. & Shows error
  message ``Eco: Unable to proceed as no active editor or file path
  found.'' & All assertions pass. & \cellcolor{green} Pass \\
  \midrule
  TC\testcount & PR-RFT1, FR6 & Attempting to refactor when no smells
  are detected in the file & Shows error message ``Eco: No smells
  detected in the file for refactoring.'' & All assertions pass. &
  \cellcolor{green} Pass \\
  \midrule
  TC\testcount & FR6 & Attempting to refactor when selected line
  doesn't match any smell & Shows error message ``Eco: No matching
  smell found for refactoring.'' & All assertions pass. &
  \cellcolor{green} Pass \\
  \midrule
  TC\testcount & FR5, FR6, FR10 & Refactoring a smell when found on
  the selected line & Saves the current file. Calls
  \texttt{refactorSmell} method with correct parameters. Shows
  message ``Refactoring report available in sidebar''. Executes
  command to focus refactor sidebar. Opens and shows the refactored
  preview. Highlights updated smells. Updates the UI with new smells
  & All assertions pass. & \cellcolor{green} Pass \\
  \midrule
  TC\testcount & PR-RFT2 & Handling API failure during refactoring &
  Shows error message ``Eco: Refactoring failed. See console for
  details.'' & All assertions pass. & \cellcolor{green} Pass \\
\end{longtable}

\subsubsection{File Highlighter}

\begin{longtable}{c
    >{\raggedright\arraybackslash}p{1.5cm}
    >{\raggedright\arraybackslash}p{4.5cm}
    >{\raggedright\arraybackslash}p{4cm}
  >{\raggedright\arraybackslash}p{3cm} c}
  \toprule
  \textbf{ID} & \textbf{Ref. Req.} & \textbf{Action} &
  \textbf{Expected Result} & \textbf{Actual Result} & \textbf{Result} \\
  \midrule
  \endfirsthead

  \multicolumn{6}{l}{\textit{(Continued from previous page)}} \\
  \toprule
  \textbf{ID} & \textbf{Ref. Req.} & \textbf{Action} &
  \textbf{Expected Result} & \textbf{Actual Result} & \textbf{Result} \\
  \midrule
  \endhead

  \multicolumn{6}{r}{\textit{Continued on next page}} \\
  \endfoot

  \bottomrule
  \caption{File Highlighter Test Cases}
  \label{table:plugin_file_highlighter_tests}
  \endlastfoot

  TC\testcount & FR10, OER-IAS1, LFR-AP2 & Creates decorations for a
  given color. & Decoration is created using
  \lstinline|vscode.window.createTextEditor DecorationType|. & All
  assertions pass. & \cellcolor{green} Pass \\
  \midrule
  TC\testcount & FR10, OER-IAS1, LFR-AP2 & Highlights smells in the
  active text editor. & Decorations are set using
  \texttt{setDecorations}. & All assertions pass. & \cellcolor{green} Pass \\
  \midrule
  TC\testcount & FR10, OER-IAS1, LFR-AP2 & Does not reset highlight
  decorations on first initialization. & Decorations are not disposed
  of on the first call. & All assertions pass. & \cellcolor{green} Pass \\
  \midrule
  TC\testcount & FR10, OER-IAS1, LFR-AP2 & Resets highlight
  decorations on subsequent calls. & Decorations are disposed of on
  subsequent calls. & All assertions pass. & \cellcolor{green} Pass \\
\end{longtable}

\subsubsection{File Hashing}

\begin{longtable}{c
    >{\raggedright\arraybackslash}p{1.5cm}
    >{\raggedright\arraybackslash}p{4.5cm}
    >{\raggedright\arraybackslash}p{4cm}
  >{\raggedright\arraybackslash}p{3cm} c}
  \toprule
  \textbf{ID} & \textbf{Ref. Req.} & \textbf{Action} &
  \textbf{Expected Result} & \textbf{Actual Result} & \textbf{Result} \\
  \midrule
  \endfirsthead

  \multicolumn{6}{l}{\textit{(Continued from previous page)}} \\
  \toprule
  \textbf{ID} & \textbf{Ref. Req.} & \textbf{Action} &
  \textbf{Expected Result} & \textbf{Actual Result} & \textbf{Result} \\
  \midrule
  \endhead

  \multicolumn{6}{r}{\textit{Continued on next page}} \\
  \endfoot

  \bottomrule
  \caption{Hashing Tools Test Cases}
  \label{table:plugin_hashing_tests}
  \endlastfoot

  TC\testcount & FR10, OER-IAS1 & Document hash has not changed. &
  Does not update workspace storage. & All assertions pass. &
  \cellcolor{green} Pass \\
  \midrule
  TC\testcount & FR10, OER-IAS1 & Document hash has changed. &
  Updates workspace storage. & All assertions pass. & \cellcolor{green} Pass \\
  \midrule
  TC\testcount & FR10, OER-IAS1 & No hash exists for the document. &
  Updates workspace storage. & All assertions pass. & \cellcolor{green} Pass \\
\end{longtable}

\subsubsection{Line Selection Manager Module}
\begin{longtable}{c
    >{\raggedright\arraybackslash}p{1.5cm}
    >{\raggedright\arraybackslash}p{4.5cm}
    >{\raggedright\arraybackslash}p{4cm}
  >{\raggedright\arraybackslash}p{3cm} c}
  \toprule
  \textbf{ID} & \textbf{Ref. Req.} & \textbf{Action} &
  \textbf{Expected Result} & \textbf{Actual Result} & \textbf{Result} \\
  \midrule
  \endfirsthead

  \multicolumn{6}{l}{\textit{(Continued from previous page)}} \\
  \toprule
  \textbf{ID} & \textbf{Ref. Req.} & \textbf{Action} &
  \textbf{Expected Result} & \textbf{Actual Result} & \textbf{Result} \\
  \midrule
  \endhead

  \multicolumn{6}{r}{\textit{Continued on next page}} \\
  \endfoot

  \bottomrule
  \caption{Line Selection Module Test Cases}
  \label{table:line_selection_tests}
  \endlastfoot

  TC\testcount & UHR-EOU1 & Call the `removeLastComment` method after
  adding a comment. & The decoration is removed and no comment
  remains on the line. & The decoration is removed, and no comment
  appears on the selected line. & \cellcolor{green} Pass \\ \midrule
  TC\testcount & UHR-EOU1 & Call `commentLine` method with null
  editor. & The method does not throw an error. & The method does not
  throw an error. & \cellcolor{green} Pass \\ \midrule
  TC\testcount & UHR-EOU1 & Call `commentLine` on a file with no
  detected smells. & No comment is added to the line. & No decoration
  is added, and the line remains unchanged. & \cellcolor{green} Pass \\ \midrule
  TC\testcount & UHR-EOU1 & Call `commentLine` on a file where the
  document hash does not match. & The method does not add a comment
  because the document has changed. & No decoration is added due to
  the document hash mismatch. & \cellcolor{green} Pass \\ \midrule
  TC\testcount & UHR-EOU1 & Call `commentLine` with a multi-line
  selection. & The method returns early without adding a comment. &
  No comment is added to any lines in the selection. &
  \cellcolor{green} Pass \\ \midrule
  TC\testcount & UHR-EOU1 & Call `commentLine` on a line with no
  detected smells. & No comment is added for the line. & No
  decoration is added, and the line remains unchanged. &
  \cellcolor{green} Pass \\ \midrule
  TC\testcount & UHR-EOU1 & Call `commentLine` on a line with a
  single detected smell. & The comment shows the first smell symbol
  without a count. & Comment shows the first smell symbol: `Smell:
  PERF-001`. & \cellcolor{green} Pass \\ \midrule
  TC\testcount & UHR-EOU1 & Call `commentLine` on a line with a
  detected smell. & A comment is added on the selected line in the
  editor showing the detected smell. & Comment added with the correct
  smell symbol and count. & \cellcolor{green} Pass \\ \midrule
  TC\testcount & UHR-EOU1 & Call `commentLine` on a line with
  multiple detected smells. & The comment shows the first smell
  followed by the count of additional smells. & Comment shows `Smell:
  PERF-001 | (+1)`. & \cellcolor{green} Pass \\
\end{longtable}

\subsubsection{Hover Manager Module}
\begin{longtable}{c
    >{\raggedright\arraybackslash}p{1.5cm}
    >{\raggedright\arraybackslash}p{4.5cm}
    >{\raggedright\arraybackslash}p{4cm}
  >{\raggedright\arraybackslash}p{3cm} c}
  \toprule
  \textbf{ID} & \textbf{Ref. Req.} & \textbf{Action} &
  \textbf{Expected Result} & \textbf{Actual Result} & \textbf{Result} \\
  \midrule
  \endfirsthead

  \multicolumn{6}{l}{\textit{(Continued from previous page)}} \\
  \toprule
  \textbf{ID} & \textbf{Ref. Req.} & \textbf{Action} &
  \textbf{Expected Result} & \textbf{Actual Result} & \textbf{Result} \\
  \midrule
  \endhead

  \multicolumn{6}{r}{\textit{Continued on next page}} \\
  \endfoot

  \bottomrule
  \caption{Hover Manager Module Test Cases}
  \label{table:hover_manager_tests}
  \endlastfoot

  TC\testcount & LFR-AP2 & Register hover provider for Python files.
  & Hover provider registered for Python files. & Hover provider is
  registered for Python files. & \cellcolor{green} Pass \\ \midrule
  TC\testcount & LFR-AP2 & Subscribe hover provider. & Hover provider
  subscription registered. & Hover provider subscription registered.
  & \cellcolor{green} Pass \\ \midrule
  TC\testcount & LFR-AP2 & Return hover content with no smells. &
  Returns null for hover content. & Hover content = null. &
  \cellcolor{green} Pass \\ \midrule
  TC\testcount & LFR-AP2, FR2 & Update smells with new data. & Smells
  updated correctly with new data. & Smells are updated correctly
  with new smells data. & \cellcolor{green} Pass \\ \midrule
  TC\testcount & LFR-AP2, FR2 & Update smells correctly. & Smells
  updated with new content. & Current smells content updated to new
  smells content provided.  & \cellcolor{green} Pass \\ \midrule
  TC\testcount & LFR-AP2 & Generate valid hover content. & Generates
  hover content with correct smell information. & Correct and valid
  hover content generated for given smell. & \cellcolor{green} Pass \\ \midrule
  TC\testcount & LFR-AP2 & Register refactor commands. & Both
  commands registered correctly on initialization & Refactor commands
  registered correctly. & \cellcolor{green} Pass \\
\end{longtable}

\subsubsection{Handle Smell Settings Module}

\begin{longtable}{c
    >{\raggedright\arraybackslash}p{1.5cm}
    >{\raggedright\arraybackslash}p{4.5cm}
    >{\raggedright\arraybackslash}p{4cm}
  >{\raggedright\arraybackslash}p{3cm} c}
  \toprule
  \textbf{ID} & \textbf{Ref. Req.} & \textbf{Action} &
  \textbf{Expected Result} & \textbf{Actual Result} & \textbf{Result} \\
  \midrule
  \endfirsthead

  \multicolumn{6}{l}{\textit{(Continued from previous page)}} \\
  \toprule
  \textbf{ID} & \textbf{Ref. Req.} & \textbf{Action} &
  \textbf{Expected Result} & \textbf{Actual Result} & \textbf{Result} \\
  \midrule
  \endhead

  \multicolumn{6}{r}{\textit{Continued on next page}} \\
  \endfoot

  \bottomrule

  \caption{VS Code Settings Management Module Test Cases}
  \label{table:vs_code_settings_tests}
  \endlastfoot

  TC\testcount & FR10, UHR-PSI1 & Test retrieval of enabled smells
  from settings. & Function should return the current enabled smells.
  & All assertions pass. & \cellcolor{green} Pass \\ \midrule
  TC\testcount & FR10, UHR-PSI1 & Test retrieval of enabled smells
  when no settings exist. & Function should return an empty object. &
  All assertions pass. & \cellcolor{green} Pass \\ \midrule
  TC\testcount & FR10, UHR-PSI1, UHR-EOU2 & Test enabling a smell and
  verifying notification. & Notification should be displayed and
  cache wiped. & All assertions pass. & \cellcolor{green} Pass \\ \midrule
  TC\testcount & FR10, UHR-PSI1, UHR-EOU2 & Test disabling a smell
  and verifying notification. & Notification should be displayed and
  cache wiped. & All assertions pass. & \cellcolor{green} Pass \\ \midrule
  TC\testcount & FR10, UHR-PSI1, UHR-EOU2 & Test that cache is not
  wiped if no changes occur. & No notification or cache wipe should
  happen. & All assertions pass. & \cellcolor{green} Pass \\ \midrule
  TC\testcount & FR10, UHR-PSI1 & Test formatting of kebab-case smell
  names. & Smell names should be correctly converted to readable
  format. & All assertions pass. & \cellcolor{green} Pass \\ \midrule
  TC\testcount & FR10, UHR-PSI1 & Test formatting with an empty
  string input. & Function should return an empty string without
  errors. & All assertions pass. & \cellcolor{green} Pass \\

\end{longtable}

\subsubsection{Handle Smell Settings Module}

\subsubsection{Wipe Workspace Cache Command}

\begin{longtable}{c
    >{\raggedright\arraybackslash}p{1.5cm}
    >{\raggedright\arraybackslash}p{4.5cm}
    >{\raggedright\arraybackslash}p{4cm}
  >{\raggedright\arraybackslash}p{3cm} c}
  \toprule
  \textbf{ID} & \textbf{Ref. Req.} & \textbf{Action} &
  \textbf{Expected Result} & \textbf{Actual Result} & \textbf{Result} \\
  \midrule
  \endfirsthead

  \multicolumn{6}{l}{\textit{(Continued from previous page)}} \\
  \toprule
  \textbf{ID} & \textbf{Ref. Req.} & \textbf{Action} &
  \textbf{Expected Result} & \textbf{Actual Result} & \textbf{Result} \\
  \midrule
  \endhead

  \multicolumn{6}{r}{\textit{Continued on next page}} \\
  \endfoot

  \bottomrule
  \caption{Wipe Workspace Cache Command Test Cases}
  \label{table:plugin_wipe_cache_tests}
  \endlastfoot

  TC\testcount & FR5, FR8 & Trigger the cache wipe with no reason
  provided. & The smells cache should be cleared and reset to an
  empty state. A success message indicating that the workspace cache
  was successfully wiped should be displayed. & All assertions pass.
  & \cellcolor{green} Pass \\
  \midrule
  TC\testcount & FR5, FR8 & Trigger the cache wipe with the reason
  ``manual''. & Both the smells cache and file changes cache should
  be cleared and reset to empty states. A success message indicating
  that the workspace cache was manually wiped by the user should be
  displayed. & All assertions pass. & \cellcolor{green} Pass \\
  \midrule
  TC\testcount & FR5, FR8 & Trigger the cache wipe when there are no
  open files. & A log message indicating that there are no open files
  to update should be generated. & All assertions pass. &
  \cellcolor{green} Pass \\
  \midrule
  TC\testcount & FR5, FR8 & Trigger the cache wipe when there are
  open files. & A message indicating the number of visible files
  should be logged, and the hashes for each open file should be
  updated. & All assertions pass. & \cellcolor{green} Pass \\
  \midrule
  TC\testcount & FR5, FR8 & Trigger the cache wipe with the reason
  ``settings''. & Only the smells cache should be cleared. A success
  message indicating that the cache was wiped due to smell detection
  settings changes should be displayed. & All assertions pass. &
  \cellcolor{green} Pass \\
  \midrule
  TC\testcount & FR3, FR5, FR8 & Trigger the cache wipe when an error
  occurs. & An error message should be logged, and an error message
  indicating failure to wipe the workspace cache should be displayed
  to the user. & All assertions pass. & \cellcolor{green} Pass \\

\end{longtable}

\subsubsection{Backend}

\begin{longtable}{c
    >{\raggedright\arraybackslash}p{1.5cm}
    >{\raggedright\arraybackslash}p{4.5cm}
    >{\raggedright\arraybackslash}p{4cm}
  >{\raggedright\arraybackslash}p{3cm} c}
  \toprule
  \textbf{ID} & \textbf{Ref. Req.} & \textbf{Action} &
  \textbf{Expected Result} & \textbf{Actual Result} & \textbf{Result} \\
  \midrule
  \endfirsthead

  \multicolumn{6}{l}{\textit{(Continued from previous page)}} \\
  \toprule
  \textbf{ID} & \textbf{Ref. Req.} & \textbf{Action} &
  \textbf{Expected Result} & \textbf{Actual Result} & \textbf{Result} \\
  \midrule
  \endhead

  \multicolumn{6}{r}{\textit{Continued on next page}} \\
  \endfoot

  \bottomrule
  \caption{Backend Test Cases}
  \label{table:plugin_backend_tests}
  \endlastfoot

  TC\testcount & PR-SCR1, PR-RFT1 & Trigger request to check server
  status when server responds with a successful status. & The set
  status method should be called with the
  \texttt{ServerStatusType.UP} status & All assertions pass. &
  \cellcolor{green} Pass \\
  \midrule
  TC\testcount & PR-SCR1, PR-RFT2 & Trigger request to check server
  status when server responds with an error status or fails to
  respond. & The set status method should be called with the
  \texttt{ServerStatusType.DOWN} status & All assertions pass. &
  \cellcolor{green} Pass \\
  \midrule
  TC\testcount & FR-6, PR-RFT1 & Trigger intialize logs call with a
  valid directory path, and backend responds with success. & The
  function should return \texttt{true} indicating successful log
  initialization. & All assertions pass. & \cellcolor{green} Pass \\
  \midrule
  TC\testcount & PR-SCR1, PR-RFT2 & Trigger intialize logs call with
  a valid directory path, and backend responds with a failure. & TThe
  function should return \texttt{false} indicating failure to
  initialize logs. & All assertions pass. & \cellcolor{green} Pass \\
\end{longtable}

\section{Changes Due to Testing}

\wss{This section should highlight how feedback from the users and from
  the supervisor (when one exists) shaped the final product.  In particular
  the feedback from the Rev 0 demo to the supervisor (or to potential users)
should be highlighted.}

During the testing phase, several changes were made to the tool based
on feedback from user testing, supervisor reviews, and edge cases
encountered during unit and integration testing. These changes were
necessary to improve the tool’s usability, functionality, and robustness.

\subsection{Usability and User Input Adjustments}
One of the key findings from testing was the balance between
\textbf{automating refactorings} and \textbf{allowing user control}
over changes. Initially, the tool required users to manually approve
every refactoring, which slowed down the workflow. However, after
usability testing, it became evident that an \textbf{option to
refactor all occurrences of the same smell type} would significantly
improve efficiency. This led to the introduction of a
\textbf{"Refactor Smell of Same Type"} feature in the VS Code
extension, allowing users to apply the same refactoring across
multiple instances of a detected smell simultaneously. Additionally,
we refined the \textbf{Accept/Reject UI elements} to make them more
intuitive and streamlined the workflow for batch refactoring actions.

\subsection{Detection and Refactoring Improvements}
Heavy modifications were made to the \textbf{detection and
refactoring modules}, particularly in handling \textbf{multi-file
projects}. Initially, the detectors and refactorers assumed a
\textbf{single-file scope}, leading to missed optimizations when
function calls or variable dependencies spanned across multiple
files. After extensive testing, the detection system was updated to
track \textbf{cross-file dependencies}, ensuring that refactoring
suggestions accounted for the broader codebase.

\subsection{VS Code Extension Enhancements}
Through usability testing, it became apparent that
\textbf{integrating the tool as a VS Code extension} was a
significant improvement over a standalone CLI tool. This led to the
following enhancements:
\begin{itemize}
  \item \textbf{Enhanced Hover Tooltips} – Descriptions for detected
    code smells were rewritten to be clearer and more informative.
  \item \textbf{Smell Filtering Options} – Users can now enable or
    disable specific code smell detections directly from the VS Code
    settings menu.
\end{itemize}

\subsection{Future Revisions and Remaining Work}
Certain features, including \textbf{report generation and full
documentation availability}, have yet to be fully implemented. These
components will be finalized in \textbf{Revision 1}, where testing
will ensure that:
\begin{itemize}
  \item The \textbf{reporting system correctly logs detected smells,
    applied refactorings, and energy savings}.
  \item The \textbf{documentation includes detailed installation,
    usage, and troubleshooting guides}.
\end{itemize}

Additionally, once all features are complete, the \textbf{VS Code
extension will be packaged and tested as a full release} to ensure
seamless installation via the VS Code Marketplace.

\bigskip
Overall, the testing phase played a crucial role in refining the
tool’s functionality, optimizing performance, and improving
usability. The feedback gathered led to meaningful changes that
enhance both the developer experience and the effectiveness of
automated refactoring.

\section{Automated Testing}

All test for the Python backend as well as the individual modules on
the TypeScript side (for the VSCode extension) are automated. The
Python tests are run using Pytest by simply typing \texttt{pytest} in
the command line in the root project directory. All the Typescript
tests can be run similarly, though they run with Jest and through the
command \texttt{npm run test}. The results for both are printed to the console.

\section{Trace to Requirements}

\section{Trace to Modules}

\section{Code Coverage Metrics}

The following analyzes the code coverage metrics for the Python
backend and frontend (TypeScript) of the VSCode extension. The
analysis is based on the coverage data provided in Figure
\ref{img:python-cov} (Python backend) and Figure \ref{img:vscode-cov}
(frontend). Code coverage is a measure of how well the codebase is
tested, and it helps identify areas that may require additional testing.

\begin{figure}[H]
  \centering
  \includegraphics[width=0.7\textwidth]{../Images/python-coverage.png}
  \caption{Coverage Report of the Python Backend Library}
  \label{img:python-cov}
\end{figure}

\begin{figure}[H]
  \centering
  \includegraphics[width=0.7\textwidth]{../Images/vscode-coverage.png}
  \caption{Coverage Report of the VSCode Extension}
  \label{img:vscode-cov}
\end{figure}

\subsection{VSCode Extension}
The frontend codebase has an overall coverage of 45.43\% for
statements, 36.48\% for branches, 42.62\% for functions, and 45.53\%
for lines (Figure \ref{img:vscode-cov}). These metrics fall below the
global coverage thresholds of 80\% for the following reasons. The
file \texttt{extension.ts}, which contains the core logic for the
VSCode extension, has 0\% coverage as it is mainly made up of
initialization commands with no real logic that can be tested. The
file \texttt{refactorView.ts}, responsible for the refactoring view,
also has 0\% coverage. This module is a UI component and will be
tested for revision 1. Since \texttt{handleEditorChange.ts} is
closely related to the UI component, its testing has also been put off.\\

The file \texttt{refactorSmell.ts} has moderate coverage (55.37\%
statements, 45.23\% branches), with significant gaps in testing
around lines 143–269 and 328–337 (Figure \ref{img:vscode-cov}). This
is due to a feature that is not fully implemented and therefore not
tested. Finally, \texttt{configManager.ts} has not been tested as yet
due to evolving configuration options, but will be tested for revision 1.

\subsection{Python Backend}
The backend codebase has an overall coverage of 91\% (Figure
\ref{img:python-cov}) and has been thoroughly tested as it contains
the key features of project and the bulk of the logic. The exception
is \texttt{show\_logs.py}, which handles the websocket endpoint for
logging, due to the complex nature of this module testing has been
omitted. Since its function is mainly to broadcast logs it is also
relatively simple to verify its functionality manually\\

\newpage
\section*{Appendix A -- Usability Testing Data} \label{appendix:usability}

\subsection*{Protocol}

\section*{Purpose}
The purpose of this usability test is to evaluate the ease of use,
efficiency, and overall user experience of the VSCode extension for
refactoring Python code to improve energy efficiency. The test will
identify usability issues that may hinder adoption by software developers.

\section*{Objective}
Evaluate the usability of the extension’s \textbf{smell detection},
\textbf{refactoring process}, \textbf{customization settings}, and
\textbf{refactoring view}.

\begin{itemize}
  \item Assess how easily developers can navigate the extension interface.
  \item Measure the efficiency of the workflow when applying or
    rejecting refactorings.
  \item Identify areas of confusion or frustration.
\end{itemize}

\section*{Methodology}
\subsection*{Test Type}
Moderated usability testing.

\subsection*{Participants}
\begin{itemize}
  \item \textbf{Target Users:} Python developers who use VSCode.
  \item \textbf{Number of Participants:} 5–7.
  \item \textbf{Recruitment Criteria:}
    \begin{itemize}
      \item Experience with Python development.
      \item Familiarity with VSCode.
      \item No prior experience with this extension.
    \end{itemize}
\end{itemize}

\subsection*{Testing Environment}
\begin{itemize}
  \item \textbf{Hardware:} Provided computer.
  \item \textbf{Software:}
    \begin{itemize}
      \item VSCode (latest stable release).
      \item The VSCode extension installed.
      \item Screen recording software (optional, for post-test analysis).
      \item A sample project with \textbf{predefined code snippets}
        containing various \textbf{code smells}.
    \end{itemize}
  \item \textbf{Network Requirements:} Stable internet connection for
    remote testing.
\end{itemize}

\subsection*{Test Moderator Role}
\begin{itemize}
  \item Introduce the test and explain objectives.
  \item Observe user interactions without providing assistance unless necessary.
  \item Take notes on usability issues, pain points, and confusion.
  \item Ask follow-up questions after each task.
  \item Encourage participants to \textbf{think aloud}.
\end{itemize}

\section*{Data Collection}
\subsection*{Metrics}
\begin{itemize}
  \item \textbf{Task Success Rate:} Percentage of users who complete
    tasks without assistance.
  \item \textbf{Error Rate:} Number of errors or missteps per task.
  \item \textbf{User Satisfaction:} Post-test rating on a scale of 1–5.
\end{itemize}

\subsection*{Qualitative Data}
\begin{itemize}
  \item Observations of confusion, hesitation, or frustration.
  \item Participant comments and feedback.
  \item Follow-up questions about expectations vs. actual experience.
  \item Pre-test survey.
  \item Post-test survey.
\end{itemize}

\section*{Analysis and Reporting}
\begin{itemize}
  \item Identify common pain points and recurring issues.
  \item Categorize usability issues by severity:
    \begin{itemize}
      \item \textbf{Critical:} Blocks users from completing tasks.
      \item \textbf{Major:} Causes significant frustration but has workarounds.
      \item \textbf{Minor:} Slight inconvenience, but doesn’t impact
        core functionality.
    \end{itemize}
  \item Provide recommendations for UI/UX improvements.
  \item Summarize key findings and next steps.
\end{itemize}

\section*{Next Steps}
\begin{itemize}
  \item Fix major usability issues before release.
  \item Conduct follow-up usability tests if significant changes are made.
  \item Gather further feedback from real users post-release.
\end{itemize}

\subsection*{Task List}

\section*{Mock Installation Documentation}
The extension can be installed to detect energy inefficiencies
(smells) in your code and refactor them.

\subsection*{Commands}
Open the VSCode command palette (\texttt{CTRL+SHIFT+P}):

\begin{itemize}
  \item \textbf{Detect Smells:} \texttt{Eco: Detect Smells}
  \item \textbf{Refactor Smells:} \texttt{Eco: Refactor Smell} or
    \texttt{CTRL+SHIFT+R} (or to be discovered).
\end{itemize}

\section*{Tasks}
Report your observations \textbf{aloud}!

\subsection*{Task 1: Smell Detection}
\begin{enumerate}
  \item Open the \texttt{sample.py} file.
  \item Detect the smells in the file.
  \item What do you see?
\end{enumerate}

\subsection*{Task 2: Line Selection}
\begin{enumerate}
  \item In the same \texttt{sample.py} file, select one of the
    highlighted lines.
  \item What do you see?
  \item Select another line.
\end{enumerate}

\subsection*{Task 3: Hover}
\begin{enumerate}
  \item In the same file, hover over a highlighted line.
  \item What do you see?
\end{enumerate}

\subsection*{Task 4: Initiate Refactoring (Single)}
\begin{enumerate}
  \item In the same file, refactor any smell of your choice.
  \item What do you observe immediately after?
  \item Does a sidebar pop up after some time?
\end{enumerate}

\subsection*{Task 5: Refactor Smell (Sidebar)}
\begin{enumerate}
  \item What information do you see in the sidebar?
  \item Do you understand the information communicated?
  \item Do you see what was changed in the file?
  \item Try rejecting a smell. Did the file change?
  \item Repeat Tasks 1, 4, and 5, but reject a smell. Did the file
    stay the same?
\end{enumerate}

\subsection*{Task 6: Refactor Multi-File Smell}
\begin{enumerate}
  \item Open the \texttt{main.py} file.
  \item Detect the smells in the file.
  \item Refactor any smell of your choice.
  \item Do you see anything different in the sidebar?
  \item Try clicking on the new addition to the sidebar. Notice anything?
  \item Try accepting the refactoring. Did both files change?
\end{enumerate}

\subsection*{Task 7: Change Smell Settings}
\begin{enumerate}
  \item Open the \texttt{sample.py} file.
  \item Detect the smells in the file.
  \item Take note of the smells detected.
  \item Open the settings page (\texttt{CTRL+,}).
  \item Navigate to the \textbf{Extensions} drop-down and select
    \textbf{Eco Optimizer}.
  \item Unselect one of the smells you noticed earlier.
  \item Navigate back to the \texttt{sample.py} file.
  \item Detect the smells again. Is the smell you unselected still there?
\end{enumerate}

\subsection*{Participant Data}
The following links point to the data collected from each participant:\\

{\noindent
  \href{run:./../Extras/UsabilityTesting/test_data/participant1-data.csv}{Participant
  1} \\[2mm]
  \href{run:./../Extras/UsabilityTesting/test_data/participant2-data.csv}{Participant
  2} \\[2mm]
  \href{run:./../Extras/UsabilityTesting/test_data/participant3-data.csv}{Participant
  3} \\[2mm]
  \href{run:./../Extras/UsabilityTesting/test_data/participant4-data.csv}{Participant
  4} \\[2mm]
  \href{run:./../Extras/UsabilityTesting/test_data/participant5-data.csv}{Participant
  5}
}

\subsection*{Pre-Test Survey Data}
The following link points to a CSV file containing the pre-survey data:\\

\noindent
\href{run:./../Extras/UsabilityTesting/surveys/pre-test-survey-data.csv}{Click
here to access the survey results CSV file}.

\subsection*{Post-Test Survey Data}
The following link points to a CSV file containing the post-survey data:\\

\noindent
\href{run:./../Extras/UsabilityTesting/surveys/post-test-survey-data.csv}{Click
here to access the survey results CSV file}.

\newpage{}
\section*{Appendix --- Reflection}

The information in this section will be used to evaluate the team members on the
graduate attribute of Reflection.

\input{../Reflection.tex}

\begin{enumerate}
  \item What went well while writing this deliverable?
  \item What pain points did you experience during this deliverable, and how
    did you resolve them?
  \item Which parts of this document stemmed from speaking to your client(s) or
    a proxy (e.g. your peers)? Which ones were not, and why?
  \item In what ways was the Verification and Validation (VnV) Plan different
    from the activities that were actually conducted for VnV?  If there were
    differences, what changes required the modification in the plan?  Why did
    these changes occur?  Would you be able to anticipate these
    changes in future
    projects?  If there weren't any differences, how was your team
    able to clearly
    predict a feasible amount of effort and the right tasks needed to build the
    evidence that demonstrates the required quality?  (It is expected that most
    teams will have had to deviate from their original VnV Plan.)
\end{enumerate}

\subsubsection*{Mya Hussain}
\begin{itemize}
  \item \textit{What went well while writing this deliverable?} \\

    One of the most rewarding parts of completing this report was
    writing and compiling
    the benchmarking and performance analysis. Seeing the data come
    to life through plots
    and visualizations was very satisfying as we could see the
    underlying patterns we
    knew existed in our code in a visual format. It also outted
    everyones performance
    on their corresponding refactorers which was cool. Overall, the
    whole process of turning raw data
    into meaningful insights was really fulfilling.  It felt like I
    was uncovering useful information that
    could really help improve the tool, which made the effort feel worthwhile

  \item \textit{What pain points did you experience during this
    deliverable, and how did you resolve them?}\\

    The biggest pain point for me was definitely the sheer amount of
    unit testing that had to
    be done before even starting the report. Writing all those tests
    and making sure everything
    worked as expected was a lot of legwork, it felt like I was stuck
    in an endless loop of running
    tests, fixing bugs, and then running more tests. It was necessary
    but not the most exciting part of
    the process. The tricky part was making sure the report actually
    reflected all that effort. As we spent hours
    testing, and finding bugs, and fixing them, so the tool is a lot
    better, but logging all of those fixes without
    1. sounding like the tool was broken to start and 2. overselling
    all the trivial tests we felt like
    we had to do to achive coverage, was a challenge.

\end{itemize}

\subsubsection*{Sevhena Walker}
\begin{itemize}
  \item \textit{What went well while writing this deliverable?} \\

    A big win was how much of our work naturally fed into the report.
    Since we had already been refining our verification and
    validation (V\&V) process throughout development, we weren’t
    starting from scratch, we just had to document what we had done.
    Having clear test cases in place made it easier to describe our
    approach and results, rather than writing purely in the abstract.
    Another positive was that our understanding of the system had
    improved significantly by this point, so explaining our reasoning
    behind certain tests felt more natural.

  \item \textit{What pain points did you experience during this
    deliverable, and how did you resolve them?}\\

    One challenge was finalizing our tests while also writing about
    them. Since we were still adjusting some test cases, we had to
    ensure that any changes were reflected correctly in the report,
    which meant some back-and-forth edits. Another issue was
    balancing detail; some sections needed more explanation than
    expected, while others felt overly technical. We resolved this by
    reviewing each section with fresh eyes and making sure we
    explained things clearly without unnecessary complexity. Time was
    also a factor, as wrapping up both testing and documentation at
    the same time was a bit hectic. We managed by setting smaller
    milestones to keep things on track and making sure to check in
    regularly to avoid last-minute rushes.
\end{itemize}

\subsubsection*{Ayushi Amin}
\begin{itemize}
  \item \textit{What went well while writing this deliverable?} \\

    One of the best parts of working on this deliverable was how well
    my team collaborated. We had a clear understanding of what needed
    to be covered, which made it easier to organize our thoughts and
    avoid unnecessary back-and-forth. Writing about our unit tests
    was also pretty smooth since we had already put a lot of effort
    into designing them in the vnv-plan.
    It was satisfying to document the thought process behind them,
    especially since they played a big role in making sure the tool
    was accurate and functioned correctly.
    Another thing I really enjoyed was usability testing. It was fun
    to see how others interacted with our tool and to get real
    feedback on what worked and what did not. Seeing users struggle
    with certain parts that we thought were intuitive was interesting
    to find out, but it also made the process more rewarding because
    we could make meaningful improvements.

  \item \textit{What pain points did you experience during this
    deliverable, and how did you resolve them?}\\

    One pain point I experienced was structering the unit tests
    report and tracing back to the VnV plan tests. This is because
    The samples we had were really all over the place and not consistent at all.
    It was difficult to know what information was required for
    certian portions when most of the samples did not cover some
    portions. Also tracing back to VnV Plan tests, I realized that
    that some tests were not
    feasible and it would make no sense to do them for this project.
    Not entirely sure what we were thinking when we wrote them. So we
    decided to modify our VnV plan to be more realistic with the time
    frame we have
    and since a lot was changed in the scope of this project, we
    removed certain tests to better suit our current project.

\end{itemize}

\subsubsection*{Nivetha Kuruparan}
\begin{itemize}
  \item \textit{What went well while writing this deliverable?}

    One of the things that went well while working on this
    deliverable was our ability to catch a significant number of bugs
    and edge cases during testing. Through extensive unit and
    integration testing, we identified multiple issues related to
    multi-file refactoring, detection accuracy, and performance
    optimization. This allowed us to refine our detection and
    refactoring mechanisms, making them more reliable and robust.

  \item \textit{What pain points did you experience during this
    deliverable, and how did you resolve them?}

    One of the biggest challenges we faced was the overwhelming
    number of tests outlined in the original V\&V Plan. While
    comprehensive, implementing every test and writing detailed
    reports for each became highly time-consuming and impractical. As
    a result, we had to carefully trim down and consolidate tests to
    focus on the most critical functionalities while still
    maintaining full coverage of our system requirements. This
    process involved combining similar tests and prioritizing cases
    that had the most significant impact on correctness, usability,
    and performance. While this required careful review and
    restructuring, it ultimately streamlined the validation process
    and improved efficiency in writing the report.
\end{itemize}

\subsubsection*{Tanveer}
\begin{itemize}
  \item \textit{What went well while writing this deliverable?} \\

    The fun part was validating different requirements that we had
    defined in the VnV Plan against our tool. I saw that some of them
    were too ambitious versus others could have more points added for
    the verification. Overall, it was fun mapping non functional
    requirements against the features of the tool. At the end of it,
    I was able to deduce which NFR maps to a certain feature of the tool.

  \item \textit{What pain points did you experience during this
    deliverable, and how did you resolve them?}\\

    Writing unit tests turned out to be harder than actual
    implementation because (1) not only did I come across bugs when
    testing but also (2) mocking dependencies such as
    \texttt{vscode.workspace} for our plugin was definitely a
    learning curve. It is important to mentiont that I don't believe
    that the course and capstone would have been the same as testing,
    the team was testing left and right to get the maximum coverage.
    To resolve the learning curve I referred to multiple tutorials
    online and eventually the process became getting rid of the
    syntax errors or bugs in the unit test implementation so that the
    tests could pass.
\end{itemize}

\subsubsection*{Group}
\begin{itemize}
  \item \textit{Which parts of this document stemmed from speaking to
      your client(s) or
    a proxy (e.g. your peers)? Which ones were not, and why?} \\

    Parts of this document stemmed from speaking to users who acted
    as proxies for clients. Specifically:
    \begin{itemize}
      \item \textbf{Usability Testing Findings:} The document details
        usability testing conducted with student developers, who
        served as proxies for real-world users. Their feedback on
        sidebar visibility, refactoring speed, UI clarity, and energy
        savings feedback directly influenced the report.
      \item \textbf{Methodology and Results:} The task completion
        rates, user satisfaction scores, and qualitative insights
        were derived from these interactions, making them user-driven.
      \item \textbf{Non-functional Requirements:} This is based on
        client as some requirements like look and feel is evaluated
        by client and usability testers since they will be the ones
        using the application.
    \end{itemize}

    Parts of the document that did not stem from client or proxy
    interactions include:
    \begin{itemize}

      \item \textbf{Functional Requirements Evaluations:} These
        sections reference predefined specifications/industry
        standards rather than direct client input.
      \item \textbf{Implementation and Technical Explanations:} These
        were formulated based on the development team’s decisions,
        software documentation, and prior knowledge rather than
        external feedback.

    \end{itemize}


  \item \textit{In what ways was the Verification and Validation
      (VnV) Plan different
      from the activities that were actually conducted for VnV?  If there were
      differences, what changes required the modification in the plan?  Why did
      these changes occur?  Would you be able to anticipate these
      changes in future
      projects?  If there weren't any differences, how was your team
      able to clearly
      predict a feasible amount of effort and the right tasks needed
      to build the
      evidence that demonstrates the required quality?  (It is
        expected that most
    teams will have had to deviate from their original VnV Plan.)}\\

    There were definitely some differences between what we assumed
    would happen during
    the VnV Plan and the results we actually got during testing. For
    example, the plan initially assumed that
    energy measurement times would vary significantly with file size, but the
    testing revealed that they were actually decently consistent. This meant we
    had to adjust our focus in the report to highlight the fixed
    overhead of energy
    measurement rather than exploring variability.

    For the most part, though, all of the unit testing we planned in
    the VnV Plan was
    written out per spec, and the code was fixed until all of them passed. This
    rigorous testing process actually caught a lot of bugs and edge
    cases that we
    hadn't fully anticipated in the plan. For instance, some
    refactoring operations
    worked fine on smaller files but broke on larger ones. Testing
    also revealed edge cases,
    like how the tool handled files with \textbf{multiline
    whitespace}, \textbf{nested structures},
    \textbf{degenerate/trivial input} (e.g., empty files or files
    with a single line), and
    \textbf{wrong input} (e.g., malformed code or unsupported
    syntax). These cases weren't
    explicitly called out in the original plan, but they became a big
    part of the testing process
    once we realized how critical they were to the tool's reliability.

    For example:
    \begin{itemize}
      \item \textbf{Multiline whitespace:} The tool initially
        struggled with files that had excessive or irregular
        whitespace, which caused false positives in code smell
        detection. We had to update the detection logic to handle
        these cases gracefully.
      \item \textbf{Nested structures:} Deeply nested code (e.g.,
        loops within loops or functions within functions) exposed
        performance bottlenecks and sometimes caused the tool to
        crash. This led to optimizations in the refactoring algorithms.
      \item \textbf{Degenerate/trivial input:} Empty files or files
        with minimal content revealed that some refactoring
        operations weren't properly handling edge cases, so we added
        checks to ensure the tool behaved correctly in these scenarios.
      \item \textbf{Wrong input:} Malformed or unsupported code
        caused unexpected errors, so we improved error handling and
        added clearer feedback for users.
    \end{itemize}

    Fixing these issues required additional effort, but it ultimately
    made the tool more robust and user-friendly.


    These changes happened because testing revealed patterns in the data
    and uncovered bugs that weren't obvious during the planning phase.
    The bugs and edge cases we found during testing forced us to
    revisit parts of the code and make improvements we hadn't planned
    for initially.

    Some of these changes could be anticipated in future projects
    with more thorough initial testing.
    If i could do it again I'd build more flexibility into the VnV
    Plan to account for unexpected results and allocate extra time
    for debugging and edge-case testing. I’d also include a broader
    range of test cases (e.g., multiline whitespace, wrong input) in
    the initial plan to catch these issues sooner.

\end{itemize}

\end{document}
