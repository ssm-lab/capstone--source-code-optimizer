\documentclass[12pt, titlepage]{article}

\usepackage{booktabs}
\usepackage{tabularx}
\usepackage{hyperref}
\hypersetup{
    colorlinks,
    citecolor=black,
    filecolor=black,
    linkcolor=red,
    urlcolor=blue
}
\usepackage[round]{natbib}

%% Comments

\usepackage{color}

% \newif\ifcomments\commentstrue %displays comments
\newif\ifcomments\commentsfalse %so that comments do not display

\ifcomments
\newcommand{\authornote}[3]{\textcolor{#1}{[#3 ---#2]}}
\newcommand{\todo}[1]{\textcolor{red}{[TODO: #1]}}
\else
\newcommand{\authornote}[3]{}
\newcommand{\todo}[1]{}
\fi

\newcommand{\wss}[1]{\authornote{blue}{SS}{#1}} 
\newcommand{\plt}[1]{\authornote{magenta}{TPLT}{#1}} %For explanation of the template
\newcommand{\an}[1]{\authornote{cyan}{Author}{#1}}

%% Common Parts

\newcommand{\progname}{Software Engineering} % PUT YOUR PROGRAM NAME HERE
\newcommand{\authname}{\textbf{Team 4, EcoOptimizers} \\
  \\ Nivetha Kuruparan
  \\ Sevhena Walker
  \\ Tanveer Brar
  \\ Mya Hussain
\\ Ayushi Amin} % AUTHOR NAMES

\usepackage{hyperref}
\hypersetup{colorlinks=true, linkcolor=blue, citecolor=blue, filecolor=blue,
urlcolor=blue, unicode=false}
\urlstyle{same}



\begin{document}

\title{Verification and Validation Report: \progname} 
\author{\authname}
\date{\today}
	
\maketitle

\pagenumbering{roman}

\section{Revision History}

\begin{tabularx}{\textwidth}{p{3cm}p{2cm}X}
\toprule {\bf Date} & {\bf Version} & {\bf Notes}\\
\midrule
March 8th, 2025 & 0.0 & Started VnV Report\\
Date 2 & 1.1 & Notes\\
\bottomrule
\end{tabularx}

~\newpage

\section{Symbols, Abbreviations and Acronyms}

\renewcommand{\arraystretch}{1.2}
\begin{tabular}{l l} 
  \toprule		
  \textbf{symbol} & \textbf{description}\\
  \midrule 
  T & Test\\
  \bottomrule
\end{tabular}\\

\wss{symbols, abbreviations or acronyms -- you can reference the SRS tables if needed}

\newpage

\tableofcontents

\listoftables %if appropriate

\listoffigures %if appropriate

\newpage

\pagenumbering{arabic}

This document ...

\section{Functional Requirements Evaluation}

\section{Nonfunctional Requirements Evaluation}

\subsection{Usability}
		
\subsection{Performance}

\subsection{etc.}
	
\section{Comparison to Existing Implementation}	

This section will not be appropriate for every project.

\section{Unit Testing}

\subsection{Line Selection Manager Module}
\begin{table}[h!]
\centering
\begin{tabular}{|c|p{2.5cm}|p{3cm}|p{3.5cm}|p{3.5cm}|c|}
\hline
\textbf{ID} & \textbf{Reference Requirement} & \textbf{Action} & \textbf{Expected Result} & \textbf{Actual Result} & \textbf{Result} \\ \hline
LSM1 & UHR-EOU1 & Call the `removeLastComment` method after adding a comment. & The decoration is removed and no comment remains on the line. & The decoration is removed, and no comment appears on the selected line. & \cellcolor{green!20} Pass \\ \hline
LSM2 & UHR-EOU1 & Call `commentLine` method with null editor. & The method does not throw an error. & The method does not throw an error. & \cellcolor{green!20} Pass \\ \hline
LSM3 & UHR-EOU1 & Call `commentLine` on a file with no detected smells. & No comment is added to the line. & No decoration is added, and the line remains unchanged. & \cellcolor{green!20} Pass \\ \hline
LSM4 & UHR-EOU1 & Call `commentLine` on a file where the document hash does not match. & The method does not add a comment because the document has changed. & No decoration is added due to the document hash mismatch. & \cellcolor{green!20} Pass \\ \hline
LSM5 & UHR-EOU1 & Call `commentLine` with a multi-line selection. & The method returns early without adding a comment. & No comment is added to any lines in the selection. & \cellcolor{green!20} Pass \\ \hline
LSM6 & UHR-EOU1 & Call `commentLine` on a line with no detected smells. & No comment is added for the line. & No decoration is added, and the line remains unchanged. & \cellcolor{green!20} Pass \\ \hline
LSM7 & UHR-EOU1 & Call `commentLine` on a line with a single detected smell. & The comment shows the first smell symbol without a count. & Comment shows the first smell symbol: `Smell: PERF-001`. & \cellcolor{green!20} Pass \\ \hline
LSM8 & UHR-EOU1 & Call `commentLine` on a line with a detected smell. & A comment is added on the selected line in the editor showing the detected smell. & Comment added with the correct smell symbol and count. & \cellcolor{green!20} Pass \\ \hline
LSM9 & UHR-EOU1 & Call `commentLine` on a line with multiple detected smells. & The comment shows the first smell followed by the count of additional smells. & Comment shows `Smell: PERF-001 | (+1)`. & \cellcolor{green!20} Pass \\ \hline
\end{tabular}
\caption{Line Selection Module Test Cases}
\label{table:line_selection_tests}
\end{table}

\subsection{Hover Manager Module}
\begin{table}[h!]
\centering
\begin{tabular}{|c|p{2.5cm}|p{3cm}|p{3.5cm}|p{3.5cm}|c|}
\hline
\textbf{ID} & \textbf{Reference Requirement} & \textbf{Action} & \textbf{Expected Result} & \textbf{Actual Result} & \textbf{Result} \\ \hline

HM1 & LFR-AP2 & Register hover provider for Python files. & Hover provider registered for Python files. & Hover provider is registered for Python files. & \cellcolor{green!20} Pass \\ \hline
HM2 & LFR-AP2 & Subscribe hover provider. & Hover provider subscription registered. & Hover provider subscription registered. & \cellcolor{green!20} Pass \\ \hline
HM3 & LFR-AP2 & Return hover content with no smells. & Returns null for hover content. & Hover content = null. & \cellcolor{green!20} Pass \\ \hline
HM4 & LFR-AP2, FR2 & Update smells with new data. & Smells updated correctly with new data. & Smells are updated correctly with new smells data. & \cellcolor{green!20} Pass \\ \hline
HM5 & LFR-AP2, FR2 & Update smells correctly. & Smells updated with new content. & Current smells content updated to new smells content provided.  & \cellcolor{green!20} Pass \\ \hline
HM6 & LFR-AP2 & Generate valid hover content. & Generates hover content with correct smell information. & Correct and valid hover content generated for given smell. & \cellcolor{green!20} Pass \\ \hline
HM7 & LFR-AP2 & Register refactor commands. & Both commands registered correctly on initialization & Refactor commands registered correctly. & \cellcolor{green!20} Pass \\ \hline

\end{tabular}
\caption{Hover Manager Module Test Cases}
\label{table:hover_manager_tests}
\end{table}


\subsection{Refactorer Controller Module}

\renewcommand{\arraystretch}{1.2} % Adjust row height for better readability

\begin{longtable}{|c|p{2.5cm}|p{2cm}|p{4cm}|p{4cm}|c|}
\hline
\textbf{ID} & \textbf{Reference Requirement} & \textbf{Action} & \textbf{Expected Result} & \textbf{Actual Result} & \textbf{Result} \\ \hline
\endfirsthead

\multicolumn{6}{c}{\textit{(Continued from previous page)}} \\ \hline
\textbf{ID} & \textbf{Reference Requirement} & \textbf{Action} & \textbf{Expected Result} & \textbf{Actual Result} & \textbf{Result} \\ \hline
\endhead

\hline \multicolumn{6}{|r|}{\textit{Continued on next page}} \\ \hline
\endfoot

\hline
\endlastfoot

RC1 & FR5 & User requests to refactor a smell. & Correct smell is identified. Logger logs ``Running refactoring for long-element-chain using TestRefactorer.'' Correct refactorer is called once with correct arguments. Output path is \texttt{test\_path.LEC001\_1.py}. & All assertions pass. & \cellcolor{green!20} Pass \\ \hline
RC2 & UHR-UPLD1 & System handles missing refactorer. & Raises \texttt{NotImplementedError} with message ``No refactorer implemented for smell: long-element-chain.'' Logger logs error. & All assertions pass. & \cellcolor{green!20} Pass \\ \hline
RC3 & FR5 & Multiple refactorer calls are handled correctly. & Correct smell counter incremented. Refactorer is called twice. First output: \texttt{test\_path.LEC001\_1.py}. Second output: \texttt{test\_path.LEC001\_2.py}. & All assertions pass. & \cellcolor{green!20} Pass \\ \hline
RC4 & FR5 & Refactorer runs with overwrite set to False. & Refactorer is called once. Overwrite argument is set to False. & All assertions pass. & \cellcolor{green!20} Pass \\ \hline
RC5 & PR-RFT 1, FR5 & System handles empty modified files correctly. & Modified files list remains empty (\texttt{[]} in output). & All assertions pass. & \cellcolor{green!20} Pass \\ \hline

\caption{Refactorer Controller Module Test Cases}
\label{table:refactorer_controller_tests}
\end{longtable}


\subsection{Long Element Chain Detector Module}

\begin{longtable}{|c|p{2.5cm}|p{2.5cm}|p{4cm}|p{3cm}|c|}
\hline
\textbf{ID} & \textbf{Reference Requirement} & \textbf{Action} & \textbf{Expected Result} & \textbf{Actual Result} & \textbf{Result} \\ \hline
\endfirsthead

\multicolumn{6}{c}{\textit{(Continued from previous page)}} \\ \hline
\textbf{ID} & \textbf{Reference Requirement} & \textbf{Action} & \textbf{Expected Result} & \textbf{Actual Result} & \textbf{Result} \\ \hline
\endhead

\hline \multicolumn{6}{|r|}{\textit{Continued on next page}} \\ \hline
\endfoot

\hline
\endlastfoot

DLEC1 & FR2 & Test with code that has no chains. & No chains should be detected. & All assertions pass. & \cellcolor{green!20} Pass \\ \hline
DLEC2 & FR2 & Test with chains shorter than threshold. & No chains should be detected for threshold of 5. & All assertions pass. & \cellcolor{green!20} Pass \\ \hline
DLEC3 & FR2  & Test with chains exactly at threshold. & One chain should be detected at line 3. & All assertions pass. & \cellcolor{green!20} Pass \\ \hline
DLEC4 & FR2 & Test with chains longer than threshold. & One chain should be detected with message ``Dictionary chain too long (4/3)''. & All assertions pass. & \cellcolor{green!20} Pass \\ \hline
DLEC5 & FR2 & Test with multiple chains in the same file. & Two chains should be detected at different lines. & All assertions pass. & \cellcolor{green!20} Pass \\ \hline
DLEC6 & FR2 & Test chains inside nested functions and classes. & Two chains should be detected, one inside a function, one inside a class. & All assertions pass. & \cellcolor{green!20} Pass \\ \hline
DLEC7 & FR2 & Test that chains on the same line are reported only once. & One chain should be detected at line 4. & All assertions pass. & \cellcolor{green!20} Pass \\ \hline
DLEC8 & FR2 & Test chains with different variable types. & Two chains should be detected, one in a list and one in a tuple. & All assertions pass. & \cellcolor{green!20} Pass \\ \hline
DLEC9 & FR2 & Test with a custom threshold value. & No chains detected with threshold 4. One chain detected with threshold 2. & All assertions pass. & \cellcolor{green!20} Pass \\ \hline
DLEC10 & FR2 & Test the structure of the returned LECSmell object. & Object should have correct type, path, module, symbol, and occurrence details. & All assertions pass. & \cellcolor{green!20} Pass \\ \hline
DLEC11 & FR2 & Test chains within complex expressions. & Three chains should be detected in different contexts. & All assertions pass. & \cellcolor{green!20} Pass \\ \hline
DLEC12 & FR2 & Test with an empty file. & No chains should be detected. & All assertions pass. & \cellcolor{green!20} Pass \\ \hline
DLEC13 & FR2 & Test with threshold of 1 (every subscript reported). & One chain should be detected with message ``Dictionary chain too long (5/5)''. & All assertions pass. & \cellcolor{green!20} Pass \\ \hline

\caption{Long Element Chain Detector Module Test Cases}
\label{table:lec_tests}
\end{longtable}


\subsection{Long Element Chain Refactorer Module}

\begin{longtable}{|c|p{2.5cm}|p{3cm}|p{4cm}|p{2.5cm}|c|}
\hline
\textbf{ID} & \textbf{Reference Requirement} & \textbf{Action} & \textbf{Expected Result} & \textbf{Actual Result} & \textbf{Result} \\ \hline
\endfirsthead

\multicolumn{6}{c}{\textit{(Continued from previous page)}} \\ \hline
\textbf{ID} & \textbf{Reference Requirement} & \textbf{Action} & \textbf{Expected Result} & \textbf{Actual Result} & \textbf{Result} \\ \hline
\endhead

\hline \multicolumn{6}{|r|}{\textit{Continued on next page}} \\ \hline
\endfoot

\hline
\endlastfoot

LEC1 & PR-PAR3, FR6, FR3 & Test the long element chain refactorer on basic nested dictionary access & Dictionary should be flattened, and access updated & Refactoring applied successfully, dictionary access updated & \cellcolor{green!20} Pass \\ \hline
LEC2 & PR-PAR3, FR6, FR3 & Test the long element chain refactorer across multiple files & Dictionary access across multiple files should be updated & Refactoring applied successfully across multiple files & \cellcolor{yellow!20} TBD \\ \hline
LEC3 & PR-PAR3, FR6, FR3 & Test the refactorer on dictionary access via class attributes & Class attributes should be flattened and access updated & Refactoring applied successfully on class attribute accesses. All accesses changed correctly. & \cellcolor{green!20} Pass \\ \hline
LEC4 & PR-PAR3, FR6, FR3 & Ensure the refactorer skips shallow dictionary access & Refactoring should be skipped for shallow access & Refactoring correctly skipped for shallow access & \cellcolor{green!20} Pass \\ \hline
LEC5 & PR-PAR3, FR6, FR3 & Test the refactorer on dictionary access with mixed depths & Flatten the dictionary up to the minimum access depth & All dictionary access chains flattened to minimum access depth and dictionary flattened successfully. & \cellcolor{green!20} Pass \\ \hline

\caption{Long Element Chain Refactorer Test Cases}
\label{table:lec_refactorer_tests}
\end{longtable}


\section{Changes Due to Testing}

\wss{This section should highlight how feedback from the users and from 
the supervisor (when one exists) shaped the final product.  In particular 
the feedback from the Rev 0 demo to the supervisor (or to potential users) 
should be highlighted.}

\section{Automated Testing}
		
\section{Trace to Requirements}
		
\section{Trace to Modules}		

\section{Code Coverage Metrics}

\bibliographystyle{plainnat}
\bibliography{../../refs/References}

\newpage{}
\section*{Appendix --- Reflection}

The information in this section will be used to evaluate the team members on the
graduate attribute of Reflection.

\input{../Reflection.tex}

\begin{enumerate}
  \item What went well while writing this deliverable? 
  \item What pain points did you experience during this deliverable, and how
    did you resolve them?
  \item Which parts of this document stemmed from speaking to your client(s) or
  a proxy (e.g. your peers)? Which ones were not, and why?
  \item In what ways was the Verification and Validation (VnV) Plan different
  from the activities that were actually conducted for VnV?  If there were
  differences, what changes required the modification in the plan?  Why did
  these changes occur?  Would you be able to anticipate these changes in future
  projects?  If there weren't any differences, how was your team able to clearly
  predict a feasible amount of effort and the right tasks needed to build the
  evidence that demonstrates the required quality?  (It is expected that most
  teams will have had to deviate from their original VnV Plan.)
\end{enumerate}

\end{document}